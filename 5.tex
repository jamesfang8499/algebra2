\chapter{复数}

\section{引言——数系的发展与复数的起源}
数的概念及运算是从生产和科学研究的实践中产生和发
展起来的。截至目前,同学们学过的数系是下表中实线画出
的部分:












正如表中所列出的,从正整数(自然数)集$\N$到实数集
$\R$, 数系已经经历了三次扩充(下表中实箭头表示的部分)。
本章将研究第四次扩充(虚箭头表示的部分):










“温故知新”,让我们先回顾一下前三次扩充的事实和规
律。

\subsection*{1.从数集上的元素看}
数集的每次扩充都是在原有数集的基础上引入新数,构
造出一个新数集,且使原有的数集成为新数集上的一个真子
集。也就是说,数集以\textbf{逐次包含}的方式扩充。

\subsection*{2.从数集上的运算看}
在新数集上我们这样定义新的运算法则:一方面要使新
法则运用到原有数集上时必须与原有数集上的运算结果相一
致;另一方面我们还希望新法则能使原数集上的运算规律保
持不变。例如,在有理数集上计算$(+5)+(+6)$
的结果
应与在算术(正有理数集)中计算$5+6$的结果相一致,而且
我们还希望在算术中对加法和乘法成立的交换律、结合律和
分配律能在有理数集上保持不变。也就是说,在原来数集上
成立的算律,在扩充后的数集上希望它能继续成立。这是很
重要的\textbf{数集扩充的原则}。


\subsection*{3.从数集扩充的动力看}
数集的每次扩充既是度量的需要(这一点大家比较清
楚),又是数学理论发展的需要(特别是解方程的需要)。
例如:

方程$x+4=0$
在自然数集上无解,而在扩充后的整数集
上有唯一解
$x=-4$;

方程$5x=3$,在整数集上无解,而在扩充后的有理数集
上有唯一解
$x=\frac{3}{5}$
(若不扩充数集,连这种最简单的一元一
次方程的理论也不能完整)。

方程$x^2=2$
在有理数集上无解,在扩充后的实数集上有
两个解$x=\pm\sqrt{2}$.

最早呼唤扩充实数集的问题是解二次方程。事实上,在
实数集上不能提供二次方程的完整理论。如方程
\begin{equation}
    x^2=-1 \tag{1}
\end{equation}
在实数集上就没有解,因为任意实数的平方都不可能为负
数。

摆在人们面前有两种选择:要么宣布方程(1)无解;要么
沿着数集扩充的方向朝前走——引入新数,扩充实数集,使
方程(1)有解。一些严峻的事实迫使人们选择了后者。于是在
16世纪中叶,人们开始引入一个由等式
\begin{equation}
    i^2=-1  \tag{2}
\end{equation}
所定义的新数$i$(它显然不是实数)\footnote{$i$是英文单词imaginares(虚的)的字头,瑞士大数学家欧拉首先用它
表示虚数单位。},它被称做虚数单位。
这样,方程(1)至少有了一个根$i$. 进而,根据数集扩充的原
则,我们还规定实数可以和它进行四则运算,并且进行四则
运算时,原有的加、乘算律仍然成立。于是自然造出了诸如
$2i$, $3i$, $-i$, $2+5i$
这样一些新数。

应该指出:引进的新数$i$以及上面造出的形如
$a+bi\; (a,b\in\R)$的数,起初人们对它感到迷惑不解,特别是当$b\ne 0$时,
认为这些都是“虚假的数”[这是因为当时人们习惯于把数作
为某种计数手段,然而形如$a+bi\; (a,
b\in\R, \text{ 且}b\ne 0)$的新数
却起不到计数手段的作用],“虚数”正是由此而得名。

虚数诞生后,除了用来表示方程的解以外,一时尚找不
到更多的应用,因此发展十分缓慢。到了18世纪末叶,维塞
尔(Wessel)、阿尔纲(Argand)和高斯(Gauss)几乎同
时给这些新数以几何解释(见6.2节)以后,才在数学、物
理的应用和研究中逐渐有了越来越多且越来越重要的用途,
这进一步推动了人们研究的兴趣。19世纪初,高斯把形如
$a+bi\; (a,b\in\R)$的数称为\textbf{复数}。如果说微积分的研究统治了
18世纪的话,那么,19世纪数学家们的兴奋中心便是关于复
数理论及其应用的研究。现在复数已经发展成为一门十分重
要的基础数学分支——复变函数论,它是数学、物理和技术
科学中有力的数学工具之一。

本章将学习复数的初步知识。








