\subsection{利用三角形式进行复数的乘方运算}

现在我们研究$n$个复数相乘的问题。

设$z_k=r_k(\cos\theta_k+i\sin\theta_k),\quad k=1,2,3,\ldots,n$。利用数学归纳法容易证明
\[z_1\cdot z_2\cdots z_n=r_1\cdot r_2\cdots r_n\left[\pcx{\theta_1+\theta_2+\cdots+\theta_n}\right]\]

特别是当$z_1= z_2=\cdots= z_n$时,即
\[r_1=r2=\cdots=r_n=r,\qquad \theta_1=\theta_2=\cdots=\theta_n=\theta\]
代入上式就有
\[[r(\pc{\theta})]^n=r^n(\pc{n\theta})\quad (n\in\N)\]
这就是说,
\textbf{复数$z$的$n\; (n\in\N)$次幂,其模等于$z$的模的$n$次幂,其辐角等于$z$的辐角的$n$倍}。
这就是著名的\textbf{棣莫佛\footnote{棣莫佛(Abraham de Moivre)1667--1754年,法国数学家.}定理}。







\begin{example}
    
\end{example}

\begin{solution}
    
\end{solution}


\begin{example}
    
\end{example}

\begin{solution}
    
\end{solution}



\begin{example}
    
\end{example}

\begin{solution}
    
\end{solution}


\begin{example}
    
\end{example}

\begin{solution}
    
\end{solution}



\begin{example}
    
\end{example}

\begin{solution}
    
\end{solution}



\begin{example}
    
\end{example}

\begin{solution}
    
\end{solution}