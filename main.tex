\documentclass[b5paper, openany]{ctexbook}
\usepackage{zref-abspage}
\usepackage[margin=2.5cm]{geometry}


\usepackage{pifont}
\usepackage[perpage,symbol*]{footmisc}
\DefineFNsymbols{circled}{{\ding{192}}{\ding{193}}{\ding{194}}
{\ding{195}}{\ding{196}}{\ding{197}}{\ding{198}}{\ding{199}}{\ding{200}}{\ding{201}}}
\setfnsymbol{circled}



\usepackage{amsmath,amsfonts,mathrsfs,amssymb}
\usepackage{graphicx}

\usepackage[font=bf,labelfont=bf,labelsep=quad]{caption}

\usepackage{tikz}


\usepackage{ntheorem}
\theoremseparator{\;}



\usepackage{blkarray}
\usepackage{bm}
\usepackage[colorlinks=true, linkcolor=black]{hyperref}



\theoremstyle{plain}
\theoremheaderfont{\normalfont\bfseries} 
\theorembodyfont{\normalfont}


\usepackage[framemethod=tikz]{mdframed}


\newtheorem{example}{\bf 例}[chapter]
\newenvironment{solution}{\noindent {\bf 解:}}{}
\newenvironment{analyze}{\noindent {\bf 分析:}}{}
\newenvironment{rmk}{ {\bf 评述:}}{}
\newenvironment{remark}{ {\bf 注意:}}{}
\newenvironment{note}{ {\bf 说明:}}{}

\renewcommand{\emptyset}{\varnothing}

% \renewcommand{\proofname}{\bf 证明:}
\newenvironment{proof}{{\noindent \bf 证明:}}{}%{\hfill $\square$\par}

\newcommand{\E}{\mathbb{E}}
\renewcommand{\Pr}{\mathbb{P}}
\newcommand{\EP}{\mathbb{E}^{\mathbb{P}}}
\newcommand{\EQ}{\mathbb{E}^{\mathbb{Q}}}
\newcommand{\dif}{\,{\rm d}}
\newcommand{\Var}{{\rm Var}}
\newcommand{\Cov}{{\rm Cov}}
\newcommand{\x}{\times}
\renewcommand{\Longrightarrow}{\;\Rightarrow\;}
\newcommand{\map}[3]{#1:\; #2\mapsto #3}



 \usepackage{tcolorbox}
 \tcbuselibrary{breakable}
 \tcbuselibrary{most}



\newtcolorbox{ex}[1][]
  {colback = white, colframe = cyan!75!black, fonttitle = \bfseries,
    colbacktitle = cyan!85!black, enhanced,
    attach boxed title to top center={yshift=-2mm},breakable, 
    title=练习, #1}

\newtcolorbox{blk}[1][]
  {colback = white, colframe = magenta!75!black, fonttitle = \bfseries,
    colbacktitle = magenta!85!black, enhanced,
    attach boxed title to top left={xshift=5mm, yshift=-2mm},breakable, 
    title=思考题, #1}

\newtcolorbox{thm}[2][]
  {colback = white, colframe = magenta!75!black, fonttitle = \bfseries,
    colbacktitle = magenta!85!black, enhanced,
    attach boxed title to top left={xshift=5mm, yshift=-2mm},breakable, 
    title=#2, #1}

% \newtcolorbox{note}[1][]
%   {colback = white, colframe = blue!75!black, fonttitle = \bfseries,
%     colbacktitle = blue!85!black, enhanced,
%     attach boxed title to top left={xshift=5mm, yshift=-2mm},breakable, 
%     title=说明, #1}



\setcounter{tocdepth}{1}

\setcounter{secnumdepth}{2}



% \ctexset {
% section = {
% 	name = {第,节},
%  	number = \chinese{section}},
% subsection = {
% 	name = {,、\hspace{-1em}},
% 	number = \chinese{subsection}
% },
% subsubsection = {
% 	name = {(,)\hspace{-1em}},
% 	number = \chinese{subsubsection},
% }
% }



\renewcommand{\contentsname}{目~~录}

\newcommand{\poly}{\polynomial[reciprocal]}
\newcommand{\Q}{\mathbb{Q}}
\newcommand{\R}{\mathbb{R}}
\newcommand{\N}{\mathbb{N}}
\newcommand{\Z}{\mathbb{Z}}


\usepackage{mathtools}

\setlength{\abovecaptionskip}{0.cm}
\setlength{\belowcaptionskip}{-0.cm}

\usetikzlibrary{decorations.pathmorphing, patterns}
\usetikzlibrary{calc, patterns, decorations.markings}
\usetikzlibrary{positioning, snakes, hobby}

\usepackage{lscape}

\usepackage{yhmath}
\usepackage{longdivision}
\usepackage{polynom}
\usepackage{polynomial}
\usepackage{cancel}

\renewcommand{\frac}{\dfrac}
\newcommand{\oc}{$^{\circ}{\rm C}$}
\newcommand{\blank}{\underline{\qquad}}

\usepackage{multicol}
\usepackage{cases}
% \usepackage{enumitem}
\usepackage{ulem}
\usepackage{enumerate}
\usepackage{tkz-euclide}
\renewcommand{\ge}{\geqslant}
\renewcommand{\le}{\leqslant}
\renewcommand{\geq}{\geqslant}
\renewcommand{\leq}{\leqslant}
\usepackage{esvect}
\newcommand{\VEC}{\vv}

\newcommand{\pc}[1]{\cos#1+i\sin#1}
\newcommand{\pcx}[1]{\cos\left(#1\right)+i\sin\left(#1\right)}
\newcommand{\Lim}[2]{\lim\limits_{#1\to #2}}




\begin{document}



\title{\Huge\bfseries 北京四中高中教学讲义\\代数(第二册)}



\author{\Large 北京四中教学处~~编}
\date{\Large 1996年1月}

\maketitle




\frontmatter

\chapter{出版说明}

当前,中学教学改革已经深入到课程设置和教材改革领域。
我校数学教材的改革,以发展学生的数学思维为目标,以不改变现行教学大纲规定的教学闪容为前提,试图通过对知识结构及其展开方式的统盘考虑,实现整体优化。经多年反复探索、实验,编成了这套尝试融教材与教法、学法于一体的《北京四中高中教学讲义》。

这套讲义的产生可以上溯到 1982 年。从那时起,为了发展学生智能,提高数学素养,我校部分同志就开始对高中数学教学进行以教材改革为龙头,以学法教育为重点的“整体优化实验研究”。正是在这项研究的基础上,逐步形成了这套讲义编写的特色和风格。这就是:
\begin{enumerate}
    \item 为形成学生良好的认知结构,讲义的知识结构力求脉络分明,使学生能从整体上理解教材。
    \item 为了提高学生的数学素养,本讲义把数学思想的阐述放到了重要位置。数学思想既包含对数学知识点(概念、定理、公式、法则和方法)的本质认识,也包含对问题解决的数学基本观点。它是数学中的精华,对形成和发展学生的数学能力具有特别重要的意义。为此,讲义注重展现思维过程(概念、法则被概括的过程,教学关系被抽象的过程,解题思路探索形成的过程)。在过程中认识知识点的本质,在过程中总结思维规律,在过程中揭示数学思想的指导作用。力图使学生能深刻领悟教材。
    \item “再创造,再发现”在数学学习中对培养创造维能力至关重要,为引导学生积极参与“发现”,讲义在设计上做了某些尝试。
    \item 例题和习题的选配,力求典型、适量、成龙配套。习题分为 A 组(基本题)、 B 组(提高题)和 C 组(研究题)。教师可根据学生不同的学习水平适当选用。
    \item 教材是学生学习的依据。应有利于培养自学能力,本书注重启迪学法,并在书末附有全部习题的答案或提示,以供学习时参考。
 \end{enumerate}   

这套讲义在研究、试教和成书的过程中,始终得到了北京市和西城区教育部门有关领导的关怀和帮助,得到了北京师范大学数学系钟善基教授、曹才翰教授的热情指导,清华附中的瞿宁远老师也积极参与了我们的实验研究,并对这套教材做出了贡献,在此一并致以诚挚的谢意。

在编写过程中,北京四中数学组的教师们积极参加研讨,对他们的热情支持表示感谢。

这套讲义包括六册:高中代数第一、二、三册,三角、立体几何、解析几何各一册。

编写适应素质教育的教材,对我们来说是个尝试。由于水平所限,书中不当之处在所难免,诚恳希望专家、同行和同学们提出宝贵意见。

\begin{flushright}
    北京四中教学处\\
    1996 年 1 月
\end{flushright}
\tableofcontents


\mainmatter

\setcounter{chapter}{3}  

\chapter{不等式的性质、证明和解法}
两个同类量相比较,有“相等关系”和“不等关系”,用数学式子表达它们就出现了“等式”和“不等式”。不等式不仅在数学理论中是极为重要的数学工具,而且在科学技术中有广泛的实用价值。

本章将系统地学习不等式的性质、证明和解法,并通过不等式的证明系统地学习数学论证的常用方法。

不等式与等式既有相似之处,也有不同之点。学习时要随时加以对比,尤为重要的是注意它们的不同之处。

\section{实数集的有关知识}

不等式理论的基础是实数理论。我们回顾一下实数集的有关知识。

\begin{enumerate}
    \item 任取$x\in\R$,则$x\in\R^+$、$x\in\{0\}$和$x\in \R^-$三种情况有且只有一种成立。
    \item 两实数加、乘运算具有以下性质:
    \begin{itemize}
        \item $a,b\in R^+$,则$a+b\in\R^+$, $a\cdot b\in\R^+$
        \item $a,b\in R^-$,则$a+b\in\R^-$, $a\cdot b\in\R^+$
        \item $a\in\R^+,\; b\in R^-$,则$a\cdot b\in\R^-$
        \item 任取$a\in\R$,恒有$a^2\ge 0$,当且仅当$a=0$时取等号
    \end{itemize}
    \item 比较两实数大小的法则。
\end{enumerate}

我们知道,实数可以比较大小。在初中我们学过实数比较大小的几何法则:在数轴上,两个不同的点$A$与$B$分别表示两个不同的实数$a$与$b$,右边的点表示的数比左边的点表示的数大。

现在我们学习与几何法则等价的比较两实数大小的代数法则。

\begin{thm}{定义}
对于任意两个实数$a,b$
\begin{itemize}
    \item 若$a-b$为正数,称$a$大于$b$,记作$a>b$;
    \item 若$a-b$为零,称$a$等于$b$,记作$a=b$;
    \item 若$a-b$为负数,称$a$小于$b$,记作$a<b$.
\end{itemize}
\end{thm}

根据这个定义,可知:
\[\begin{split}
    a-b>0 &\Longleftrightarrow a>b\\
    a-b=0 &\Longleftrightarrow a=b\\
    a-b<0 &\Longleftrightarrow a<b\\
\end{split}\]

\begin{example}
    若$a>b>c>0$,试比较$\frac{b}{a-b}$与$\frac{c}{a-c}$的大小.
\end{example}

\begin{analyze}
    根据实数比大小的定义,欲比较$\frac{b}{a-b}$与$\frac{c}{a-c}$的大小,应先研究它们的差。
\end{analyze}

\begin{solution}
$\frac{b}{a-b}-\frac{c}{a-c}=\frac{b(a-c)-c(a-b)}{(a-b)(a-c)}=\frac{a(b-c)}{(a-b)(a-c)}$

$\because\quad a>b>c>0$

$\therefore\quad b-c>0, \; a-b>0,\; a-c>0 \Longrightarrow $上式的值$>0$

$\therefore\quad \frac{b}{a-b}>\frac{c}{a-c}$.
\end{solution}

\begin{example}
若$a\in\R$,试比较$(a^2+\sqrt{2}a+1)(a^2-\sqrt{2}a+1)$与$(a^2+a+1)(a^2-a+1)$的大小
\end{example}

\begin{analyze}
    “式子繁,先化简”,再求差。
\end{analyze}

\begin{solution}
$(a^2+\sqrt{2}a+1)(a^2-\sqrt{2}a+1)=(a^2+1)^2-\left(\sqrt{2}a\right)^2=(a^2+1)^2-2a^2$

$(a^2+a+1)(a^2-a+1)=(a^2+1)^2-a^2$

\[[(a^2+1)^2-2a^2]-[(a^2+1)^2-a^2]=-a^2\]

$\because\quad a\in\R \Longrightarrow -a^2\le 0$

从而:\begin{enumerate}
    \item 当$a=0$时,$(a^2+\sqrt{2}a+1)(a^2-\sqrt{2}a+1)=(a^2+a+1)(a^2-a+1)$
    \item 当$a\ne 0$时,$(a^2+\sqrt{2}a+1)(a^2-\sqrt{2}a+1)<(a^2+a+1)(a^2-a+1)$
\end{enumerate}

\end{solution}

\begin{note}
以上两例说明:两实数比大小的问题,根据定义,等价于研究其差的符号,因而“求差”是基本方法。
\end{note}

关于两实数比大小,还有下面的定理

\begin{thm}{定理}
    对于任意两个实数$a$、$b$,下列三种大小关系:
\[a>b;\qquad a=b;\qquad a<b\]
有且只有一种成立。
\end{thm}

\begin{proof}
    根据实数集的性质,实数$(a-b)$或为正数,或为负数,或为零,三者必居其一且仅居其一。再由实数比大小的定义,立刻知道本定理成立。
\end{proof} 

最后,我们约定,在本章中所出现的字母,不加说明时都表示实数。

\section{不等式的概念}
用不等号($>$,$<$,$\ne$等)连接两个数学解析式(代数式,指数式,对数式,三角式等等)所成的式子称为\textbf{不等式}。例如:
\begin{align}
a^2+1&>0,\tag{1}\\
3x&<6,\tag{2}\\
|x|&<0,\tag{3}\\
\lg x&>\frac{1}{2},\tag{4}\\
m+2&<5+m,\tag{5}\\
-1&>2.\tag{6}    
\end{align}

在两个不等式中,如果每一个的左边都大于右边,如(1)和(4),或者每一个的左边都小于右边,如(2)和(3),象这样的两个不等式叫做\textbf{同向不等式}。如果一个不等式的左边大于右边,而另一个不等式的左边小于右边,如(1)和(2),象这两个不等式叫做\textbf{异向不等式}。

通常我们按不等式在字母取值范围内是否成立把不等式分作三类:
\begin{enumerate}
    \item 若在字母取值范围内不等式恒成立,这种不等式叫做\textbf{绝对不等式},如(1)、(5);
    \item 若在字母取值范围内有些值使不等式成立,而另外一些值使不等式不成立,这种不等式叫做\textbf{条件不等式},如(2)、(4);
\item 若在字母取值范围内不等式恒不成立,这种不等式叫做\textbf{矛盾不等式},如(3)、(6).
\end{enumerate}

不等式这一章的基本问题是绝对不等式的证明和条件不等式的求解。

为了解决这两个基本问题,我们要先学习不等式的性质。

\section*{习题一}
\begin{center}
    \bfseries A
\end{center}

\begin{enumerate}
    \item 比较$(x+1)(x+2)$与$(x-3)$的大小(其中$x\in\R$).
    \item 已知$a$是实数,$a^4+1$与$2a^2$谁大?
    \item 若$a>b$, $e>f$, $c>0$,试比较$f-ac$与$e-bc$的大小。
    \item 若$m\in\R$,试比较$\left(\frac{m}{\sqrt{3}}+1\right)^3-\left(\frac{m}{\sqrt{3}}-1\right)^3$与2的大小。
    \item 若$a>b>0$, $c<d<0$, $e>0$,求证:
    \[\frac{e}{a-c}>\frac{e}{b-d}\]
    \item 试说出下列不等式中哪些是绝对不等式,哪些是矛盾不等式,哪些是条件不等式:
\begin{enumerate}[(1)]
    \item $a^2<0$
    \item $a^2<\pi\left(\frac{\sqrt{2}}{2}a\right)^2$,其中$a>0$
    \item $a^2+b^2+c^2>0$
\end{enumerate}
\end{enumerate}

\section{不等式的性质}
在初中曾学过“等式”的一些性质,并根据对等式运算的需要做出了若干推论,现在整理如下:
\begin{thm}{性质1}
$$a=b\Longleftrightarrow b=a$$
\end{thm}

\begin{thm}{性质2}
\begin{equation}
    a=b,\; b=c\Longrightarrow a=c \tag{传递性}
\end{equation}
\end{thm}

\begin{thm}{性质3}
    \begin{equation}
        a=b\Longrightarrow a+m=b+m\tag{等量加同量}
    \end{equation}
\end{thm}

\begin{thm}{推论1}
    移项法则(如何叙述?)
\end{thm}

\begin{thm}{推论2}
\[\left.\begin{array}{c}
    a=b\\
    c=d
\end{array}\right\} \Longrightarrow a+c=b+d\]
\end{thm}

\begin{thm}{推论3}
    \[\left.\begin{array}{c}
        a=b\\
        c=d
    \end{array}\right\} \Longrightarrow a-c=b-d\]
\end{thm}


\begin{thm}{性质4}
\begin{equation}
    a=b\Longrightarrow am=bm\tag{等量乘同量}
\end{equation}    
\end{thm}

\begin{thm}{推论4}
    \[\left.\begin{array}{c}
        a=b\\
        c=d
    \end{array}\right\} \Longrightarrow ac=bd\]    
\end{thm}

\begin{thm}{推论5}
    \[a=b\ne 0\Longrightarrow \frac{1}{a}=\frac{1}{b}\]
\end{thm}

\begin{thm}{推论6}
    \[\left.\begin{array}{c}
        a=b\\
        c=d\ne 0
    \end{array}\right\} \Longrightarrow \frac{a}{c}=\frac{b}{d}\]
\end{thm}

\begin{thm}{推论7}
    \[a=b\Longrightarrow a^n=b^n\quad (n\in\N)\]
\end{thm}

\begin{thm}{推论8}
    \[a=b>0\Longrightarrow \sqrt[n]{a}=\sqrt[n]{b}\quad (n\in\N)\]
\end{thm}

不等式与等式是有密切联系的。类比等式的性质和推论,可以得到不等式的一些性质,并根据对不等式运算(变形)
的需要可以做出若干推论(最好你能独立想出这些结论和证明方法).

\begin{thm}{性质定理1}
    \[a>b\Longleftrightarrow b<a\]
\end{thm}

\begin{thm}{性质定理2}
    \begin{equation}
        a>b,\; b>c \Longrightarrow a>c \tag{传递性}
    \end{equation}
\end{thm}

\begin{thm}{性质定理3}
    \[a>b\Longrightarrow a+m>b+m\]
\end{thm}


\begin{thm}{推论1}
    \[a+m>b\Longrightarrow a>b-m\]
\end{thm}

\begin{thm}{推论2}
    \[\left.\begin{array}{c}
        a>b\\
        c>d
    \end{array}\right\} \Longrightarrow a+c=b+d\]    
\end{thm}

\begin{thm}{推论3}
    \[\left.\begin{array}{c}
        a>b\\
        c<d
    \end{array}\right\} \Longrightarrow a-c>b-d\]    
\end{thm}


\begin{thm}{性质定理4}
\begin{enumerate}
    \item $a>b,\; m>0\Longrightarrow am>bm$
    \item $a>b,\; m<0\Longrightarrow am<bm$
\end{enumerate}
\end{thm}

\begin{thm}{推论4}
    \[\left.\begin{array}{c}
        a>b>0\\
        c>d>0
    \end{array}\right\} \Longrightarrow ac>bd\]    
\end{thm}

\begin{thm}{推论5}
    \[\left.\begin{array}{c}
        a>b>0\\
        0<c<d
    \end{array}\right\} \Longrightarrow \frac{a}{c}>\frac{b}{d}\]    
\end{thm}

\begin{thm}{推论6}
    \[a>b>0\Longrightarrow \frac{1}{a}<\frac{1}{b}\]
\end{thm}

\begin{thm}{推论7}
    \[a>b\ge 0\Longrightarrow a^n>b^n\quad (n\in\N)\]
\end{thm}

\begin{thm}{推论8}
    \[a>b\ge 0\Longrightarrow \sqrt[n]{a}>\sqrt[n]{b}\quad (n\in\N,\; \text{且}\; n>1)\]
\end{thm}

\begin{thm}{推论9}当$a,b$都是正数时:
\[\begin{split}
    \frac{a}{b}>1&\Longleftrightarrow a>b\\
    \frac{a}{b}=1&\Longleftrightarrow a=b\\
    \frac{a}{b}<1&\Longleftrightarrow a<b\\
\end{split}\]
\end{thm}

以下研究这些定理的证明方法。

首先应明确:四条性质定理在这个理论结构中是基础。它们的证明应依据实数比大小的定义和实数加乘运算的法则。

例如,
求证:性质定理1\quad $a>b\Longleftrightarrow b<a$


\begin{proof}
先证“若$a>b$,则$b<a$”. 

由$a>b\Longrightarrow a-b>0$

$\therefore\quad -(a-b)<0$,即$b-a<0$

$\therefore\quad b<a$

再证“若$b<a$,则$a>b$”

由$b<a\Longrightarrow b-a<0 \Longrightarrow -(b-a)>0$,
即$a-b>0\Longrightarrow a>b$.

由以上两个方面,可得$a>b\Longleftrightarrow b<a$
\end{proof}

又如,求证:性质定理2\quad $a>b,\; b>c\Longrightarrow a>c$

\begin{proof}
\begin{align}
    a>b&\Longrightarrow a-b>0 \tag{1}\\
    b>c&\Longrightarrow b-c>0 \tag{2}
\end{align}
根据正数加正数仍为正数,由(1)(2)可得:
$$(a-b)+(b-c)>0$$即$a-c>0\Longrightarrow a>c$
\end{proof}

现在研究推论的证明,例如,求证:推论4 
\[\left.\begin{array}{c}
    a>b>0\\
    c>d>0
\end{array}\right\} \Longrightarrow ac>bd\] 
\begin{proof}
$\because\quad a>b,\; c>0$ \qquad $\therefore\quad ac>bc$\hfill(1)

又$\because\quad c>d,\; b>0$ \qquad $\therefore\quad bc>bd$\hfill(2)

由(1)(2)据传递性可得:$ac>bd$.
\end{proof}

又如,求证推论7
\[a>b\ge 0\Longrightarrow a^n>b^n,\quad n\in\N\]
\begin{proof}
\textbf{证法1:}由推论4可以证出(作为练习)。

\textbf{证法2:}也可以从“分析”结论入手:

要证$a^n>b^n$,只要证$a^n-b^n>0$. 根据乘法公式
\[a^n-b^n=(a-b)(a^{n-1}+a^{n-2}b +a^{n-3}b^2+\cdots+ab^{n-2} +b^{n-1})\]
由条件$a>b\ge 0$, 得
\[a-b>0,\quad (a^{n-1}+a^{n-2}b +a^{n-3}b^2+\cdots+ab^{n-2} +b^{n-1})>0\] 
$\therefore\quad a^n-b^n>0$,从而$a^n>b^n$
\end{proof}

\begin{rmk}
从上面的证法可以看出,欲证某结论成立,先找出使结论成立的充分条件,这对于把握论证的方向是有好处的。
\end{rmk}

 以上三节,结论较多,整理出它的知识结构,能帮助我们从整体上系统地掌握理论。   

\begin{center}
\begin{tikzpicture}[>=stealth]
\node[draw, rectangle] (A) at (0,0){两实数比大小的定义};
\node[draw, rectangle, text width=3.5cm, align=center] (A1) at (0,2){两实数比大小的方法\\(作差比较法)};
\node[draw, rectangle] (A2) at (0,-2){两实数比大小的定理};
\node[draw, rectangle] (B) at (5,0){四条性质定理};
\node[draw, rectangle] (C) at (8,0){九条推论};
\draw[double, ->](A)--(A1);
\draw[double, ->](A)--(A2);
\draw[double, ->](B)--(C);
\draw[double, ->](2.3,0)--(B);
\draw[decorate, very thick, decoration={brace, amplitude=5pt}](2,2)--(2,-2);
\end{tikzpicture}
\end{center}

 在这个理论结构中,有的同学可能对性质定理1的理解不深,甚至认为它无用。其实正因为有了它,后面的定理和推论只要对“$>$”成立的,对“$<$”也都成立。例如$a>b,\; b>c\Longrightarrow a>c$,由定理1,$c<b,\; b<a\Longrightarrow c<a$,等等。所以,在上述理论体系中,对每条性质都只叙述了对“$>$”成立的情况。

 \begin{thm}{问题}
 在上述四条性质定理和九条推论的条件中,若把“$>$”换成“$\ge $”,对结果会有什么影响呢?
 \end{thm}

 我们知道,“$\ge $”包括“$>$”或“$=$”两种情况,而这两种情况我们都有现成的结论,只要把这两种情况“合起来”就是了。例如,由
\[\left.\begin{array}{cc}
    a>b\Longrightarrow a+m>b+m\\
    a=b\Longrightarrow a+m=b+m\\
\end{array}\right\}\]
就有$a\ge b\Longrightarrow a+m\ge b+m$.

\begin{example}
在实数范围内回答下列问题(要简述理由):
\begin{enumerate}[(1)]
\item 若$a>b$,能推出$ac^2>bc^2$吗?
\item 若$ac^2>bc^2$,能推出$a>b$吗?
\item 若$a>b$, $ab\ne 0$,能推出$\frac{1}{a}<\frac{1}{b}$吗?
\item 若$a>b>0\; c>d>0$能推出$\frac{a}{c}>\frac{b}{d}$吗?
\end{enumerate}
\end{example}

\begin{solution}
\begin{enumerate}[(1)]
\item 不能。事实上,当$c=0$时,就得不出$ac^2>bc^2$
\item 能。事实上,由$ac^2>bc^2\Longrightarrow c\ne 0$,此时$c^2>0$, 由$ac^2>bc^2$,据定理4两边同乘$\frac{1}{c^2}\; (c^2>0)$即可推出$a>b$.
\item 不能。事实上当$a>0>b$时,$\frac{1}{a}>0$, $\frac{1}{b}<0$,此时$\frac{1}{a}>\frac{1}{b}$.

(由此可见:推论6的条件是充分条件,而不是必要条件。)
\item 不能。事实上
若取$a=c$, $b=d$,则$\frac{a}{c}=1,\; \frac{b}{a}=1$,有$\frac{a}{c}=\frac{b}{d}$.

(由此可以看出,推论5的条件是充分的,但不是必要的。)
\end{enumerate}
\end{solution}

\begin{rmk}
    解这类题,要把题目与不等式的性质“挂钩”,思考才有明确的方向,答题才有针对性。如题(1),目的是检查不等式性质定理4,明确了这一点,就不难想出当$c^2=0$时要出毛病。
\end{rmk}

\section*{习题二}
\begin{center}
    \bfseries A
\end{center}
\begin{enumerate}
    \item 证明性质定理3、4(i)。
\item 证明推论5。
\item 在实数集上回答下面的问题,并简述理由:
\begin{enumerate}[(1)]
    \item 若$a>b$, $c<d$,是否能得出$a+c>b+d$?
    \item 若$\sqrt[3]{a}>\sqrt[3]{b}$,是否能得出$a>b$?
    \item 若$a>b$, $c<d$,$a,b,c,d$都不为零,是否能得出$\frac{a}{c}>\frac{b}{d}$?
    \item $a>b$, $c>d$是否能得出$ac>bd$?
\end{enumerate}
\end{enumerate}

\begin{center}
    \bfseries B
\end{center}
\begin{enumerate}\setcounter{enumi}{3}
    \item 利用公式
\[\begin{split}
   a-b&=\left(\sqrt[n]{a}-\sqrt[n]{b}\right)\left[\left(\sqrt[n]{a}\right)^{n-1}+\left(\sqrt[n]{a}\right)^{n-2}\sqrt[n]{b}\right.\\
   &\qquad \left.+\left(\sqrt[n]{a}\right)^{n-3}\left(\sqrt[n]{b}\right)^2 +\cdots +\sqrt[n]{a}\left(\sqrt[n]{b}\right)^{n-2}+\left(\sqrt[n]{b}\right)^{n-1}\right] 
\end{split}
    \]
证明推论8
\item 用反证法证明推论8。
\item 求证$a>b>0$, $c<d<0\Longrightarrow ac<bd$
\item 若$ab>0$,且$a<b$,比较$a^2$和$b^2$的大小.
\item 若$a\in\R$,比较$\frac{1}{1+a}$与$1-a$的大小。
\end{enumerate}

\begin{center}
    \bfseries C
\end{center}

\begin{enumerate}
 \setcounter{enumi}{8}   \item 若$ab\ne 0$,试比较$\sqrt[3]{a^3+b^3}$与$\sqrt{a^2+b^2}$的大小
\end{enumerate}

\section{不等式证明的基本方法}
本节将通过实例学习不等式证明的基本方法:比较法、分析法、综合法,着重讲解方法的原理(依据)、格式和基本技巧。

\subsection{比较法}
\begin{example}
    求证$a^2+3>3a$.
\end{example}

\begin{analyze}
    欲比较$a^2+3$与$3a$的大小,只需研究其差即可。
\end{analyze}

\begin{proof}
\textbf{证法1:} 

$\because\quad a^2+3-3a=a^2-3a+\left(\frac{3}{2}\right)^2+3-\left(\frac{3}{2}\right)^2=\left(a-\frac{3}{2}\right)^2+\frac{3}{4}>0$

$\therefore\quad a^2+3>3a$.

\textbf{证法2:} $a^2+3-3a=a^2-3a+3$, 
(这是关于$a$的二次函数,也可用判别式确定它的值的符号。)

$\because\quad a^2$的系数为正,且$\Delta =(-3)^2-4\cdot 1\cdot 3<0$,

$\therefore\quad a^2-3a+3>0$,从而$a^2+3>3a$.    
\end{proof}

\begin{example}
    求证$a^2+b^2+2\ge 2(a+b)$.
\end{example}

\begin{proof}
    $\because\quad a^2+b^2+2-2a-2b=(a^2-2a+1)+(b^2-2b+1)=(a-1)^2+(b-1)^2\ge 0$

当且仅当$a=1$且$b=1$时等号成立。

$\therefore\quad a^2+b^2+2\ge 2(a+b)$
\end{proof}

\begin{example}
    求证$a^2+b^2+c^2\ge ab+bc+ca$.
\end{example}

\begin{proof}
\[\begin{split}
  \because\quad  a^2+b^2+c^2-ab-bc-ca  &=\frac{1}{2}(2a^2+2b^2+2c^2-2ab-2bc-2ca)\\
  &=\frac{1}{2}[(a-b)^2+(b-c)^2+(c-a)^2]\ge 0
\end{split}\]
当且仅当$a=b=c$时等号成立,
故$a^2+b^2+c^2\ge ab+bc+ca$(当且仅当$a=b=c$时取等号).
\end{proof}

\begin{example}
    已知$a,b\in\R^+$,且$a\ne b$,求证$a^3+b^3>a^2b+ab^2$.
\end{example}

\begin{proof}
    (比较法)
\[\begin{split}
    \text{设}A&=a^3+b^3-(a^2b+ab^2)=(a^3-a^2b)+(b^3-ab^2)\\
    &=a^2(a-b)+b^2(b-a)=(a-b)(a^2-b^2)\\
    &=(a+b)(a-b)^2
\end{split}\]

$\because\quad a>0,\; b>0,\; a\ne b$

$\therefore\quad a+b>0,\; (a-b)^2>0$,从而$A>0$

$\therefore\quad a^3+b^3>a^2b+ab^2$
\end{proof}

\begin{thm}{比较法原理}
     为了证左$>$右,根据实数比大小的定义,只要证$\text{左}-\text{右}>0$,这种方法称为比较法,是证明不等式最基本最常用的方法。步骤是作差、变形、定号。题目不同,变形、定号的方法就不同,一般说来,有的是用因式分解,有的是用配成完全平方,有的则用二次三项式的性质。
\end{thm}

对于例4.7,还可以根据推论9来证。

\begin{proof}
$\because\quad \text{左}=a^2+b^2>0,\qquad \text{右}=a^2b+ab^2>0$

根据推论9,欲证$a^2+b^2>a^2b+ab^2$\hfill (1)

等价于证明
\begin{equation}
    \frac{a^2+b^2}{a^2b+ab^2}>1 \tag{2}
\end{equation}
即:
\begin{equation}
    \frac{a^2-ab+b^2}{ab}>1 \tag{3}
\end{equation}
欲证(3),只要证$\frac{a^2-ab+b^2}{ab}-1>0$,即
\begin{equation}
    \frac{(a-b)^2}{ab}>0 \tag{4}
\end{equation}

$\because\quad a,b$为不等正数

$\therefore\quad $(4)显然成立

$\therefore\quad $(1)式成立.

\end{proof}

\begin{rmk}
这种证法是先“作商”,再研究商与1的大小。当等号两边都是正数的乘积或指数式时,常用这种证法(通常称为比商法)。这是因为指数式或乘积在“作商”后往往能约分化简。
\end{rmk}

\begin{example}
    若$a,b\in\R^+$,求证$a^a b^b\ge a^b b^a$,并指出等号成立的条件。
\end{example}

\begin{proof}
$\because\quad a>0, \quad b>0$

$\therefore\quad a^a b^b$和$a^b b^a$都是正数. (验证这一点,为的是使用推论9)

欲证原不等式成立,只需证$\frac{a^a b^b}{a^b b^a}\ge 1$.

而左边$=a^{a-b}b^{b-a}=\left(\frac{a}{b}\right)^{a-b}$,讨论如下:
\begin{enumerate}
    \item 若$a>b>0$,则$\frac{a}{b}>1$, $a-b>0\Longrightarrow \left(\frac{a}{b}\right)^{a-b}>1$(根据指数函数的性质)
    \item 若$a=b>0$,则$\frac{a}{b}=1$, $a-b=0\Longrightarrow \left(\frac{a}{b}\right)^{a-b}=1$
    \item 若$0<a<b$,则$\frac{a}{b}<1$, $a-b<0\Longrightarrow \left(\frac{a}{b}\right)^{a-b}>1$
\end{enumerate}
综合上述$\frac{a^a b^b}{a^b b^a}\ge 1$,当且仅当$a=b$时等号成立。

$\therefore\quad a^a b^b\ge a^b b^a$,当且仅当$a=b$时等号成立。
\end{proof}

\subsection{分析法}
它基于这样一种逻辑考虑:为了证命题$A$,去找$A$成立的充分条件$B$;为了证$B$,去找$B$成立的充分条件$C$;为了证
$C$,又去找$C$成立的充分条件$D$,……直至找到一个明显成立的不等式。

\begin{example}
    已知$a,b\in\R^+$,且$a\ne b$,用分析法证明
$a^3+b^3>a^2b+ab^2$
\end{example}

\begin{proof}
    \textbf{证法1:}欲证$a^3+b^3>a^2b+ab^2$ \hfill (1)
即证
\begin{equation}
    (a+b)(a^2-ab+b^2)>(a+b)ab \tag{2}
\end{equation}
由于$a$、$b$是正数$\Longrightarrow a+b>0$,所以欲证(2),只要证
\begin{equation}
    a^2-ab+b^2>ab \tag{3}
\end{equation}

欲证(3),只要证$a^2-2ab+b^2>0$,即证
\begin{equation}
    (a-b)^2>0\tag{4}    
\end{equation}

$\because\quad a\ne b$

$\therefore\quad $(4)显然成立.

$\therefore\quad $(1)式成立。
\end{proof}

\begin{thm}{分析法原理 }
    执果索因,逐步\underline{逆找}结论成立的\underline{充分条件},直至找到明显成立的不等式为止。
\end{thm}

很明显,逆找的过程正是把“欲证”由繁化简的过程。因而分析法对于形式复杂的证明题是一种行之有效的方法。

\begin{analyze}
    因为题设中$a$、$b$的地位是对等的(以$a$去代换$b$,同时以$b$去代换$a$,所得到的式子与原式相同),由此,在证明中不妨设$a>b>0$(事实上,若$b>a>0$,证法是完全相同的)。这种通过加强假设而简化证明过程的办法称作\textbf{优化假设}。它是数学证明中很有用的一种思考方法。
\end{analyze}

\begin{proof}
\textbf{证法2:}不妨设$a>b>0$. 欲证$a^3+b^3>a^2b+ab^2$\hfill (1)

只要证$a^3-a^2b>ab^2-b^3$, 即证
\begin{equation}
    a^2(a-b)>b^2(a-b) \tag{2}
\end{equation}
由$a>b>0\Longrightarrow a-b>0\; a^2>b^2$

据定理4,(2)成立$\Longrightarrow$(1)成立.
\end{proof}

\begin{rmk}
    若不用优化假设$a>b$,欲证(2)就应分$a>b$, $a<b$两种情况讨论之.
\end{rmk}

\begin{ex}
\begin{enumerate}
    \item 证明以下二次不等式:
\begin{enumerate}[(1)]
    \item $x+1>\frac{2}{3}x$
    \item $a+2b^2+c^2\ge 2ab-2bc$
\end{enumerate}

    \item 证明以下二次不等式,并指出等号成立的条件:
\begin{enumerate}[(1)]
\item $a^2+b^2+c^2+3\ge 2(a+b+c)$,
\item $x^2+5y^2+1\ge 4xy+2y$.
\end{enumerate}

    \item 用两种方法(比较法、分析法)证明:
\begin{enumerate}[(1)]
    \item 若$a,b\in\R^+$,且$a\ne b$,则$a^4+b^4>a^3b+ab^3$,
    \item 若$a,b,m\in\R^+$,且$a<b$,则$\frac{a}{b}<\frac{a+m}{b+m}$
\end{enumerate}

    (此题表明:分子、分母都为正数的真分数,分子、分母同加上正数$m$,分数值变大——但不超过1.这是分数的一
    个重要性质。若$a,b\in\R^+$,且$\frac{a}{b}$是个假分数又如何呢?)
\end{enumerate}
\end{ex}


\subsection{综合法}
从已知条件出发,根据学过的定义、定理等知识,逐步推出欲证不等式。这种证明方法称为综合法。

很明显,把分析法的过程逆写出来就是一种综合法。例
如,以上面例4.9中的证法2为线索逆写如下:

不妨设$a>b>0\Longrightarrow a-b>0$, $a^2>b^2$,

$\therefore\quad a^2(a-b)>b^2(a-b)$

即$a^3-a^2b>ab^2-b^3$,

$\therefore\quad a^3+b^3>a^2b+ab^2$.

\begin{thm}{综合法原理 }
    由因导果,逐步\underline{顺找}已知成立的必要条件,直至导出结论为止。
(这里的做法是以分析法找思路,以综合法写证明。很明显,这样用综合法写,过程比较简捷)
\end{thm}

应该说明:用综合法证不等式,非常重要的一类是从平均不等式出发。这类问题本书将在4.5节集中研究。


\section*{习题三}
\begin{center}
    \bfseries A
\end{center}
\begin{enumerate}
    \item 证明二次不等式,并指出等号成立的条件:
\begin{enumerate}[(1)]
\item $a^2+b^2\ge 2(a-b-1)$;
\item $3x^2-4xy+5y^2\ge 0$;
\item $x^2-2xy+2y^2+2x-4y+2\ge 0$.
\end{enumerate}

   \item  若$a,b\in\R^+$,求证
  $\left(\frac{a^2}{b}\right)^{\tfrac{1}{2}}+\left(\frac{b^2}{a}\right)^{\tfrac{1}{2}}\ge \sqrt{a}+\sqrt{b}$
\end{enumerate}

\begin{center}
    \bfseries B
\end{center}

\begin{enumerate}\setcounter{enumi}{2}
    \item 用两种方法(比较法、分析法)证明:若$a,b\in\R^+$,则$a^5+b^5>a^3b^2+a^2b^3$
    \item 若$a>1$,求证:$a^3>a+\frac{1}{a}-2$
    \item $a>c,\; b>c>0$,求证:$\sqrt{(a+c)(b+c)}+\sqrt{(a-c)(b-c)}\le 2\sqrt{ab}$
    \item 若$a,b,x,y$都为正数,求证$\frac{(a+b)xy}{ay+bx}\le \frac{ax+by}{a+b}$
    \item 证明:
\begin{enumerate}[(1)]
    \item $\frac{1}{\log_5 19}+\frac{2}{\log_3 19}+\frac{3}{\log_2 19}<2$
    \item $\frac{2}{5}<\log_5 2<\frac{4}{9}$
    \item 自编一个类似于(2)的题,并加以证明.
\end{enumerate}

\item 当$0<x\ne 1$, 且$m>n>0$时,$x^m+\frac{1}{x^m}$与$x^n+\frac{1}{x^n}$哪个大
($m,n\in\N$)?
\item 用反证法证明:
若$a$,$b$,$c$都是正数,则在$b+c-a$, $c+a-b$和$a+b-c$至少有两个是正的。
\item 若$a,b,c\in\R^+$,求证
\[a^a b^b c^c\ge a^{\tfrac{b+c}{2}}\cdot  b^{\tfrac{c+a}{2}}\cdot  c^{\tfrac{a+b}{2}}\]
\end{enumerate}

\begin{center}
    \bfseries C
\end{center}

\begin{enumerate}\setcounter{enumi}{10}
    \item 研究例4.9的推广。
    
    所谓推广,就是把真命题放在更广的范围内考查,因而是一种创造性的思维活动。要想做出推广,认清式子的“结构特征”是突破口。就此例而言,不等号两边都是二元的三次齐次式。推广至少有两个方向:(i)次数能否提高?(ii)“元”能否增多?对于学有余力的同学,不妨试试看。

\item     已知$a$,$b$都是正数,
\begin{enumerate}[(1)]
\item 求证$(a+b)(a^3+b^3)\ge (a^2+b^2)^2$
\item 你能把(1)加以推广吗?
\end{enumerate}

\item    设$a,b,c,d\in\R$,求证
    \[ac+bd\le \sqrt{a^2+b^2}\cdot \sqrt{c^2+d^2}\]
    (用多种方法,其中至少用一种几何方法或三角方法)
\end{enumerate}

\section{平均不等式}
用综合法证明不等式时,常常选用一些常见的不等式作为论证的起点。其中很重要的一类就是平均不等式。本节先介绍这些不等式,然后举例说明如何应用它们来证不等式。

\subsection{两个基本概念}

\begin{thm}
 {定义} 设$a_1,a_2,\ldots,a_n$是$n$个实数,$\frac{a_1+a_2+\cdots +a_n}{n}$叫做$a_1,a_2,\ldots,a_n$的\textbf{算术平均数}($n\in\N$)   
\end{thm}

\begin{thm}
{定义} 设$a_1,a_2,\ldots,a_n$是$n$个非负数,$\sqrt[n]{a_1a_2\cdots a_n}$叫做$a_1,a_2,\ldots,a_n$的\textbf{几何平均数}($n\in\N$)    
\end{thm}

“几何平均数”的名称来源于下述简单的几何问题:把边长分别为$a$,$b$的长方形变成和它面积相等的正方形(图4.1),求这个正方形的边长是多少?事实上可设正方形的边长为$x$,由等积这个条件得$ab=x^2$,所以$x=\sqrt{ab}$. 这里边长$x$是线段$a$和$b$在这种几何意义下的平均值。这是$n=2$时的情形。对于自然数$n\ge 3$,情况是类似的。
\begin{figure}[htp]
    \centering
\begin{tikzpicture}[>=stealth]
\draw(0,-.5) rectangle (3,.5);
\draw[->, very thick](3.5,0)--(4.5,0);
\draw(5,-1) rectangle (7,1);
\node at (1.5,-.5)[below]{$a$};
\node at (0,0)[left]{$b$};
\node at (6,-1)[below]{$x$};


\end{tikzpicture}
    \caption{}
\end{figure}

\subsection{算术平均数与几何平均数之间的大小关系}
现在,研究$n$个非负数的算术平均数$A_n$与几何平均数$G_n$之间的大小关系。

让我们任取$n$个($n=2,3,4,\ldots$)非负数(见下表),通过计算(可以用计算器完成)可得:

\begin{center}
\begin{tabular}{cccc}
    \hline  
&  $A_n=\frac{a_1+a_2+\cdots+a_n}{n}$ & $G_n=\sqrt[n]{a_1a_2\cdots a_n}$ & $A_n$与$G_n$的大小 \\[2ex]
\hline
1, 2&  $\frac{3}{2}=1.5$ & $\sqrt{2}=1.414\cdots$ &$A_2>G_2$\\[2ex]
5, 7&  $\frac{12}{2}=6$ & $\sqrt{35}=5.916\cdots$ &$A_2>G_2$\\[2ex]
1, 2, 5& $\frac{8}{3}=2.666\cdots$ & $\sqrt[3]{10}=2.154\cdots$ &$A_3>G_3$\\[2ex]
2, 4, 6, 8& $\frac{20}{4}=5$ & $\sqrt[4]{384}=4.426\cdots$ &$A_4>G_4$\\[2ex]
3, 3, 3, 3& $\frac{12}{4}=3$ & $\sqrt[4]{81}=3$ &$A_4=G_4$\\[2ex]
$\cdots$ & $\cdots$ & $\cdots$ & $\cdots$\\
\hline
\end{tabular}
\end{center}

由此作出猜想:任意$n$个非负数的算术平均数$A_n$一定不小于它们的几何平均数$G_n$,而
\begin{equation}
    \frac{a_1+a_2+a_3+\cdots+a_n}{n}\ge \sqrt[n]{a_1a_2a_3\cdots a_n}\tag{*}
\end{equation}
其中,$a_1,a_2,\ldots,a_n$都是非负数,$n\in\N$.

应该指出:上面通过具体数字反复试验而后做出猜想的方法是数学中在探索新定理时经常使用的。

(*)就是著名的\textbf{平均不等式}。它是一个非常基本的不等式:数学家们基于各种想法给出它的证明的方法有几十种之多。它既是推证许多重要不等式的基石,而且在解最值问题当中还是一个十分得力的工具。

以下研究(*)的证明。

先从最简单的情况($n=2$)开始,就是要证:
若$a$,$b$都是非负数,则
\begin{equation}
    \frac{a+b}{2}\ge \sqrt{ab}\tag{1}
\end{equation}

\begin{proof}
由于$a$、$b$是非负数,所以有
\[a=(\sqrt{a})^2,\qquad b=(\sqrt{b})^2\]
则不等式(1)等价于
\begin{equation}
    (\sqrt{a})^2+(\sqrt{b})^2\ge 2\sqrt{a}\sqrt{b}\tag{2}
\end{equation}
欲证(2),只要证
\[    (\sqrt{a})^2+(\sqrt{b})^2- 2\sqrt{a}\sqrt{b}\ge 0\]
即证:$(\sqrt{a}-\sqrt{b})^2\ge 0$.

此式显然成立,当且仅当$a=b$时取等号. 
(同样的方法可以证明:当$x,y\in\R$时,$x^2+y^2\ge 2xy$, 等号成立的充要条件是$x=y$)
\end{proof}

当$n=3$时,就是证明:若$a$、$b$、$c$都为非负数,则
\begin{equation}
    \frac{a+b+c}{3}=\sqrt[3]{abc} \tag{3}
\end{equation}

\begin{proof}
因为$a=(\sqrt[3]{a})^3,\; b=(\sqrt[3]{b})^3,\; c=(\sqrt[3]{c})^3$,设$x=\sqrt[3]{a},\; y=\sqrt[3]{b},\; z=\sqrt[3]{c}$,则(3)等价于
\begin{equation}
    x^3+y^3+z^3\ge 3xyz  \tag{4}
\end{equation}
\[\begin{split}
    x^{3}+ y^{3}+ z^{3}- 3xyz&=(x+y)^{3}+z^{3}-3x^{2}y-3xy^{2}-3xyz\\
    &=[(x+y)^{3}+z^{3}]-3xy(x+y+z)\\
    &=(x+y+z)[(x+y)^{2}-(x+y)z+z^{2}]-3xy(x+y+z)\\
    &=(x+y+z)[(x+y)^{2}-(x+y)z+z^{2}-3xy]\\
    &=\left(x+y+z\right)\left(x^{2}+y^{2}+z^{2}-xy-yz-zx\right)
\end{split}\]

$\because \quad a,b,c$是非负数,\qquad $\therefore \quad x,y,z$是非负数,

$\therefore\quad x+ y+ z\ge 0$,

又$x^2+y^2+z^2\geq xy+yz+zx$(上节例4.6已证)

$\therefore\quad x^2+ y^2+ z^2- xy- yz- zx\ge  0$, 当且仅当 $x=y=z$ 时取等号.

$\therefore\quad x^3+ y^3+ z^3\ge 3xyz$

$\therefore\quad \frac{a+b+c}{3}\ge \sqrt[3]{abc}$, 当且仅当 $a=b=c$ 时取等号.
\end{proof}

下面作为选学内容,介绍法国数学家Cauchy(柯西)
证明$n$元平均不等式的精美的构思。

他首先用上面的方法证明了(1),然后去证
\begin{equation}
a,b,c,d\text{都是非负数}\Longrightarrow \frac{a+b+c+d}{4}\ge \sqrt[4]{abcd} \tag{5}
\end{equation}

\begin{proof}
这只要在(1)中,以$\frac{a+b}{2}$,$\frac{c+d}{2}$分别代替$a,b$即可. 事实上,由于$\frac{a+b}{2}\ge 0,\; \frac{c+d}{2}\ge 0$,可得:
\[\frac{\frac{a+b}{2}+\frac{c+d}{2}}{2}\ge \sqrt{\frac{a+b}{2}\cdot \frac{c+d}{2}}\ge \sqrt{\sqrt{ab}\cdot \sqrt{cd}}\]
显然,当且仅当$a=b=c=d$时等号成立,即
\[\frac{a+b+c+d}{4}\ge \sqrt[4]{abcd} \]
\end{proof}

接着,他提出用同样的方法可以证明:若$a_1,a_2,\ldots,a_8$都是非负数,则
\[\frac{a_1+a_2+\cdots+a_8}{8}\ge \sqrt[8]{a_1a_2\cdots a_8}\]
(你能完成这个证明吗?)

如此继续下去,就能证得对于所有的自然数$n=2,4,8,16,\ldots, 2^k,\ldots \; (k \in\N)$,(*)都成立。

然后,他再利用已经证得的结论,去证明对于刚才“跳跃”过去的那些自然数$n=3,5,6,7,9,10,\ldots$,(*)都成立. 如利用$n=8$时的结论可以证明:

若$a_1,a_2,\ldots,a_5$都是非负数,则
\[\frac{a_1+a_2+\cdots+a_5}{5}\ge \sqrt[5]{a_1 a_2 \cdots a_5}\]

\begin{proof}
对于$a_1,a_2,\ldots,a_5$,我们凑出8个非负数:$a_1,a_2,\ldots,a_5,m,m,m$,其中$m=\frac{a_1+a_2+\cdots+a_5}{5}$,则
\[\frac{a_1+a_2+\cdots+a_5+m+m+m}{8}\ge \sqrt[8]{a_1 a_2 \cdots a_5\cdot m\cdot m\cdot m}\]
即:$m\ge \sqrt[8]{a_1a_2\cdots a_5\cdot m^3}$

$\therefore\quad m^8\ge a_1a_2\cdots a_5\cdot m^3 \Longrightarrow m^5\ge a_1a_2\cdots a_5\Longrightarrow m\ge \sqrt[5]{a_1a_2\cdots a_5}$

即:$\frac{a_1+a_2+\cdots +a_5}{5}\ge \sqrt[5]{a_1a_2\cdots a_5}$
其中,等号当且仅当$a_1=a_2=a_3=a_4=a_5$时成立。
\end{proof}

\begin{rmk}
    柯西构思的巧妙之处在于不仅省去了一个又一个的繁琐的因式分解,而且对任意的自然数$n$,运用的都是通法。
\end{rmk}

\begin{ex}
\begin{enumerate}
    \item 利用(5)去证明(6).
    \item 利用(5)去证明(3).
    \item 利用(6)去证明$n=6$的情况。
    \item 根据图4.2,用几何方法证明(1).
\end{enumerate}        
\end{ex}

\begin{figure}[htp]
    \centering
\begin{tikzpicture}[>=stealth]
\coordinate (A) at (-2,0);
\coordinate (O) at (0,0);
\coordinate (B) at (2,0);
\coordinate (D) at (1,0);
\coordinate (C) at (60:2);
\draw(B) arc (0:180:2);
\draw(A)node[left]{$A$}--(B)node[right]{$B$};
\draw[->, very thick](O)node[below]{$O$}--node[left]{$\frac{a+b}{2}$}(C)node[above]{$C$};
\draw(A)--+(0,-1);
\draw(D)--+(0,-1);
\draw(B)--+(0,-1);
\draw[<->](-2,-.75)--node[fill=white]{$a$}(1,-.75);
\draw[<->](1,-.75)--node[fill=white]{$b$}(2,-.75);
\draw[very thick](C)--node[right]{$\sqrt{ab}$}(D)node[below right]{$D$};


\end{tikzpicture}
    \caption{}
\end{figure}

\subsection{对平均不等式的认识}
先将上面的结果整理如下:
\begin{thm}{定理1}
    若$a$,$b$都是非负数,则
\begin{equation}
    \frac{a+b}{2}\ge \sqrt{ab}\qquad 
\text{(等号当且仅当$a=b$时成立)}\tag{1}
\end{equation}
进而还有:若$x,y\in\R$,则
\begin{equation}
    x^2+y^2\ge 2xy\qquad \text{(等号当且仅当$x=y$时成立)}\tag{2}
\end{equation}
\end{thm}

\begin{thm}{推论1 }
\[\begin{split}
    a\text{是正数}&\Longrightarrow a+\frac{1}{a}\ge 2\\
    a\text{是负数}&\Longrightarrow a+\frac{1}{a}\le -2
\end{split}\]
\end{thm}

\begin{thm}{推论2}
\[\begin{split}
    a,b\text{同号}&\Longrightarrow \frac{a}{b}+\frac{b}{a}\ge 2\\
a,b\text{异号} &\Longrightarrow \frac{a}{b}+\frac{b}{a}\ge -2\\
\end{split}\]
\end{thm}

\begin{thm}{定理2}
    若$a$,$b$,$c$都是非负数,则
\begin{equation}
    \frac{a+b+c}{3}\ge \sqrt[3]{abc}\tag{3}
\end{equation}
(等号当且仅当$a=b=c$时成立)
\end{thm}

(3)等价于:
若$x$,$y$,$z$都是非负数,则
\begin{equation}
    x^3+y^3+z^3\ge 3xyz \tag{4}
\end{equation}
(等号当且仅当$x=y=z$时成立)

更一般地有

\begin{thm}{定理3}
    若$a_1,a_2,\ldots,a_n$都是非负数,则
    \begin{equation}
\frac{a_1+a_2+\cdots +a_n}{n}\ge \sqrt[n]{a_1a_2\cdots a_n},\quad (n\in\N)       \tag{*}
    \end{equation}
当且仅当$a_1=a_2=\cdots= a_n$时等号成立。
\end{thm}
 
(*)等价于:若$x_1,x_2,\ldots,x_n$都是非负数,则
\begin{equation}
x^n_1+x^n_2+\cdots +x^n_n  \ge nx_1x_2\cdots x_n \tag{**}
\end{equation}
(等号当且仅当$x_1=x_2=\cdots=x_n$时成立)

对于(*)或(**),应特别注意理解:
\begin{enumerate}
    \item 式子的结构特征都是非负数的
\[\text{和的形式}\ge \text{积的形式}\]
若欲证不等式具有这样的结构特征,就有可能用“平均不等式”作出证明。这是能否利用“平均不等式”来证不等式的线索。
\item (*)或(**)还表明,$n$个非负数的和与积通过缩小或放大可以互相转化。认识了这一点用起(*)来就灵活多了。
\end{enumerate}


以下,举例说明怎样利用“平均不等式”去证明某些不等式.

\begin{example}
  求证下列不等式:  
\begin{enumerate}[(1)]
    \item $a^2+b^2+c^2\ge ab+bc+ca$
    \item 若$a$,$b$,$c$是不全相等的正数,求证
   \[ a(b^2+c^2)+b(c+a)+c(a^2+b^2)>6abc\]
    \item $a^4+b^4+c^4\ge abc(a+b+c)$,并指出等号成立的条件.
\end{enumerate}
\end{example}
 
\begin{proof}
\begin{enumerate}[(1)]
    \item 

    \begin{flushleft}
\begin{tikzpicture}
    \node at (-2.5, 3){$\because$};
\node at (0,3){$a^2+b^2\ge 2ab$};
\node at (0,2.3){$b^2+c^2\ge 2bc$};
\node at (0,1.6){$c^2+a^2\ge 2ca$};
\node at (.25,.9){$2(a^2+b^2+c^2)\ge 2(ab+bc+ca)$};
\draw(-3,1.2)--(3,1.2);
\node at (-2.5,1.6){$+)$};
\end{tikzpicture}
\end{flushleft}

$\therefore\quad a^2+b^2+c^2\ge ab+bc+ca$

\item \textbf{证法1:}由$b^2+c^2\ge 2bc,\; a>0$,得:
\[a(b^2+c^2)\ge 2abc\]
同理:
\[
    b(c^2+a^2)\ge 2abc, \qquad 
    c(a^2+b^2)\ge 2abc
\]

$\because\quad a,b,c$不全相等,

$\therefore\quad $上述三个不等式的等号不能同时成立。把三式左、右分别相加得
\[a(b^2+c^2)+b(c^2+a^2)+c(a^2+b^2)>6abc\]

\textbf{证法2:}$a,b,c$是不全相等的正数,


$\therefore\quad a(b^2+c^2),\; b(c^2+a^2),\; c(a^2+b^2)$也为正数。

由平均不等式($n=3$的情况)可得
\[    \begin{split}
a(b^2+c^2)+b(c^2+a^2)+c(a^2+b^2)&\ge 3\sqrt[3]{a(b^2+c^2)\cdot b(c^2+a^2)\cdot c(a^2+b^2)}\\
&>3\sqrt[3]{a(2bc)\cdot b(2ca)\cdot c(2ab)}\qquad \text{(理由?)}\\
&=3\sqrt[3]{(2abc)^3}=3\cdot 2abc=6abc
\end{split}\]
$\therefore\quad $原式成立.

\item $\because$
\begin{flushleft}
    \begin{tikzpicture}
        % \node at (-2.5, 3){$\because$};
    \node at (0,3){$a^4+b^4\ge 2a^2b^2$};
    \node at (0,2.3){$b^4+c^4\ge 2b^2c^2$};
    \node at (0,1.6){$c^4+a^4\ge 2c^2a^2$};
    \node at (.65,.9){$2(a^4+b^4+c^4)\ge 2(a^2b^2+b^2c^2+c^2a^2)$};
    \draw(-3,1.2)--(4,1.2);
    \node at (-2.5,1.6){$+)$};
    \end{tikzpicture}
    \end{flushleft}
当且仅当$a^2=b^2=c^2$时等号成立.

而
\begin{flushleft}
    \begin{tikzpicture}
        % \node at (-2.5, 3){$\because$};
    \node at (0,3){$a^2b^2+b^2c^2\ge 2ab^2c$};
    \node at (0,2.3){$b^2c^2+c^2a^2\ge 2abc^2$};
    \node at (0,1.6){$c^2a^2+a^2b^2\ge 2a^2bc$};
    \node at (0,.9){$2(a^2b^2+b^2c^2+c^2a^2)\ge 2abc(a+b+c)$};
    \draw(-3.5,1.2)--(3.5,1.2);
    \node at (-3,1.6){$+)$};
    \end{tikzpicture}
    \end{flushleft}
当且仅当$ab=bc=ca$时等号成立。
由此,
$2(a^4+b^4+c^4)\ge 2abc(a+b+c)$,

$\therefore\quad a^4+b^4+c^4>abc(a+b+c)$,
当且仅当$a=b=c$时成立。
\end{enumerate}

\end{proof}

\begin{rmk}
    这三个小题式子的结构特征都是“和的形式$\ge $积的形式”的迭加,从而先用平均不等式,再迭加即可。
\end{rmk}

\begin{example}
    已知$a$,$b$,$c$为正数,求证:
\begin{enumerate}[(1)]
    \item $(a+b)(b+c)(c+a)\ge 8abc$;
    \item $(a+b+c)^4\cdot (a^2+b^2+c^2)\ge 243a^2b^2c^2$.
\end{enumerate}
\end{example}

\begin{analyze}
    这类不等式可看作是“和的形式$\ge $积的形式”经迭乘而成。
\end{analyze}

\begin{proof}
\begin{enumerate}[(1)]
    \item $\because\quad a>0,\; b>0,\; c>0$

$\therefore\quad a+b\ge 2\sqrt{ab}>0,\quad b+c\ge 2\sqrt{bc}>0,\quad c+a\ge 2\sqrt{ca}>0$

以上三式左、右分别相乘得
\[(a+b)(b+c)(c+a)\ge 8\sqrt{ab\cdot bc\cdot ca}=8abc\]
\item $\because\quad a,b,c$都是正数,

$\therefore\quad a+b+c\ge 3\sqrt[3]{abc}>0\Longrightarrow (a+b+c)^4\ge 3^4\cdot (abc)^{\tfrac{4}{3}}>0$

又$a^2+b^2+c^2\ge 3\sqrt[3]{a^2b^2c^2}=3(abc)^{\tfrac{2}{3}}>0$

以上三式左、右分别相乘得
\[(a+b+c)^4\cdot (a^2+b^2+c^2)\ge 3^4\cdot 3(abc)^{\tfrac{4}{3}\times \tfrac{3}{2}}=243a^2b^2c^2\]
\end{enumerate}
\end{proof}

\begin{example}
    求证:$\lg 9\cdot \lg 11<1$\hfill(1)
\end{example}

\begin{analyze}
因$\lg11>1$, $\lg9<1$,故用两式相乘得不出欲证结果,怎么办?这个题能否用“平均”去作?事实上,从(1)式的结构上看,左边是两个正数$\lg9$与$\lg11$的“积”,这与平均不等式的结构特征相似,因此可用“平均”去做。此时有两个不等式可供选用:
\[ab\le \frac{a^2+b^2}{2}\quad \text{或}\quad \sqrt{ab}\le\frac{a+b}{2}\]
\end{analyze}

\begin{proof}
\textbf{证法1:}
\begin{flushleft}
\begin{tikzpicture}
    \node at (0,2.2){$\lg 9\cdot \lg 11<\frac{\lg^2 9+\lg 11}{2}$};
    \node at (0,1.3){$\lg 9\cdot \lg 11=\frac{2\lg 9\cdot \lg 11}{2}$};
\node at (-3,1.3){$+)$};
\draw(-3.5,0.8)--(2.5,.8);
\node at (0,.2){$2\lg9\cdot \lg 11<\frac{(\lg 9+\lg 11)^2}{2}$};
\end{tikzpicture}
\end{flushleft}

于是:\[
    \frac{(\lg 9+\lg 11)^2}{2}=\frac{(\lg99)^2}{2}< \frac{\lg 100}{2}=\frac{2}{2}=1
\]
$\therefore\quad \lg 9\cdot \lg 11<1$

\textbf{证法2:}$\because\quad \lg 9>0,\; \lg 11>0$

$\therefore\quad \sqrt{\lg 9\cdot \lg 11}<\frac{\lg 9+\lg11}{2}=\frac{\lg 99}{2}<\frac{\lg 100}{2}=\frac{2}{2}=1$

两边平方得:$\lg 9\cdot \lg 11<1$
\end{proof}

\begin{blk}
你能把例4.12加以推广吗?
\end{blk}

\begin{example}
若$a>b>0$,求证$a+\frac{1}{(a-b)b}\ge 3$,并指出等号成立的条件.
\end{example}

\begin{analyze}
    $a+\frac{1}{(a-b)b}$可写成$(a-b)+b+\frac{1}{(a-b)b}$,且由于$a>b
    >0$,可知$a-b>0,\; b>0,\; \frac{1}{(a-b)b}>0$,所以,可用“平均”来做.
\end{analyze}

\begin{proof}
\[a+\frac{1}{(a-b)b}=(a-b)+b+\frac{1}{(a-b)b}\]

$\because\quad a>b>0$

$\therefore\quad a-b>0,\; b>0,\; \frac{1}{(a-b)b}>0$,可得
\[(a-b)+b+\frac{1}{(a-b)b}\ge 3\sqrt[3]{(a-b)\cdot b\cdot \frac{1}{(a-b)b}}=3\]
从而原式成立. 

当且仅当$a-b=b=\frac{1}{(a-b)b}$,即$a=2,\; b=1$时等号成立.
\end{proof}

\begin{rmk}
这个题的结构特征是在“$\ge $”号的左边是一个各项皆正的“和的形式”,而右边是个特征系数3。这就启示我们有可能用“平均”去做。方法是“凑”,使左边凑出一个三项和。
\end{rmk}

\section*{习题四}
\begin{center}
    \bfseries A
\end{center}

\begin{enumerate}
    \item 已知$a$,$b$,$c$,$d$都是正数,求证
    \[a^2+b^2+c^2+d^2\ge ab+bc+cd+da\]
    \item 已知$a,b,c\in\R$,求证
    $a^2+b^2+c^2+3\ge 2(a+b+c)$,并指出等号成立的条件。
    \item $a$,$b$都是正数,求证
    $a+b^2(a+1)+a^2(b+1)+b\ge 6ab$
    \item 若$a>1$, $b>1$, $c>1$,
    求证$(a+1)(b+1)(a+c)(b+c)>16abc$.
    \item 已知$a_1,a_2,\ldots, a_n$都是正数,求证:
    \[(a_1+a_2+\cdots +a_n)\left(\frac{1}{a_1}+\frac{1}{a_2}+\cdots+\frac{1}{a_n}\right)\ge n^2\]
    \item 已知$a$,$b$,$c$,$d$都是正数,求证:
    \begin{enumerate}[(1)]
        \item $(ab+cd)(ac+bd)\ge 4abcd$
        \item $\left(\frac{a}{c}+\frac{b}{c}+\frac{c}{d}+\frac{d}{a}\right)(a^3+b^3+c^3)\ge 12abc$
    \end{enumerate}  
\end{enumerate}

\begin{center}
    \bfseries B
\end{center}

\begin{enumerate}\setcounter{enumi}{6}
    \item 已知$a>0$,求证:
\begin{enumerate}[(1)]
    \item $a+\frac{4}{a^2}\ge 3$
    \item $a^4+\frac{4}{a^2}\ge 3\sqrt[3]{4}$
\end{enumerate}
\item 求证$\frac{x^2+2}{\sqrt{x^2+1}}\ge 2$,其中$x\in\R$
\item 若$x\in\R$,求证:$1+2x^4\ge x^2+2x^3$
\item 若$x,y\in\R^+$,求证:$\frac{1}{x}+\frac{1}{y}\ge \frac{4}{x+y}$
\item 当$a\ge 2$时,求证$\log_a(a-1)\cdot \log_a(a+1)<1$(这个题是例4.12的推广)
\item \begin{enumerate}[(1)]
    \item 若$a>b>c>d$,求证:$a-d+\frac{25^2}{(a-b)(b-c)(c-d)}\ge 20$
    \item 若$m>n>0$,求证:$m+n+\frac{16}{(m-n)n^2}\ge 8$
\end{enumerate}

\item 若$a,b,c$为三角形的三边,求证:
\begin{enumerate}[(1)]
    \item $(a+b+c)^3\ge 27(a+b-c)(b+c-a)(c+a-b)$
    \item $\frac{1}{a+b-c}+\frac{1}{b+c-a}+\frac{1}{c+a-b}\ge \frac{9}{a+b+c}$
\end{enumerate}

\end{enumerate}

\begin{center}
    \bfseries C
\end{center}

\begin{enumerate}\setcounter{enumi}{13}
    \item 若$x>0$,求证:$1+\frac{1+x}{9}>\sqrt[9]{2+x}$
    \item 对于任意正数$a,b$,$a\ne b$,求证$\sqrt[n+1]{ab^n}<\frac{a+nb}{n+1}$
    \item 若$a,b,c,d$为正数,求证:
\begin{enumerate}[(1)]
    \item $\frac{1}{a+b+c}+\frac{1}{b+c+d}+\frac{1}{c+d+a}+\frac{1}{d+a+b}\ge \frac{16}{3}\cdot \frac{1}{a+b+c+d}$
    \item $\frac{1}{a+3b+5c+7d}+\frac{1}{b+3c+5d+7a}+\frac{1}{c+3d+5a+7b}+\frac{1}{d+3a+5b+7c}\ge \frac{1}{a+b+c+d}$
\end{enumerate}
\end{enumerate}

\section{用放缩法证明不等式}

先研究一个例子。

\begin{example}
    若$a$,$b$,$c$,$d$为任意正数,求证
\[1<\frac{a}{a+b+c}+\frac{b}{b+c+d}+\frac{c}{c+d+a}+\frac{d}{d+a+b}<2\]
\end{example}

\begin{analyze}
    记$m=\frac{a}{a+b+c}+\frac{b}{b+c+d}+\frac{c}{c+d+a}+\frac{d}{d+a+b}$

这个题并不是让我们去求$m$的值,而是证明$m$的值存在的范围是在区间$(1,2)$之中,因而,就实质而言,这是一个“估值问题”。显然,先通分、求和再估值是不可取的。处理估值问题的方法,一般是对所给的式子的值先放大(或缩小),使之由繁化简,再进一步研究。
\end{analyze}

\begin{proof}
    由于$a$、$b$、$c$、$d$为正数,可先通过“放大”分母,把$m$缩小。
\[\begin{split}
    m&>\frac{a}{a+b+c+d}+\frac{b}{b+c+d+a}+\frac{c}{c+d+a+b}+\frac{d}{d+a+b+c}\\
    &=\frac{a+b+c+d}{a+b+c+d}=1
\end{split}\]
(“放大”之后构
造出公分母,实现了由繁化简)。再把分母“缩小”,达到把$m$放大.
\[m<\frac{a}{a+c}+\frac{b}{b+d}+\frac{c}{c+a}+\frac{d}{d+b}=\frac{a+c}{a+c}+\frac{b+d}{b+d}=2\]
$\therefore\quad 1<m<2$.
\end{proof}

\begin{thm}{放缩法原理}
由不等式的传递性,若$a>b$, $b>c$,则$a>c$. 由此,欲证$a>c$,可以先把$a$逐步缩小:
\[a>b_1>b_2>\cdots >b_n\]
而最后只要证出$b_n>c$,就可断言$a>c$. 类似地,欲证$a<c$,可以先把$a$逐步放大:
\[a<d_1<d_2<\cdots <d_n\]
最后,只要证出$d_n<c$,就可断言$a<c$.
\end{thm}

\begin{example}
    若$a\ge b>0$, $n\in\N$,求证
\begin{equation}
    n(a-b)b^{n-1}\le a^n-b^n\le n(a-b)a^{n-1}\tag{1}
\end{equation} 
\end{example}

\begin{proof}
    $\because\quad a\ge b>0$, $n\in\N$,且
\begin{equation}
    a^n-b^n=(a-b)(a^{n-1}+a^{n-2}b+a^{n-3}b^2+\cdots+ab^{n-2}+b^{n-1})  \tag{2}
\end{equation}
\begin{enumerate}[(i)]
\item 当$a=b$时,(1)式显然成立;
\item 当$a>b>0$时,(1)式等价于
\begin{equation}
nb^{n-1}\le a^{n-1}+a^{n-2}b+a^{n-3}b^2+\cdots+ab^{n-2}+b^{n-1}\le na^{n-1}\tag{3}
\end{equation}
\end{enumerate}
对(3)中间的和式使用放缩法,有
\[\begin{split}
    a^{n-1}+a^{n-2}b+a^{n-3}b^2+\cdots+ab^{n-2}+b^{n-1}&<a^{n-1}+a^{n-1}+\cdots+a^{n-1}=na^{n-1}\\
    a^{n-1}+a^{n-2}b+a^{n-3}b^2+\cdots+ab^{n-2}+b^{n-1}&>b^{n-1}+b^{n-1}+\cdots+b^{n-1}=nb^{n-1}\\
\end{split}\]

$\therefore\quad $(3)式成立,从而(1)式成立.
\end{proof}

\begin{example}
若$S_n=1+\frac{1}{\sqrt{2}}+\frac{1}{\sqrt{3}}+\cdots+\frac{1}{\sqrt{n}}$, $n\in\N$,求证:
\[2\left(\sqrt{n+1}-1\right)<S_n<2\sqrt{n}\]
\end{example}

\begin{analyze}
可把$S_n$看作是下面一列数
\[1,\; \frac{1}{\sqrt{2}},\; \frac{1}{\sqrt{3}}\cdots,  \frac{1}{\sqrt{n}}, \cdots\]
的前边$n$个项的和。这里并不要求计算$S_n$的精确值,而是估计$S_n$的值存在的范围。因而可用放缩法。由于$S_n$有$n$项,放缩后裂项相抵消是最理想的。
\end{analyze}

\begin{proof}
    先考虑上面一列数中的第$n$个项的通项怎样放缩(放缩的办法很多,下面的方法兼有裂项的目的)
\[\begin{split}
\frac{1}{\sqrt{k}} &= \frac{2}{\sqrt{k}+\sqrt{k}}<\frac{2}{\sqrt{k}+\sqrt{k-1}} =2(\sqrt{k}-\sqrt{k-1}) \\   
\frac{1}{\sqrt{k}} &=  \frac{2}{\sqrt{k}+\sqrt{k}}>\frac{2}{\sqrt{k}+\sqrt{k+1}} =2(\sqrt{k+1}-\sqrt{k}) \\    
\end{split}\]
再令$k=1,2,3,\ldots,n-1,n$,得:
\begin{flushleft}
\begin{tikzpicture}
\node at (0,3){$2\left(\sqrt{2}-\sqrt{1}\right)<\frac{1}{\sqrt{1}}<2\left(\sqrt{1}-\sqrt{0}\right)$}  ;  
\node at (0,2){$2\left(\sqrt{3}-\sqrt{2}\right)<\frac{1}{\sqrt{2}}<2\left(\sqrt{2}-\sqrt{1}\right)$}  ;  
\node at (0,1){$\cdots \cdots\cdots \cdots\cdots \cdots\cdots \cdots$}  ;  
\node at (0,0){$2\left(\sqrt{n}-\sqrt{n-1}\right)<\frac{1}{\sqrt{n-1}}<2\left(\sqrt{n-1}-\sqrt{n-2}\right)$}  ;  
\node at (0,-1){$2\left(\sqrt{n+1}-\sqrt{n}\right)<\frac{1}{\sqrt{n}}<2\left(\sqrt{n}-\sqrt{n-1}\right)$}  ;  
\node at (-4.5,-1){$+)$}  ;  
\draw(-5,-1.5)--(5,-1.5);
\node at (0,-2){$2\left(\sqrt{n+1}-1\right)<S_n<2\left(\sqrt{n}-0\right)$};
\end{tikzpicture}
\end{flushleft}

$\therefore\quad 2\left(\sqrt{n+1}-1\right)<S_n<2\sqrt{n}$.

\end{proof}

\begin{rmk}
  这里的“放”或“缩”,由于把握了分式和根式的性质,使得求和成为可能。 
\end{rmk}

\section*{习题五}
\begin{center}
    \bfseries A
\end{center}

\begin{enumerate}
    \item 若$a>0$, $b>0$,求证:
\[\frac{a+b}{1+a+b}<\frac{a}{1+a}+\frac{b}{1+b}<\frac{2(a+b)}{1+a+b}\]
\item 求证:$-\sqrt{2}\le \sin x+\cos x\le \sqrt{2}$
\end{enumerate}

\begin{center}
    \bfseries B
\end{center}


\begin{enumerate}\setcounter{enumi}{2}
    \item 求证: $1\le \frac{1}{1^2}+ \frac{1}{2^2}+\cdots + \frac{1}{n^2}<2$
    
(提示:利用$k^2>k(k-1)$)

\item 求证:$1+\frac{1}{1!}+\frac{1}{2!}+\frac{1}{3!}+\cdots+\frac{1}{n!}<3$

(提示:利用$\frac{1}{n!}<\frac{1}{(n-1)n}$或$\frac{1}{n!}<\frac{1}{2^{n-1}}$)

\item 设$a_n=\sqrt{1\cdot 2}+\sqrt{2\cdot 3}+\cdots +\sqrt{n(n+1)}$,求证:
\[\frac{n(n+1)}{2}<a_n<\frac{(n+1)^2}{2}\quad (n\in\N)\]
\end{enumerate}

\begin{center}
    \bfseries C
\end{center}


\begin{enumerate}\setcounter{enumi}{5}
\item 若$x_n=\frac{n^2-n+2}{3n^2+2n-4}$,求证:当$n>2$时,有
\[\left|x_n-\frac{1}{3}\right|<\frac{1}{n}\]
(提示:对$|x_n-\tfrac{1}{3}|$连续放大)

\item 若$a,b,c\in\R^+$,求证:$\frac{a}{b+c}+\frac{b}{c+a}+\frac{c}{a+b}\ge \frac{3}{2}$.
\end{enumerate}

\section{* 柯西不等式}

作为选学内容,本节介绍著名的\textbf{柯西(Cauchy)不等式}。
若$a,b,x,y\in \R$, 则
\begin{equation}
    (ax+by)^2\le (a^2+b^2)(x^2+y^2)  \tag{1}
\end{equation}
(注意,这个不等式在本章习题三第13题曾证过)

现在,探索$(1)的新证法$。

(1)式的结构特征使我们联想到一元二次 方程 的 判别式
$\Delta=b^2-4ac$。由此,欲证(1), 只要证关于$t$的二次三项式
$$f(t)=(a^2+b^2)t^2-2(ax+by)t+(x^2+y^2)\ge 0$$
这一点可以通过对$f(t)$配方实现,
\[\begin{split}
    f(t)&=(a^{2}t^{2}-2axt+x^{2})+(b^{2}t^{2}-2byt+y^{2})\\
    &=(at-x)^{2}+(bt-y)^{2}\ge 0
\end{split}\]
等号当且仅当$at=x$, $bt=y$, 也就是$a$, $b$与$x$, $y$对应成比例
时成立。

\begin{rmk}
    由(1)式的结构特征联想到$\Delta$, 从而想到去构造二次三项式$f(t)$.
\end{rmk}

把(1)推广:当$a,b,c,x,y,z\in \R$时就有
\begin{equation}
    (ax+by+cz)^{2}\le (a^{2}+b^{2}+c^{2})(x^{2}+y^{2}+z^{2})\tag{2}
\end{equation}
(等号当且仅当$a,b,c$与$x,y,z$对应成比例时成立)

当$a_1,a_2,\ldots,a_n$与$x_1,x_2,\ldots,x_n\in \R$, 又有
\begin{equation}
    (a_1x_1+a_2x_2+\cdots+a_nx_n)^2\le (a_1^2+a_2^2+\cdots+a_n^2)(x_1^2+
x_{2}^{2}+\cdots+x_{m}^{2}) \tag{3}
\end{equation}
(等号当且仅当$a_1,a_2,\cdots,a_n$与$x_1,x_2,\cdots,x_n$对应成
比例时成立).

\begin{example}
    用柯西不等式证明:
\begin{enumerate}[(1)]
    \item 若$a,b,c,d\in \R^+$, 则$\sqrt(a+c)(b+d)\geqslant\sqrt{ab}+\sqrt{cd}$,
\item 若$a,b,c\in \R^{+}$, 则$(a+b+c)\left(\frac1a+\frac1b+\frac1c\right)\ge 9$
\end{enumerate}
\end{example}

\begin{analyze}
    能否使用柯西不等式,关键要认清柯西不等式的
结构特征。其中,由$(\quad)^{2}$到$(\quad)\cdot(\quad)$可视为“放大”,
反向用可视为“缩小”。
\end{analyze}


\begin{proof}
\begin{enumerate}[(1)]
    \item $\because\quad a, b, c, d\in \R^{+ }$, 欲证原 不 等 式 , 只 要 证 
$\left(\sqrt{ab}+\sqrt{cd}\right)^2\leq (a+c)(b+d)$

即证$\left(\sqrt{a}\cdot\sqrt{b}+\sqrt{c}\cdot\sqrt{d}\right)^{2}\le \left[\left(\sqrt{a}\right)^{2}+\left(\sqrt{c}\right)^{2}\right]\cdot \left[\left(\sqrt{b}\right)^{2}+\left(\sqrt{d}\right)^{2}\right]$

由柯西不等式知此式成立

$\therefore\quad $原不等式成立。
\item $\because\quad  a,b,c\in \R^{+}$,
\[\begin{split}
&\quad (a+b+c)\left(\frac{1}{a}+\frac{1}{b}+\frac{1}{c}\right)\\
&=\left[(\sqrt{a})^2+(\sqrt{b})^2+(\sqrt{c})^2\right]\left[\left(\frac{1}{\sqrt{a}}\right)^2+\left(\frac{1}{\sqrt{b}}\right)^2+\left(\frac{1}{\sqrt{c}}\right)^2\right]\\
&\ge \left[\sqrt{a}\cdot\frac{1}{\sqrt{a}}+\sqrt{b}\cdot\frac{1}{\sqrt{b}}+\sqrt{c}\cdot\frac{1}{\sqrt{c}}\right]^2 \qquad  \text{(柯西不等式)}\\
&=(1+1+1)^2=9
\end{split}\]
\end{enumerate}
\end{proof}

\section*{*习题六}
\begin{enumerate}
    \item 证明柯西不等式(2)
    \item 若$a\in \R$, 用两种方法证明$(1+a+a^2)^2\le 3(1+a^2+a^4)$
    \item 已知$a,b,c$是互不相等的正数, $s=a+b+c$, 求证
    $$\frac{s}{s-a}+\frac{s}{s-b}+\frac{s}{s-c}>\frac{9}{2}$$
    \item 若$c\ge 0,\; a\ge c,\; b\ge c$, 求证
    \[\sqrt{c(a-c)}+\sqrt{c(b-c)}\le \sqrt{ab}\]
    \item $a_{1},a_{2},\ldots,a_{n}, b_{1},b_{2},\ldots,b_{n}\in \R^{+}$, 求证
    \[\sqrt{a_{1}b_{1}}+\sqrt{a_{2}b_{2}}+\cdots+\sqrt{a_{n}b_{n}}\le \sqrt{a_{1}+a_{2}+\cdots+a_{n}}\cdot \sqrt{b_{1}+b_{2}+\cdots+b_{n}}\]
    \item $3a^{2}+2b^{2}+c^{2}=1$, 求$3a+2b+c$的最大值.
    \item 若$a,b,c\in \R^{+},\; n\in \N$, 且$f(n)=\lg \frac{a^n+b^n+c^n}{3}$,求证:$2f(n)\le f(2n)$
\end{enumerate}

\section{附加条件的不等式的证明}
这类问题除具有前述不等式论证的共性以外,怎样使用附加条件就成了解题的关键。利用条件的方法是多种多样的,但是把条件直接代入(或变形后代入)或从条件出发作有目标的变形(变“已知”为“所求”)是最常用的两种方法。

\begin{thm}
{问} 从条件$a,b\in\R^+$,且$a+b=1$出发,你能获得哪些结论?试试看。    
\end{thm}

\begin{example}
已知$a,b,c\in\R^+$,且$a+b+c=1$,求证
\begin{equation}
    \frac{1}{a}+\frac{1}{b}+\frac{1}{c}\ge 9
\end{equation}
\end{example}

\begin{analyze}
    应从已知和所求的结构特征悟出下面的几种证法来。
\end{analyze}

\begin{proof}
\textbf{证法1:}$\because\quad a+b+c=1$,以$(a+b+c)$代换(1)之左边的分子,得
\[\begin{split}
    \frac{1}{a}&=\frac{a+b+c}{a}=1+\frac{b}{a}+\frac{c}{a}\\
    \frac{1}{b}&=\frac{a+b+c}{b}=1+\frac{a}{b}+\frac{c}{b}\\
    \frac{1}{c}&=\frac{a+b+c}{c}=1+\frac{a}{c}+\frac{b}{c}
\end{split} \]
以上三式,左、右分别相加,得
\begin{equation}
    \frac{1}{a}+\frac{1}{b}+\frac{1}{c}=3+\left(\frac{b}{a}+\frac{c}{a}\right)+\left(\frac{a}{b}+\frac{c}{b}\right)+\left(\frac{a}{c}+\frac{b}{c}\right) \tag{2}
\end{equation}

又$\because\quad a,b,c$都是正数,

$\therefore\quad \frac{b}{a}+\frac{c}{a}\ge 2,\quad \frac{a}{b}+\frac{c}{b}\ge 2,\quad \frac{a}{c}+\frac{b}{c}\ge 2$

代入(2)式得:$ \frac{1}{a}+\frac{1}{b}+\frac{1}{c}\ge 3+2+2+2=9$.

\textbf{证法2:} 已知$a+b+c=1$,代入(1)之左边
\[\begin{split}
    \frac{1}{a}+\frac{1}{b}+\frac{1}{c}&=\frac{a+b+c}{a}+\frac{a+b+c}{b}+\frac{a+b+c}{c}\\
    &=(a+b+c)\left(\frac{1}{a}+\frac{1}{b}+\frac{1}{c}\right)
\end{split}\]
又$a,b,c$都是正数,因而
\[a+b+c\ge 3\sqrt[3]{abc},\quad \frac{1}{a}+\frac{1}{b}+\frac{1}{c}\ge \sqrt[3]{\frac{1}{abc}}\]
从而:$$\frac{1}{a}+\frac{1}{b}+\frac{1}{c}\ge 3\sqrt[3]{abc}\cdot 3\sqrt[3]{\frac{1}{abc}}=9\sqrt[3]{abc\cdot \frac{1}{abc}}=9$$

\textbf{证法3:}$\because\quad a,b,c$都是正数,

$\therefore\quad \frac{1}{a}+\frac{1}{b}+\frac{1}{c}\ge 3\sqrt[3]{\frac{1}{abc}}=3\cdot \frac{1}{\sqrt[3]{abc}}$ \hfill (3)

又$\because\quad a+b+c=1$,利用$a+b+c\ge 3\sqrt[3]{abc}$可得
\begin{equation}
    1\ge 3\sqrt[3]{abc} \Longrightarrow \frac{1}{\sqrt[3]{abc}}\ge 3 \tag{4}
\end{equation}
由(3)(4)得:$\frac{1}{a}+\frac{1}{b}+\frac{1}{c}\ge 3\cdot \frac{1}{\sqrt[3]{abc}}\ge 3\cdot 3=9$
\end{proof}

\begin{example}
    $a,b,c,d,x,y$都是正数,且$x^2=a^2+b^2$, $y^2=c^2+d^2$,求证:
\begin{equation}
    xy\ge ac+bd \tag{1}
\end{equation}
\end{example}

\begin{analyze}
由条件的结构特征容易使我们联想到直角三角形,从而可以实行“三角换元”。
\end{analyze}

\begin{proof}
    由于$x^2=a^2+b^2,\quad y^2=c^2+d^2$,且$a,b,c,d,x,y$皆正,

设$\begin{cases}
    a=x\cos\alpha\\
    b=x\sin\alpha
\end{cases},\quad \begin{cases}
    c=y\cos\beta\\
    d=y\sin\beta
\end{cases}$(图4.3,$\alpha,\beta$为锐角)

\begin{figure}[htp]
    \centering
\begin{tikzpicture}[scale=.7]
\begin{scope}
\tkzDefPoints{0/0/A, 3/0/B, 3/2/C}
    \draw[very thick](A)--node[below]{$a$}(B)--node[right]{$b$}(C)--node[above]{$x$}cycle;
    \tkzMarkAngle[mark={}](B,A,C)
    \tkzLabelAngle[pos=1.5](B,A,C){$\alpha$}
    \tkzMarkRightAngle(C,B,A)
\end{scope}
\begin{scope}[xshift=5cm]
\tkzDefPoints{0/0/A, 5/0/B, 5/4/C}
    \draw[very thick](A)--node[below]{$c$}(B)--node[right]{$d$}(C)--node[above]{$y$}cycle;
    \tkzMarkAngles[mark={}](B,A,C)
    \tkzMarkRightAngle(C,B,A)
    \tkzLabelAngles[pos=1.5](B,A,C){$\beta$}
\end{scope}
\end{tikzpicture}
    \caption{}
\end{figure}

代入(1),
\[\begin{split}
    \text{右边}= ac+bd&=xy\cos\alpha\cos\beta+xy\sin\alpha\sin\beta\\
    &=xy\cos(\alpha-\beta)\le xy
\end{split}\]
$\therefore\quad $(1)成立.
\end{proof}

\begin{rmk}
    实行三角换元是高中数学中很重要的思想方法。
\end{rmk}

\begin{blk}
    若从结论入手,运用分析法,你能想出例4.19的新证法吗?
\end{blk}

\begin{example}
已知$x,y\in\R$且$x+y+z=2$,$x^2+y^2+z^2=2$,求证:$x,y,z$都不能是负数,也都不能大于$\frac{4}{3}$.
\end{example}

\begin{analyze}
    “元”多,是个难点,但$x$、$y$、$z$地位 对等,只要证出一个元满足关系即可——这就使我们想到消元。消元后出现二元二次方程,从而可用判别式法。
\end{analyze}

\begin{proof}
    由已知条件消去$x$得
$$y^{2}+\left(z-2\right)y+z^{2}-2z+1=0$$
这是关于y的一元二次方程.

$\because\quad y\in \R$,

$\therefore\quad \Delta = ( z- 2) ^{2}- 4\cdot 1\cdot ( z^{2}- 2z+ 1) \geq 0$

即   $-3z^2+4z\geq0$

$\therefore\quad 0\leq z\leq\frac{4}{3}$.

由已知可见$x,y,z$地位对等。同理可得:
$$0\le  y\le \frac{4}{3},\qquad 0\le  z\le \frac{4}{3}.$$
\end{proof}


\begin{example}
    若$x,y,z\in \R$, $A+B+C=\pi$, 求证
\begin{equation}
    x^2+ y^2+ z^2\geqslant 2yz\cos A+ 2zx\cos B+ 2xy\cos C.\tag{1}
\end{equation}
\end{example}

\begin{analyze}
    “元”多是(1)式的特点。若把$x$视为“主元”, 这就
是一元二次不等式.
\end{analyze}

\begin{proof}
欲证(1), 只要证明
\begin{equation}
    x^2-2(z\cos B+y\cos C)x+y^2+z^2-2yz\cos A \geq 0 \tag{2}
\end{equation}
这是关于$x$的二次不等式,

$\because \quad x^2$的系数$>0$,  
  
$\therefore\quad $欲证(2),只要证明$\Delta\le 0$,而
\[\begin{split}
    \Delta &= 4(z\cos B+y\cos C)^2-4\cdot1\cdot(y^2+z^2-2yz\cos A)\\
&= 4[ - z^{2}\sin ^{2}B- y^{2}\sin ^{2}C+ 2y\boldsymbol{z}( \cos A+ \cos B\cos C) ]
\end{split}\]

$\because\quad A+ B+ C= \pi $,

$\therefore\quad \cos A= \cos [ \pi - ( B+ C) ] = - \cos ( B+ C)=-\cos B\cos C+\sin B\sin C$,

代入上式,
\[\begin{split}
    \Delta&=-4[z^{2}\sin^{2}B+y^{2}\sin^{2}C-2yz\sin B\sin C]\\
    &=-4(z\sin B-y\sin C)^{2}\le 0
\end{split}\]
从而(1)式成立.
\end{proof}

\begin{blk}
    对(2)式运用“配方法”能完成证明吗?试试看.
\end{blk}

\begin{example}
若$a,b,c\in\R$,且
\begin{equation}
    a\left(\frac{1}{b}+\frac{1}{c}\right)+
    b\left(\frac{1}{c}+\frac{1}{a}\right)+
    c\left(\frac{1}{a}+\frac{1}{b}\right)+3=0  \tag{1}
\end{equation}
求证:$ab+bc+ca\le 0$
\end{example}

\begin{analyze}
    为充分利用式子的对称性,可把条件中的“3”写成
\[a\cdot \frac{1}{a}+b\cdot \frac{1}{b}+c\cdot \frac{1}{c}\]
\end{analyze}

\begin{proof}
    把条件式写成
\[a\left(\frac{1}{a}+\frac{1}{b}+\frac{1}{c}\right)+b\left(\frac{1}{a}+\frac{1}{b}+\frac{1}{c}\right)+c\left(\frac{1}{a}+\frac{1}{b}+\frac{1}{c}\right)=0\]
即:$(a+b+c)\left(\frac{1}{a}+\frac{1}{b}+\frac{1}{c}\right)=0\Longrightarrow (a+b+c)\cdot \frac{ab+bc+ca}{abc}=0$

\begin{enumerate}
    \item 若$ab+bc+ca=0$,则命题(1)成立;
    \item 若$a+b+c=0$,则$(a+b+c)^2=0$,即$a^2+b^2+c^2+2(ab+bc+ca)=0$
    
    $\therefore\quad ab+bc+ca=-\frac{1}{2}(a^2+b^2+c^2)\le 0$
\end{enumerate}
综上所述,(1)成立。
\end{proof}

\section*{习题七}
\begin{center}
    \bfseries A
\end{center}

\begin{enumerate}
    \item 若$a,b,c\in\R^+$,且$a+b+c=1$,求证:$(1-a)(1-b)(1-c)\ge 8abc$
    \item 若$a,b,c\in\R^+$,且$abc=1$,求证:$(1+a)(1+b)(1+c)\ge 8$
    \item 若$a,b,c\in\R^+$,且$ab+bc+ca=1$,求证:$a+b+c\ge \sqrt{3}$
    \item 若$a,b,c\in\R^+$,且$a+b+c=1$,求证:
\begin{enumerate}[(1)]
    \item $\frac{1}{a^2}+\frac{1}{b^2}+\frac{1}{c^2}\ge 27$
    \item $a^2+b^2+c^2\ge \frac{1}{3}$
    \item $ab+bc+ca\le \frac{1}{3}$
    \item $\left(\frac{1}{a}-1\right)\left(\frac{1}{b}-1\right)\left(\frac{1}{c}-1\right)\ge 8$
\end{enumerate}    
    \item 若$a,b\in\R^+$,且$a+b=1$,求证:$\left(1+\frac{1}{a}\right)\left(1+\frac{1}{b}\right)\ge 9$
    \item 若$a>b>c$,求证:$\frac{1}{a-b}+\frac{1}{b-c}\ge \frac{4}{a-c}$
\end{enumerate}

\begin{center}
    \bfseries B
\end{center}

\begin{enumerate}
 \setcounter{enumi}{6}   
 \item \begin{enumerate}[(1)]
     \item 若$a,b$都是非负数,且$a+b=1$,求证:$1\le \sqrt{a}+\sqrt{b}\le \sqrt{2}$.
     \item 对(1),你还能推广吗?
 \end{enumerate}
 \item 若$a,b,c\in\R^+$,且$a+b+c=3$,求证:$\sqrt{3a-2}+\sqrt{3b-2}+\sqrt{3c-2}\le 3$
 \item 用三角换元法证明(结构上的什么特征使你能联想到三角换元?):
 \begin{enumerate}[(1)]
     \item 若$a^2+b^2=1$,$x^2+y^2=1$,$a,b,x,y\in\R$,则$|ax+by|\le 1$,此题还能推广吗?
     \item 若$x^2+y^2\le 1$,则$|x^2-2xy-y^2|\le \sqrt{2}$
     \item 若$|a|\le 1$, $|b|\le 1$,则$ab+\sqrt{(1-a^2)(1-b^2)}\le 1$
 \end{enumerate}

\item 已知$a,b$都是非负数,且$a+b=1$,求证:$(ax+by)(ay+bx)\ge xy$
\item 若$a>b>0$,且$ab=1$,求证:$\frac{a^2+b^2}{a-b}\ge 2\sqrt{2}$,并指出等号成立的条件.
\item 若$a,b,c\in\R$,$a+b+c=0$,$abc=1$,求证:$a,b,c$中必有一个大于3/2.
\item 设$a\ge b>0$,求$f(x)=(a-b)\sqrt{1-x^2}+ax$的最大值.
\end{enumerate}

\section{*平方平均数}
\begin{thm}{定义}
  $a_1,a_2,\ldots , a_n$是$n$个非负数,$\sqrt{\frac{a^2_1+a^2_2+\cdots+a^2_n}{n}}$叫做这$n$个数的\textbf{平方平均数}。
\end{thm}

\begin{example}
    若$a$,$b$是两个非负数,求证
    \begin{equation}
        \frac{a+b}{2}\le \sqrt{\frac{a^2+b^2}{2}}\tag{1}
    \end{equation}
\end{example}

\begin{proof}
\[\frac{a+b}{2}=\sqrt{\left(\frac{a+b}{2}\right)^2}=\sqrt{\frac{a^2+b^2+2ab}{4}}\le \sqrt{\frac{a^2+b^2+a^2+b^2}{4}}=\sqrt{\frac{a^2+b^2}{2}}\]
(等号当且仅当$a=b$时成立)
\end{proof}

设$0<a<b$,则$a,b$的几何平均、算术平均、平方平均与$a,b$的大小关系如图4.4所示(这有助于你记住它们之间的大小顺序)
\begin{figure}[htp]
    \centering
\begin{tikzpicture}[>=stealth]
    \draw[->, very thick](-1,0)--(10,0)node[below]{$x$};
    \foreach \x/\y in {0/0, 8/b, 2/a, 4/\sqrt{ab}, 5.83/\sqrt{\frac{a^2+b^2}{2}}}
    {
        \draw(\x,0)node[below]{$\y$}--(\x,.1);
    }
    \draw(5,0)--(5,.1)node[above]{$\frac{a+b}{2}$};
\end{tikzpicture}
    \caption{}
\end{figure}

从(1)式的结构特征可以看出$\frac{a+b}{2}$与$\frac{a^2+b^2}{2}$通过放大或缩小在形式上可以互化。这就为解题提供了方便。

若将(1)式中字母的次数推广,可得
\begin{equation}
    \frac{a+b}{2}\le \sqrt[3]{\frac{a^3+b^3}{2}} \tag{2}
\end{equation}

若将(1)式中字母的个数推广,可得
\begin{equation}
    \frac{a+b+c}{3}\le \sqrt{\frac{a^2+b^2+c^2}{3}}\tag{3}
\end{equation}

\begin{example}
若$a>0$, $b>0$, $a+b=1$,求证:
\begin{equation}
    \left(a+\frac{1}{a}\right)^2 +
    \left(b+\frac{1}{b}\right)^2 \ge \frac{25}{2}
\end{equation}
\end{example}

\begin{analyze}
直接展开左边,把条件代入,则运算较繁。(4)式的结构特征启发我们,有可能用“平方平均>算术平均”来证。为此,把(4)变形,先凑出平方平均数。
\end{analyze}

\begin{proof}
    欲证(4),等价于证明
\begin{equation}
    \sqrt{\frac{\left(a+\frac{1}{a}\right)^2 +    \left(b+\frac{1}{b}\right)^2}{2}}\ge \sqrt{\frac{25}{4}}=\frac{5}{2}
\end{equation}
对左边运用“平方平均$\ge $算术平均”,得
\[\sqrt{\frac{\left(a+\frac{1}{a}\right)^2 +    \left(b+\frac{1}{b}\right)^2}{2}}\ge \frac{\left(a+\frac{1}{a}\right) +    \left(b+\frac{1}{b}\right)}{2}=\frac{(a+b)+\left(\frac{1}{a}+\frac{1}{b}\right)}{2}\]

$\because\quad a>0,\; b>0,\; a+b=1$,利用上节的方法,可得:
\[\frac{\left(a+\frac{1}{a}\right) +    \left(b+\frac{1}{b}\right)}{2}\ge \frac{1+4}{2}=\frac{5}{2}\]

$\therefore\quad $(5)成立,从而(4)成立.
\end{proof}

\section*{习题八}
\begin{enumerate}
    \item 若$a+b+1=0$,求证:$\sqrt{(a-1)^2+(b-1)^2}\ge \frac{3}{\sqrt{2}}$.
    \item 若$a,b,c$为非负数,证明:$\frac{a+b+c}{3}\le \sqrt{\frac{a^2+b^2+c^2}{3}}$,再推广一步试试看.
    \item 若$a,b,c\in\R^+$,且$a+b+c=1$,试证:
    \[\left(a+\frac{1}{a}\right)^2+\left(b+\frac{1}{b}\right)^2+\left(c+\frac{1}{c}\right)^2\ge \frac{100}{3}\]
    \item 若$a,b,c\in\R^+$,且$a+b+c=1$,试证:$\frac{a+b}{2}\le \sqrt[3]{\frac{a^3+b^3}{2}}$ (右边称为“立方平均数”)
    
    这一结论再推广可能得到什么?
    \item 若$a\ge b\ge 0$,依照从小到大的顺序用“$\le $”号连接下列各式:
\[a,\; b,\; \frac{a+b}{2},\; \sqrt{ab},\; \sqrt{\frac{a^2+b^2}{2}}\]
\end{enumerate}


\section{同解不等式}

什么是同解方程?关于同解方程有几个基本定理?证明
这些定理的方法是什么?这个问题涉及到解方程的理论根据。

类似于方程,对于不等式我们有

\begin{thm}{定义} 
    如果两个不等式的解集相等,那么,这两个不等式叫做\textbf{同解不等式}。或者简称这两个不等式\textbf{同解},或者\textbf{等价}. 两个不等式$A$,$B$同解(或等价),可以用$A\Longleftrightarrow B$来表示。
\end{thm}

关于同解不等式,有下面几个基本定理。

\begin{thm}{定理1}
    不等式$f(x)>g(x)$与$g(x)<f(x)$同解。
\end{thm}

\begin{analyze}
    根据定义,要证这两个不等式同解,必须证明它们的解集相等。
\end{analyze}

\begin{proof}
\begin{enumerate}[(i)]
    \item 设$x=a$是$f(x)>g(x)$的任何一个解,则$f(a)>g(a)$,由不等式的性质知$g(a)<f(a)$,即$x=a$也是$g(x)<f(x)$解。

    \item 设$x=b$是$g(x)<f(x)$的任何一个解,即有$g(b)<f(b)$,则$f(b)>g(b)$,
    
    $\therefore\quad x=b$也是$f(x)>g(x)$的解.
\end{enumerate}
综合(i)、(ii),知道两个不等式的解集相等。

$\therefore\quad f(x)>g(x)\Longleftrightarrow g(x)<f(x)$
\end{proof}

\begin{thm}{定理2}
    不等式$f(x)>g(x)$与$f(x)+m>g(x)+m$(其中$m$是常数)同解。
\end{thm}

\begin{thm}{推论}
  把定理中的常数$m$换成函数$m(x)$后,若所得不等式与原不等式的未知数的取值范围相同(即不等式两边同加$m(x)$后,不等式的未知数的取值范围既不扩大,也不缩小),则两不等式同解。  
\end{thm}

\begin{note}
\begin{enumerate}[(1)]
\item 用定理1的证明方法可以证明这个定理及推论;
\item 要特别注意推论中对函数式$m(x)$所要求的条件。如不等式
\[2x+8>5x+2\quad \text{与}\quad 22x+8+\frac{4}{x-1}>5x+2+\frac{4}{x-1}\]
就不一定同解。事实上,前一个不等式的未知数的取值范围为实数集$\R$,而后一个的未知数的取值范围是$x\in\R$,且$x\ne 1$. 但是,
\[\frac{3}{x-1}>\frac{x+2}{x-2}\quad \text{与}\quad \frac{3}{x-1}+\frac{3x-4}{x-1}>\frac{x+2}{x-2}+\frac{3x-4}{x-1}\]
是同解的。这是因为前一个不等式两边同加$m(x)=\frac{3x-4}{x-1}$后,两个不等式的未知数的取值范围相同。
\item 定理2及其推论是“移项”的理论根据。
\end{enumerate}
\end{note}

\begin{thm}{定理3}
\begin{enumerate}
    \item 当常数$k>0$时,不等式$f(x)>g(x)$与$kf(x)>kg(x)$同解;
    \item 当常数$k<0$时,$f(x)>g(x)$与$kf(x)<kg(x)$同解.
\end{enumerate}
\end{thm}

\begin{thm}{推论}
    把定理中的常数$k$,分别换成在不等式的未知数的取值范围上解析式$k(x)$的值恒正或恒负时,定理的结论仍成立。    
\end{thm}

\begin{note}
\begin{enumerate}[(1)]
    \item 定理3和推论的证明方法同定理1;
    \item 运用定理3及其推论解不等式时,条件必须掌握好,如:

    解不等式$\frac1x>1$。不等式的未知数的取值范围是$x\neq0$, 用$k\left(x\right)=x$乘两边,这时$k\left(x\right)$在未知数的取值范围上不是恒正或恒负,因此必须对$k(x)$分两种情况进行讨论:
    \begin{enumerate}[(i)]
    \item 当$x>0$时,$\frac1x>1{\Longleftrightarrow}x\cdot\frac1x{>}x{\cdot}1{\Longleftrightarrow}1{>}x$,
    所以$0<x<1$,
    \item 当$x<0$时,$-\frac1x>1\Longleftrightarrow x\cdot\frac1x<x\cdot1\Longleftrightarrow1<x$, 所以无解。
    \end{enumerate}
    综合(i)、(ii), 知原不等式的解是$0<x<1$.
    
    由此可见,这样讨论是较繁的,特别当所用的$k(x)$较复杂时就更繁。因此解分式不等式,我们一般不采用以$k(x)$乘两边(去分母)的办法。
\end{enumerate}
\end{note}

\begin{thm}{定理4}
\begin{enumerate}[(i)]
    \item 不等式 $\frac {f(x)}{g(x)}>0$ 与$f(x)g(x)>0$同解
    \item 不等式$\frac {f(x)}{g(x)}<0$与$f(x)g(x)<0$同解。
\end{enumerate}
\end{thm}

\begin{note}
    定理 4 的证明方法同定理 1, 此定理是将分式不
等式转化为整式不等式的依据。
\end{note}

\begin{thm}{定理5 }
 当$f(x)$与$g(x)$ 在不等式 $f(x)>g(x)$ 的未知数的取值范围上都非负时,$f(x)>g(x)\Longleftrightarrow f^n(x)>g^n(x)$, $n\in \N$.   
\end{thm}

\begin{note}
\begin{enumerate}[(1)]
    \item 定理 5 的证明方法同定理1。
\item 若$f(x), g(x)$在未知数的取值范围上非正,对$f(x)>g(x)$两边同乘$(-1)$变成非负解决之.
\item 这个定理是将无理不等式转化为有理不等式的依据。但在运用定理时,必须掌握好定理的条件。
\end{enumerate} 
\end{note}


    对于这五个基本定理,再做几点说明:
\begin{enumerate}[(1)]
\item 定理1所起的作用如同不等式的性质中定理1一样,有了它,对“$>$”类型的不等式所成立的定理2,对“$<$”类型的不等式仍然成立。
\item 五个定理中,若把“$>$”改成“$\ge $”,定理1、2、3、5仍成立,但定理4不然。
\item 定理2、3、5中的条件,都涉及原不等式中未知数的取值范围。
\end{enumerate}

    以下几节,我们将学习各类代数不等式和指数、对数不等式的解法。作为基础,先研究两个例题。

\begin{example}
    解关于$x$的不等式
$ax>b$, 
其中$a$,$b$为任意实数。
\end{example}

\begin{solution}
\begin{enumerate}[(i)]
    \item 若$a>0$,则$x>\frac{b}{a}$\hfill(定理3)

    $\therefore\quad $解集为$\left(\frac{b}{a},+\infty\right)$
\item 若$a<0$,则$x<\frac{b}{a}$\hfill(定理3)

$\therefore\quad $解集为$\left(-\infty, \frac{b}{a}\right)$
\item 若$a=0,\; b<0$,任取$x\in\R$, $ax>b$式都成立,

$\therefore\quad $解集为$\R$.
\item 若$a=0,\; b\ge 0$,任取$ax>b$都不成立,

$\therefore\quad $解集为$\emptyset$.
\end{enumerate}
上述四种情况在解不等式时,是经常有用的。
\end{solution}

\begin{example}
    解关于$x$的不等式$mx-2>x-3m$ \hfill(1)
\end{example}

\begin{solution}
    $(1)\Longleftrightarrow (m-1)x>2-3m$\hfill (2)
\begin{enumerate}[(i)]
    \item 若$m-1>0$, $x>\frac{2-3m}{m-1}$
    \item 若$m-1<0$, $x<\frac{2-3m}{m-1}$
    \item 若$m-1=0$,即:$m=1$时,

$\because\quad     2-3m=2-3\x1=-1$,

$\therefore\quad  x$为任意实数。   
\end{enumerate}

从而:当$m>1$时,解集为$\left(\frac{2-3m}{m-1},+\infty\right)$;
当$m<1$时,解集为$\left(-\infty, \frac{2-3m}{m-1}\right)$;
当$m=1$时,解集为$\R$.
\end{solution}

\section*{习题九}
\begin{center}
    \bfseries A
\end{center}

\begin{enumerate}
\begin{multicols}{2}
    \item 解不等式组
\[\begin{cases}
    x>2\\ x<13\\ x>5\\ x<8
\end{cases}\]
    \item 解下列不等式组
\begin{enumerate}[(1)]
    \item $\begin{cases}
        2x+4>7x+3\\ 5x+6>6x+5\\ 8x-2<9x-4
    \end{cases}$
    \item $\begin{cases}
        x+2\ge \frac{x-9}{6}+\frac{x+4}{2}\\
        6-\left(\frac{x-2}{4}+\frac{2}{3}\right)\ge \frac{x}{6}
    \end{cases}$

    并求出它的整数解.
    \item $3x-1>2-\frac{x+1}{3}\ge 1-\frac{2x-3}{2}$
    \item $-12\le 3\frac{1}{3}x-5\le 4$
\end{enumerate}
    \item 解下列关于$x$的不等式    
    \begin{enumerate}[(1)]
        \item $ax+b^2>bx+a^2$
        \item $2k-3x>5-kx$
        \item $mx+4<m-2x$
        \item $m(mx-1)<2(2x-1)$
    \end{enumerate}
\end{multicols}
\end{enumerate}


\section{高次不等式的解法}

不等式$(x-x_1)(x-x_2)\cdots(x-x_n)>0$
(其中$x_1,x_2,\ldots,x_n$是互不相等的实常数)是一元$n$次不等式($n\in\N$). 

若$n=1$,容易写出它的解集为$(x_1,+\infty)$;

若$n=2$,它是一元二次不等式。只要能正确地作出相应的二次函数的图象的草图(关键是开口方向和函数的零点不能弄错),再根据图象写出不等式的解集就没有什么困难了。现在的问题是,能否把这种解法——图象法,推广到$n\ge 3$的情况。

很明显,这时关键在于怎样作出相应函数的图象的草图,让我们先剖析一个实例——作出下列函数图象的草图:
\[y=f(x)=(x+1)(x-1)(x+5)\]

第一步,曲线$y=f(x)$与$x$轴的交点的横坐标(称为\textbf{函数$y=f(x)$的零点})由小到大依次是$-5$,$-1$,1,此时$x$轴被这三个交点分成四段,它们分别对应四个开区间:
\[(-\infty,-5),\quad (-5,-1),\quad (-1,1), \quad (1,+\infty)\]

第二步,研究曲线$y=f(x)$在这四个开区间上分布在横轴的上方还是下方:

在$(1,+\infty)$上,由于$x>1$,所以$x+1>0$, $x-1>0$, $x+5>0\Longrightarrow y>0 \Longrightarrow $曲线$y=f(x)$在$x$轴上方;

在$(-1,1)$上,由于$-1<x<1$,所以$x+1>0$, 
$x-1<0$, $x+5>0\Longrightarrow y<0\Longrightarrow $ 曲线$y=f(x)$在$x$轴下方;

在$(-5,-1)$上,由于$-5<x<-1$,所以$x+1<0$, 
$x-1<0$, $x+5>0\Longrightarrow y>0\Longrightarrow$ 曲线$y=f(x)$在$x$轴上方;

在$(-\infty,-5)$上,由于$x<-5$,所以$x+1<0$, $x-1<0$, $x+5<0\Longrightarrow y<0\Longrightarrow$曲线$y=f(x)$在$x$轴下方;

规律性很明显:曲线$y=f(x)$在上述四个彼此相邻的开区间内,从右到左依次位于$x$轴的上方、下方、上方、下方,而在$x=-5,-1,1$三处曲线与$x$轴相交。

第三步,根据上述分析,作出函数图像的草图(图4.5)。

\begin{figure}[htp]
    \centering
\begin{tikzpicture}[>=stealth]
\draw[->](-6,0)--(3,0)node[below]{$x$};    
\draw[->](0,-1.5)--(0,2)node[right]{$y$};

\draw[domain=-5.5:2, smooth, samples=100, very thick]plot(\x, {0.05*(\x-1)*(\x+1)*(\x+5)})node[above]{$y=f(x)$};

\foreach \x in {1,-1,-5}
{
    \draw[fill=white](\x,0)node[below]{$\x$}circle(2pt);
}
\node [above right]{$O$};
\end{tikzpicture}
    \caption{}
\end{figure}


这种草图上要明确地表示出:
\begin{enumerate}[(a)]
    \item 曲线与$x$轴的交点;
    \item 在某个开区间上函数的那段曲线是在$x$轴的上方还是下方。
\end{enumerate}
除此以外,对草图不必做更细致的要求,例如在某个开区间上函数的最大值(最小值)画得是否确切等都无关紧要。因为在此处我们只关心不等式的解。

通过剖析实例,我们获得了作形如函数
$f(x)=(x-x_1)(x-x_2)\cdots (x-x_n)$
(其中$x_1,x_2,\ldots,x_n$是互不相等的实常数)
图像的草图的一般方法:

第一步,求出曲线$f(x)$与$x$轴交点的横坐标$x_1,x_2,\ldots,x_n$,并把它们由小到大依次标在$x$轴上。标出的$n$个点把$x$轴分成$n+1$个开区间。

第二步,因为曲线$y=f(x)$在这$n+1$个开区间上总是依次上下相间地分布在$x$轴两侧,而且在函数的零点处彼此相连。所以,我们只要确定曲线在最右边的开区间上分布在$x$轴的上侧还是下侧就行了。这是容易办到的,实际上我们总能使$x$的最高次幂的系数为正,曲线总是在$x$轴的上方。

第三步,根据上述分析,作出图象的草图。

\begin{example}
    解不等式:
    \begin{multicols}{2}
\begin{enumerate}[(1)]
    \item $(x-2)(x+1)(x+3)(x+5)>0$
    \item $(x+2)(x-3)(5-x)>0$
\end{enumerate}        
    \end{multicols}
\end{example}

\begin{solution}
\begin{enumerate}[(1)]
    \item 作出函数$y=(x-2)(x+1)(x+3)(x+5)$图象的草图(图4.6)

$\therefore\quad $不等式(1)的解集为$(-\infty,-5)\cup(-3,-1)\cup (2,+\infty)$

\begin{figure}[htp]
    \centering
\begin{minipage}{0.45\textwidth}
\begin{tikzpicture}[scale=.7, >=stealth]
\draw[->](-6,0)--(3,0)node[below]{$x$};    
\draw[->](0,-1.5)--(0,2)node[right]{$y$};

\draw[domain=-5.5:2.2, smooth, samples=100, very thick]plot(\x, {0.03*(\x-2)*(\x+1)*(\x+5)*(\x+3)})node[above]{$y=f(x)$};

\foreach \x in {-1,-3,-5,2}
{
    \draw[fill=white](\x,0)node[below]{$\x$}circle(2.5pt);
}
\node [below right]{$O$};


 \end{tikzpicture}
    \caption{}   
\end{minipage}\hfill 
\begin{minipage}{0.45\textwidth}
\begin{tikzpicture}[scale=.6, >=stealth]
    \draw[->](-3,0)--(6,0)node[below]{$x$};    
\draw[->](0,-1.5)--(0,2.5)node[right]{$y$};

\draw[domain=-2.5:5.5, smooth, samples=100, very thick]plot(\x, {0.05*(\x-3)*(\x+2)*(\x-5)})node[above]{$y=f(x)$};

\foreach \x in {-2,3,5}
{
    \draw[fill=white](\x,0)node[below]{$\x$}circle(3pt);
}
\node [below left]{$O$};


 \end{tikzpicture}
    \caption{}   
\end{minipage}
\end{figure}


\item 先把原不等式变成与它等价的$(x+2)(x-3)(x-5)<0$,作出函数$y=(x+2)(x-3)(x-5)$图象的草图(图4.7)

$\therefore\quad $解集为$(-\infty,-2)\cup(3,5)$.

\end{enumerate}

注意:在解题中,我们先以$(-1)$乘原不等式,为的是使因式$(5-x)$变成$(x-5)$。这样做可以避免出错。
\end{solution}

\begin{note}
这类不等式的解法可以概括成:找零点,分区间,画草图,写解集。
\end{note}

\begin{example}
解下列不等式:
\begin{enumerate}[(1)]
    \item $(x-4)(x-1)^2(x+2)<0$
    \item $(x+2)(x+1)^2(x-1)^3(x-3)>0$
\end{enumerate}
\end{example}

\begin{analyze}
    此例中函数的解析式$y=(x-4)(x-1)^2(x+2)$出现了重因式,当$x$值由大于1变到小于1的时候(不含$x=1$),$y$的取值符号没有发生变化,如图4.8所示.

\begin{figure}[htp]
    \centering
\begin{tikzpicture}[>=stealth]
    \draw[->](-3,0)--(5.5,0)node[below]{$x$};    
\draw[->](0,-1)--(0,2)node[right]{$y$};

\draw[domain=-2.5:4.5, smooth, samples=100, very thick]plot(\x, {0.03*(\x-4)*(\x+2)*(\x-1)*(\x-1)});

\foreach \x/\y in {-2/A,1/B,4/C}
{
    \draw[fill=white](\x,0)node[below]{$\x$}circle(1.5pt)node[above]{$\y$};
}
\node [above left]{$O$};
\end{tikzpicture}
    \caption{}
\end{figure}

    由此,不等式(1)的解集为$(-2,1)\cup (1,4)$.

    基于这个想法,不难得到:若$(x-x_1)$是曲线$y=f(x)$的二重因式,则曲线在点$B(x_1)$处不穿过横轴,若$(x-x_1)$是三重因式,则曲线在点$B(x_1)$处穿过横轴,依次类推。

    对于第(2)题,依上述办法作出函数$y=(x+2)(x+1)(x-1)^3(x-3)$的草图(图4.9).由此可得(2)的解集是$(2-1)\cup (-1,1)\cup (3,+\infty)$.

\begin{figure}[htp]
    \centering
\begin{tikzpicture}[>=stealth]
    \draw[->](-3,0)--(4.5,0)node[below]{$x$};    
\draw[->](0,-3)--(0,1)node[right]{$y$};

\draw[domain=-2.3:3.04, smooth, samples=100, very thick]plot(\x, {0.03*(\x+2)*(\x+1)*(\x+1)*(\x-1)*(\x-1)*(\x-1)*(\x-3)});

\foreach \x in {-2,-1,1,3}
{
    \draw[fill=white](\x,0)node[below]{$\x$}circle(1.5pt);
}
\node [below left]{$O$};

\end{tikzpicture}
    \caption{}
\end{figure}
\end{analyze}

\begin{blk}
    如何解下列不等式:
    \begin{enumerate}[(1)]
        \item $(x+3)(x-2)(x^2-2x-3)<0$;
\item $(x-1)(x^2-x+5)>0$.
    \end{enumerate}
\end{blk}


\begin{example}
    解不等式:$(x+3)(x-2)^2(x+1)^2(x-1)\ge 0$.
\end{example}

\begin{analyze}
    这里出现了“$\ge $”。处理这类问题的办法是:在画函数图象的草图时,只要把曲线与$x$轴的交点画成“实点”即可。
\end{analyze}

\begin{solution}
    作出函数
$y=(x+3)(x-2)^2(x+1)^2(x-1)$的草图(图4.10)。

\begin{figure}[htp]
    \centering
\begin{tikzpicture}[>=stealth]
    \draw[->](-3.5,0)--(3.5,0)node[below]{$x$};    
\draw[->](0,-3)--(0,2)node[right]{$y$};

\draw[domain=-3.08:2.7, smooth, samples=100, very thick]plot(\x, {0.03*(\x-2)*(\x-2)*(\x+1)*(\x+1)*(\x+3)*(\x-1)});

\foreach \x/\y in {-3,-1,1,2}
{
    \draw[fill](\x,0)node[below]{$\x$}circle(1.5pt);
}
\node [above right]{$O$};
\end{tikzpicture}
    \caption{}
\end{figure}

$\therefore\quad $不等式的解集是$(-\infty,-3]\cup\{-1\}\cup [1,+\infty)$
\end{solution}

\section*{习题十}
\begin{center}
    \bfseries A
\end{center}

\begin{enumerate}
    \item 解下列二次不等式:
\begin{multicols}{2}
\begin{enumerate}[(1)]
    \item $x^2+4x-45\ge 0$
    \item $x^2-5ax+6a^2>0$
\end{enumerate}
\end{multicols}

\item \begin{enumerate}[(1)]
    \item 解关于$x$的不等式:$ax^2-2ax+a+3\le 0$
\item 关于$x$的不等式$ax^2+5x+b>0$的解集是$\left(\frac13,\frac12\right)$
,求$a,b$

\item 关于$x$的不等式$ax^2+bx+c>0$ 的解集为$(\alpha,\beta)$, $\alpha >0$, 求关于$x$的不等式$cx^2+bx+a<0$的解集。
\end{enumerate} 

\item 解下列不等式组:
\begin{multicols}{2}
\begin{enumerate}[(1)]
    \item $\begin{cases}
        x^2-3x+2>0\\ x^2-2x-3\le 0
    \end{cases}$
    \item $\begin{cases}
        x-a\ge 0\\ x^2-2x-3<0
    \end{cases}$
\end{enumerate}
\end{multicols}
\item $a$为何值时,不等式组$\begin{cases}
    (x-2)(x-5)\le 0\\ x(x-a)\ge 0
\end{cases}$
的解集为:
\begin{multicols}{2}
\begin{enumerate}[(1)]
    \item $\emptyset$;
    \item 单元素集;
    \item 与第一个不等式同解.
\end{enumerate}
\end{multicols}

\item 解下列高次不等式:
\begin{enumerate}[(1)]
    \item $(x+2)(x^{2}-1)>0$;
    \item $(2x+1)(3x-1)(2-x)\leq 0$;
    \item $(x+1)^{3}(x-5)(x^{2}+3x)(x-2)^{2}(2x+1)^{2}<0$;
    \item $x^{2}(x+3)(x-1)(x-2)^{2}(x-3)\ge 0$.
\end{enumerate}

\item 不等式$ax^2+ax+(a-1)<0$对所有的实数$x$都成立,求$a$的取值范围.
\end{enumerate}

\begin{center}
    \bfseries B
\end{center}

\begin{enumerate}\setcounter{enumi}{6}
    \item 已知$M= \{ x\mid x^{2}- 3x- 10\ge 0\} $,
    $N=\{x\mid x^{2}-(2a+3)x+a^{2}+3a<0\}$.

    求$a$的值,使得:(1)$M\cap N=\emptyset$,\qquad (2)$M\cap N=N$.
    
    \item  不等式组$\begin{cases}x^{2}-x-2>0,\\2x^{2}+(2k+5)x+5k<0,\end{cases}$
    的整数解只有$-2$,求$k$的取值范围。
    
    \item  已知不等式$x^{2}-x-6<0$, $x^{2}+2x-8>0$, $x^{2}-4ax+3a^{2}<0$的解集分别是$A,B,C$.
    \begin{enumerate}[(1)]
        \item 试求$a$的取值范围,使$C\supseteq A\cap B$,
        \item 试求$a$的取值范围,使$C\supseteq \overline{A}\cap \overline{B}$.
    \end{enumerate}
    \item 设$A=\{x\mid (x+2)(x-1)>0\}$, $B=\{x\mid ax^2+abx+b\ge 0, \; a\ne 0\}$. 若$A\cap B=\emptyset$,且$A\cup B=A$,求$a$与$b$的关系.
\end{enumerate}

\begin{center}
    \bfseries C
\end{center}

\begin{enumerate}\setcounter{enumi}{10}
    \item 解关于$x$的不等式$ax^2-1<x+a$.
\end{enumerate}

\section{分式不等式的解法}

分母中含有未知数的不等式称为\textbf{分式不等式}。如
\[\frac{1}{x}<2,\qquad \frac{2x+3}{x-1}>x+1\]

解分式不等式的依据是4.10节中的定理4,其基本思想是将分式不等式“\textbf{化归}”与它同解的整式不等式,从而求出它的解集。

\begin{example}
    解不等式$\frac{2x-1}{(x+2)(x-3)}<0$.
\end{example}

\begin{solution}
  \[\text{原不等式}\Longleftrightarrow (2x-1)(x+2)(x-3)<0\]  

$\therefore\quad $解集为$(-\infty,-2)\cup\left(\frac{1}{2},3\right)$.(图4.11)
\end{solution}

\begin{example}
    解不等式$\frac{x^2+x-2}{x^3+7x^2-8x}\ge 0$.
\end{example}

\begin{solution}
    \[\text{原不等式}\Longleftrightarrow \frac{(x+2)(x-1)}{x(x+8)(x-1)}\ge 0 \Longleftrightarrow \begin{cases}
        (x+2)(x-1)^2 x(x+8)\ge 0\\
        x(x+8)(x-1)\ne 0
    \end{cases}   \]  

  $\therefore\quad $原不等式的解集是$(-8,-2]\cup(0,1)\cup (1,+\infty)$.(图4.12)
  \end{solution}

\begin{figure}[htp]
    \centering
\begin{minipage}{.4\textwidth}
\begin{tikzpicture}[>=stealth,scale=.9]
\draw[->](-2.5,0)--(3.5,0)node[below]{$x$};
\draw[->](0,-1.5)--(0,2)node[left]{$y$};
\draw[domain=-2.2:3.2, smooth, very thick]plot(\x, {0.1*(\x+2)*(\x-3)*(2*\x-1)});
\foreach \x in {-2,3}
{
    \draw[fill=white](\x, 0)circle (2pt)node[below left]{$\x$};
}
\draw[fill=white](.5, 0)circle (2pt)node[below left]{$\frac{1}{2}$};
\node[below left]{$O$};
\end{tikzpicture}
\caption{}
\end{minipage}    \hfill
\begin{minipage}{.5\textwidth}
    \begin{tikzpicture}[>=stealth, scale=.5]
\draw[->](-8.5,0)--(2.5,0)node[below]{$x$};
\draw[->](0,-2.5)--(0,3)node[left]{$y$};

\draw[very thick](-8.25,-2)[bend left=5] to (-8,0)[bend left=85]to (-2,0)[bend left=-85] to (0,0)[bend left=85] to (1,0)[bend left=25] to (2.5,1.5);
\foreach \x in {-2,0,1,-8}
{
    \draw[fill=white](\x, 0)circle (4pt)node[below left]{$\x$};
}
    \end{tikzpicture}
    \caption{}
    \end{minipage}    
\end{figure}

\begin{example}
解不等式$\frac{4}{x-1}\le x-1$.
\end{example}

\begin{solution}
\[\begin{split}
    \text{原不等式}&\Longleftrightarrow \frac{4}{x-1}-(x-1)\le 0 \Longleftrightarrow \frac{4-(x-1)^2}{x-1}\le 0\\
    &\Longleftrightarrow \frac{(3-x)(x+1)}{x-1}\le 0 \Longleftrightarrow \frac{(x-3)(x+1)}{x-1}\ge 0\\
    &\Longleftrightarrow \begin{cases}
        (x-3)(x+1)(x-1)\ge 0\\
        x-1\ne 0
    \end{cases}
\end{split}\]

$\therefore\quad $原不等式的解集是$[-1,0)\cup[3,+\infty)$. (图4.13)
\end{solution}

\begin{figure}[htp]
    \centering
\begin{tikzpicture}[>=stealth]
    \draw[->](-2,0)--(4,0)node[below]{$x$};
    \draw[->](0,-1.5)--(0,2)node[left]{$y$};
    \draw[domain=-1.5:3.5, smooth, very thick]plot(\x, {0.3*(\x+1)*(\x-3)*(\x-1)});
    \foreach \x in {-1,3}
    {
        \draw[fill](\x, 0)circle (2pt);
    }
    \draw[fill=white](1, 0)circle (2pt)node[below left]{1};
    \node at (-1,0)[above left]{$-1$};
    \node at (3,0)[below]{$3$};
\node [below left]{$O$};
\end{tikzpicture}
    \caption{}
\end{figure}

\begin{example}
    解不等式$3x+5+\frac{1}{x-4}>2x+\frac{1}{x-4}+3$.
\end{example}

\begin{analyze}
    先消去不等号两边的分式将使解法简化,但消去后的不等式与原不等式不同解,这一点必须特别注意.
\end{analyze}

\begin{solution}
\[\text{原不等式}\Longleftrightarrow \begin{cases}
    3x+5>2x+3\\ x-4\ne 0
\end{cases} \Longleftrightarrow \begin{cases}
    x>-2\\ x\ne 4
\end{cases}\]

$\therefore\quad $原不等式的解集为$(-2,4)\cup (4,+\infty)$.
\end{solution}

\begin{example}
    解不等式$\frac{x-a}{(x+2)(x-3)}<0$\hfill (1)
\end{example}

\begin{solution}
\begin{equation}
    (1)\Longleftrightarrow (x-a)(x+2)(x-3)<0 \tag{2}
\end{equation}

由于$a$的不同取值使方程$(x-a)(x+2)(x-3)=0$的根在$x$轴上的相对位置不确定,故应对字母$a$的取值进行讨论。

\begin{figure}[htp]
    \centering
\begin{minipage}{.45\textwidth}
\begin{tikzpicture}[>=stealth, scale=.75]
\draw[->](-2.5,0)--(5.5,0)node[below]{$x$};
\draw[->](0,-1.5)--(0,3)node[left]{$y$};

\draw(-2.5,-1)[bend left=20] to (-2,0) [bend left=60] to (3,0)[bend right=30] to (4,0)[bend right=20] to (4.8,1.5);


\foreach \x in {-2,3}
{
    \draw[fill=white](\x, 0)circle (2pt)node[below]{$\x$};
}
\draw[fill=white](4, 0)circle (2pt)node[below]{$a$};
\node[below right]{$O$};
\end{tikzpicture}
\caption{}
\end{minipage}    \hfill
\begin{minipage}{.45\textwidth}
    \begin{tikzpicture}[>=stealth, scale=.75]
\draw[->](-2.5,0)--(5.5,0)node[below]{$x$};
\draw[->](0,-1.5)--(0,3)node[left]{$y$};

\draw(-2.5,-1)[bend left=20] to (-2,0) [bend left=60] to (3,0)[bend right=-30] to (4.8,1.5);


\foreach \x in {-2,3}
{
    \draw[fill=white](\x, 0)circle (2pt)node[below]{$\x$};
}
\node[below right]{$O$};
    \end{tikzpicture}
    \caption{}
    \end{minipage}    
\end{figure}


\begin{enumerate}[(i)]
    \item 当$a>3$时(图4.14),解集为$(-\infty,-2)\cup (3,a)$
    \item 当$a=3$时(图4.15),(2)变成$(x-3)^2(x+2)<0$
    
    $\therefore\quad $解集为$(-\infty,-2)$.
    \item 当$-2<a<3$时(图4.16),解集为$(-\infty,-2)\cup (a,3)$
    \item 当$a=-2$时(图4.17),(2)变成$(x+2)^2(x-3)<0$,解集为$(-\infty,-2)\cup (-2,3)$

\begin{figure}[htp]
    \centering
\begin{minipage}{.45\textwidth}
\begin{tikzpicture}[>=stealth, scale=.75]
\draw[->](-2.5,0)--(4.5,0)node[below]{$x$};
\draw[->](0,-1.5)--(0,3)node[left]{$y$};

\draw(-2.5,-1)[bend left=20] to (-2,0) [bend left=60] to (2,0)[bend right=30] to (3,0)[bend right=20] to (3.8,1);


\foreach \x in {-2,3}
{
    \draw[fill=white](\x, 0)circle (2pt)node[below]{$\x$};
}
\draw[fill=white](2, 0)circle (2pt)node[below]{$a$};
\node[below right]{$O$};
\end{tikzpicture}
\caption{}
\end{minipage}    \hfill
\begin{minipage}{.45\textwidth}
    \begin{tikzpicture}[>=stealth, scale=.75]
\draw[->](-3,0)--(5,0)node[below]{$x$};
\draw[->](0,-2.5)--(0,2)node[left]{$y$};

\draw(-2.75,-.75)[bend left=-20] to (-2,0) [bend right=60] to (3,0)[bend right=10] to (4.5,1.5);


\foreach \x in {-2,3}
{
    \draw[fill=white](\x, 0)circle (2pt);
}
\node[below left]{$O$};
\node at (-2,0)[above]{$-2$};
\node at (3,0)[below]{$3$};


    \end{tikzpicture}
    \caption{}
    \end{minipage}    
\end{figure}
    \item 当$a<-2$时(图4.18),解集为$(-\infty,a)\cup (-2,3)$
\end{enumerate}

\begin{figure}[htp]
    \centering
\begin{tikzpicture}[>=stealth, scale=.8]
\draw[->](-6,0)--(4.5,0)node[below]{$x$};
\draw[->](0,-2)--(0,1.5)node[left]{$y$};

\draw(-5.5,-1)[bend left=20] to (-5,0) [bend left=50] to (-2,0)[bend right=50] to  (3,0)[bend left=-20] to (3.5,1);

\draw[fill=white](-2, 0)circle (2pt)node[below left]{$-2$};
\draw[fill=white](-5, 0)circle (2pt)node[below]{$a$};
\draw[fill=white](3, 0)circle (2pt)node[below]{$3$};
\node[above right]{$O$};
\end{tikzpicture}
    \caption{}
\end{figure}

\end{solution}

\section*{习题十一}
\begin{center}
    \bfseries A
\end{center}

\begin{enumerate}
    \item 解下列不等式:
\begin{multicols}{2}
\begin{enumerate}[(1)]
    \item $3x-2+\frac{1}{5-x}>2x+1-\frac{1}{x-5}$
    \item $\frac{2x-3}{3x-4}<2$
    \item $\frac{x(x-3)}{x^2-3x+2}<0$
    \item $\frac{x^2}{x^2-6x+8}\ge 1$
    \item $\frac{x+1}{(x-2)^2 (x+1)}<1$
    \item $2+\frac{2}{x-1}\le \frac{5}{4-x}$
\end{enumerate}
\end{multicols}

\item (选择题)下列不等式中,与$\frac{x-3}{2-x}\ge 0$同解的是(\qquad )
\begin{multicols}{2}
\begin{enumerate}[(A)]
    \item $(x-3)(2-x)\ge 0$
    \item $(x-3)(2-x)>0$
    \item $\frac{2-x}{x-3}\ge 0$
    \item $\lg(x-2)\le 0$
\end{enumerate}
\end{multicols}
\end{enumerate}

\begin{center}
    \bfseries C
\end{center}
\begin{enumerate}
 \setcounter{enumi}{2}   
 \item 解关于$x$的不等式$5^{\tfrac{a(1-x)}{x-2}+1}<1$
\end{enumerate}

\section{无理不等式的解法}
在根号内含有未知数的不等式称为\textbf{无理不等式}。如$\sqrt{x-1}>x-3$, $\sqrt{x}+2<\sqrt{2x-1}+1$等.

解无理不等式应注意:
\begin{enumerate}[(1)]
\item 必须使出现在不等式中的根式有意义,这就需要求出根式中函数的定义域;
\item 无理不等式的求解,根本是\textbf{化归}为有理不等式,转化的依据是4.10节中的定理5。
\end{enumerate}

以下研究几个例子。

\begin{example}
解不等式$\sqrt{x-1}>x-3$\hfill (1)
\end{example}

\begin{analyze}
\begin{enumerate}
    \item 为使用定理5,应对有理式$x-3$进行讨论。
    \item (1)式的结构特征使我们想到换元法。
    \item (1)式两边都是我们熟知的函数。
\end{enumerate}
\end{analyze}


\begin{solution}
\textbf{解法1:}
\[(1)\Longleftrightarrow {\rm (I)}\begin{cases}
    x-3\ge 0& \text{(限制条件)}\\
x-1\ge 0 &\text{(定义域)}\\
x-1>(x-3)^2
\end{cases} \text{和}\quad {\rm (II)}\begin{cases}
    x-3<0& \text{(限制条件)}\\
    x-1\ge 0 &\text{(定义域)}\\
\end{cases}\]
而
\[{\rm (I)}\Longleftrightarrow \begin{cases}
    x\ge 3\\
    x-1>(x-3)^2
\end{cases}\Longleftrightarrow \begin{cases}
    x\ge 3\\
    (x-2)(x-5)<0
\end{cases}\]
$\therefore\quad 3\le x<5$.

\[{\rm (II)}\Longleftrightarrow \begin{cases}
    x< 3\\
    x\ge 1
\end{cases}\qquad \therefore\quad 1\le x<3\]

$\therefore\quad $(1)的解集为$[3,5)\cup[1,3)$,即$[1,5)$.

\textbf{解法2:}
(1)即$\sqrt{x-1}>(x-1)-2$.

令$t=\sqrt{x-1}\ge 0$,得$\begin{cases}
    t\ge 0\\t^2-t-2<0
\end{cases}$

解之,得$0\le t<2$,即$0\le \sqrt{x-1}<2\Longleftrightarrow \begin{cases}
    x-1\ge 0\\ x-1<4
\end{cases}$

$\therefore\quad 1\le x<5$.

\textbf{解法3:}
令$y_1=\sqrt{x-1}$,$y_2=x-3$,从而(1)的解集就是使函数$y_1>y_2$的$x$的取值范围。在同一个坐标系中分别作出两个函数的图象(图4.19). 设它们交点的横坐标是$x_0$,则$\sqrt{x_0-1}=x_0-3>0$

解之,得$x_0=2$(舍)或$x_0=5$

$\therefore\quad $(1)的解集为$[1,5)$.

\begin{figure}[htp]
    \centering
\begin{tikzpicture}[>=stealth, scale=.8]
\draw[->](-1,0)--(7,0)node[below]{$x$} ;   
\draw[->](0,-4)--(0,4)node[left]{$y$} ;
\draw[domain=1:6, smooth, samples=100, very thick]plot(\x, {sqrt(\x-1)})node[right]{$y_1=\sqrt{x-1}$};
\draw[domain=-.5:6, smooth, very thick]plot(\x, {\x-3});
\node at (1.5,-2)[right]{$y_2=x-3$};
\foreach \x in {-3,-2,-1,1,2,3}
{
    \draw(0,\x)--(.1,\x);
}
\foreach \x in {1,3}
{
    \draw(\x,0)node[below]{\x}--(\x,.1);
}
\draw[dashed](5,0)node[below]{$x_0$}--(5,2);
\node [below left]{$O$};

\draw[fill](1,0) circle (2pt);

\end{tikzpicture}
    \caption{}
\end{figure}

\end{solution}

\begin{rmk}
解法1是通法,应熟练掌握。换元法与图象法的突破口是认清式子的结构特征。
\end{rmk}

\begin{example}
解不等式$(x-1)\sqrt{x+2}\ge 0$\hfill (1)
\end{example}

\begin{solution}
\[(1)\Longleftrightarrow \begin{cases}
    (x-1)\sqrt{x+2}>0& (2)\\
    (x-1)\sqrt{x+2}=0 & (3)
\end{cases}\]
\[(2)\Longleftrightarrow \begin{cases}
    x+2\ge 0\\ x-1>0 
\end{cases}\qquad \therefore\quad x>1\]
(3)的解集是$x=1$或$x=-2$.

$\therefore\quad $(1)的解集为$[1,+\infty)\cup\{-2\}$.
\end{solution}

\begin{example}
解不等式$\sqrt{2ax-a^2}>a-x\; (a>0)$\hfill (1)
\end{example}

\begin{solution}
\[(1)\Longleftrightarrow {\rm (I)}\begin{cases}
    a-x\ge 0\\
    2ax-a^2\ge 0\\
    2ax-a^2>(a-x)^2
\end{cases}\text{和}\quad {\rm (II)}\begin{cases}
    a-x<0\\ 2ax-a^2\ge 0
\end{cases}\]
而
\[\begin{split}
{\rm (I)}&\Longleftrightarrow \begin{cases}
    a-x\ge 0\\ 2ax-a^2>(a-x)^2
\end{cases}\Longleftrightarrow \begin{cases}
    x\le a\\(x-2a)^2<2a^2
\end{cases}\\
&\Longleftrightarrow \begin{cases}
    x\le a\\|x-2a|<\sqrt{2}a
\end{cases}\Longleftrightarrow \begin{cases}
    x\le a\\ 2a-\sqrt{2}a<x\le 2a+\sqrt{2}a
\end{cases}
\end{split}\]
$\therefore\quad 2a-\sqrt{2}a<x\le a\quad (\because\quad a>0)$

\[{\rm (II)}\Longleftrightarrow \begin{cases}
    x>a\\ x\ge \frac{a}{2}
\end{cases}\qquad \therefore\quad x>a\quad (\because \quad a>0)\]
$\therefore\quad $(1)的解集是$(2a-\sqrt{2}a,a]\cup (a,+\infty)=\left((2-\sqrt{2})a,+\infty\right)$
\end{solution}


\section*{习题十二}
\begin{center}
    \bfseries A
\end{center}
解下列不等式(1—8题)
\begin{enumerate}
    \item $4x-3+\sqrt{10-x}>3x+2+\sqrt{10-x}$
    \item $\sqrt{3-x}>x-2$
    \item $\sqrt{2x^{2}-6x+4}<x+2$
    \item $\sqrt{3x-15}-\sqrt{x-4}\ge 0$
    \item $\sqrt{4-\log_{0.3}x}<\log_{0.3}x-2$ (提示:令$\log_{0.3}x=y$)
    \item $\sqrt{x-2}-\sqrt{x-5}>1$
\end{enumerate}


\begin{center}
    \bfseries B
\end{center}

\begin{enumerate}\setcounter{enumi}{6}
    \item $\sqrt{\log_{2}^{2}x+\log_{2}x-2}> 2\log_{2}x-2$
    \item $(x-1)\sqrt{x^{2}-x-2}\ge 0$
    \item 设$A= \{ x\mid 5- x> \sqrt {2(x- 1) }\}$, 
$B=\left\{x\mid x^{2}-ax\le x-a\right\}$,
要使$A\subset B$, 求实数$a$的取值范围。
\item 解关于$x$的不等式$\sqrt{2x-a}<\sqrt{x+1}$
\end{enumerate}


\begin{center}
    \bfseries C
\end{center}

\begin{enumerate}\setcounter{enumi}{10}
    \item 解关于$x$的不等式$\sqrt{a-2x}>a-x$(提示:参考例1的解法2).
\end{enumerate}

\section{绝对值不等式的性质、解法与证明}

在绝对值符号中含有未知数的不等式称为\textbf{绝对值不等式}。如
$|x-2|<3$, $|x-1|-|x+2|\ge 5$等。

\begin{thm}{实数绝对值的定义}
若$a\in\R$,那么
\[|a|=\begin{cases}
    a, &a>0\\
    0, &a=0\\
    -a, &a<0\\
\end{cases}\]
显然有$$-|a|\le a\le |a|$$
\end{thm}

\begin{thm}
    {最简绝对值不等式的解集(填空)}
\begin{enumerate}[(1)]
\item 若$|x|<R$,则$x\in \blank$;
\item 若$|x|>r,\; (r>0)$,则$x\in\blank$;
\item 若$r<|x|<R,\; (0<r<R)$,则$x\in\blank$.
\end{enumerate}
\end{thm}

\begin{note}
\begin{enumerate}[(i)]
    \item 若把$x$看作数轴上点$P$的坐标,则$|x|$就是点$P$到原点$O$的距离。那么,上述三个最简绝对值不等式的解集的几何意义十分明显。
    \item 把三个最简绝对值不等式写成与它等价的解集的形式是脱去绝对值符号的重要方法,应该熟练掌握。
    \item 为了“脱去”绝对值号,有时也采用不等式两边分别平方的办法。如
\[|A|\ge |B|\Longleftrightarrow A^2\ge B^2\]
(这是因为$|A|\ge |B|\ge 0$)
\item 这里的正数$R$、$r$不仅可以放宽到零和负数,而且进一步还可以放宽为含未知数的解析式(这无疑会给解题带来极大的方便):
\begin{itemize}
    \item 推广1:$|x|<\varphi(x)\Longleftrightarrow -\varphi(x)<x<\varphi(x)$;
    \item 推广2:$|x|>\varphi(x)\Longleftrightarrow x<-\varphi(x)\text{或}x>\varphi(x)$.
\end{itemize}
(用5.11中定理1的证法可以证明这两个推广)
\end{enumerate}
\end{note}

\begin{example}
解不等式$|x^2+3x-8|\le 10$\hfill (1)
\end{example}

\begin{solution}
    $(1)\Longleftrightarrow -10\le x^2+3x-8\le 10 $\hfill (2)
\[\begin{split}
  (2) & \Longleftrightarrow \begin{cases}
        -10\le x^2+3x-8\\
        x^2+3x-8\le 10
    \end{cases}
     \Longleftrightarrow \begin{cases}
        x^2+3x+2\ge 0\\
        x^2+3x-18\le 0
    \end{cases}\\
    &\Longleftrightarrow \begin{cases}
        (x+2)(x+1)\ge 0,&(3)\\
        (x+6)(x-3)\le 0,&(4)
    \end{cases}
\end{split}\]

很清楚(图4.20), 不等式组(3)、(4)的解集为$[-6,-2]\cup [-1,3]$.
\end{solution}

\begin{figure}[htp]
    \centering
\begin{tikzpicture}[>=stealth, scale=.7]
\draw[->](-7,0)--(4,0)node[below]{$x$} ;   
\draw[->](0,-3)--(0,2)node[left]{$y$} ;
\draw[domain=-2.5:-.5, smooth, very thick]plot(\x, {(\x+2)*(\x+1)})node[above]{$(3)$};
\draw[domain=-6.5:3.5, smooth, very thick]plot(\x, {0.1*(\x+6)*(\x-3)})node[above]{$(4)$};
\foreach \x in {-6,-2}
{
    \node at (\x,0)[below left]{$\x$};
    \draw[fill](\x,0) circle(2pt);
}
\foreach \x in {-1,3}
{
    \node at (\x,0)[below right]{$\x$};
    \draw[fill](\x,0) circle(2pt);
}
\node[below right]{$O$};
\end{tikzpicture}
    \caption{}
\end{figure}

\begin{example}
    解不等式$|5x-x^2|>6$\hfill (1)
\end{example}

\begin{solution}
(1)即 $|x^2-5x|>6$\hfill (*)

(习惯上,我们总是先按x的降幂排列,并使x的最高次项的系数为正)。

\[\begin{split}
    (*)&\Longleftrightarrow x^2-5x<-6,\text{或}x^2-5x>6 \\
    &\Longleftrightarrow x^2-5x+6<0,\text{或}x^2-5x-6>0\Longleftrightarrow \begin{cases}
        (x-2)(x-3)<0,& (2)\\
        (x-6)(x+1)>0,& (3)
    \end{cases}
\end{split} \]
(2)的解集是$(2,3)$;(3)的解集是$(-\infty,-1)\cup (6,+\infty)$.

$\therefore\quad (1)$的解集为$(2,3)\cup (-\infty,-1)\cup (6,+\infty)$
\end{solution}

\begin{example}
    解不等式$3\le |5-2x|<9$\hfill (1)
\end{example}

\begin{solution}
先把(1)改写成 $3\le |2x-5|<9$\hfill (*)

根据最简绝对值不等式(3),有
\[(*)\Longleftrightarrow \begin{cases}
    -9< 2x-5\le -3,& (2)\\
    3\le 2x-5<9,&(3)
\end{cases}\]

$\therefore\quad (1)$的解集为$(-2,1]\cup[4,7)$.
\end{solution}

\begin{rmk}
上述三例,都是利用最简绝对值不等式的解集脱去了绝对值符号,这是最常用也是最基本的方法。若要用“先平方”的办法脱去绝对值号,运算量一般都较大,而且有时还有可能增解,这时必须检验。
\end{rmk}

\begin{example}
    解不等式$\left|x^2-4\right|\le  x+2$\hfill (1)
\end{example}

\begin{solution}
    根据推广1,$(1)\Longleftrightarrow-x-2\le  x^{2}-4\le  x+2$
即
\[\begin{cases}x^{2}-4\geqslant-x-2,\\
    x^{2}-4\leqslant x+2,
\end{cases}\Longleftrightarrow
\begin{cases}
    x^{2}+x-2\geqslant0,\\
    x^{2}-x-6\leqslant0,
\end{cases}\Longleftrightarrow\begin{cases}(x+2)(x-1)\geqslant0,\\
    (x+2)(x-3)\leqslant0.
\end{cases}\]
$\therefore\quad $ (1)的解集为$[1,3]\cup \{-2\}$.
\end{solution}

\begin{thm}{思考题}
    用图象法解此题行吗?试一试。
\end{thm}

\begin{example}
    解不等式$\left|\frac{x+3}{2x-1}\right|\leqslant1$
\end{example}

\begin{solution}
\textbf{解法1:}原不等式同解于:
\[\begin{split}
    &|x+3|\leqslant|2x-1|, \text{ 且 } 2x-1\neq0,\\
&\Longleftrightarrow (x+3)^{2}\leqslant(2x-1)^{2}\text{ 且 }2x-1\neq0,\\
&\Longleftrightarrow 3x^{2}-10x-8\geqslant 0\text{ 且 }2x-1\neq0\\
&\Longleftrightarrow(3x+2)(x-4)\geqslant0\text{ 且 }2x-1\neq0.
\end{split}\]

$\therefore\quad $(1)的解集为$\left(-\infty,-\frac23\right]\cup[4,+\infty)$.    

\textbf{解法2:}
\[\begin{split}
    (1)&\Longleftrightarrow \left(\frac{x+3}{2x-1}\right)^2-1\le 0\\
    &\Longleftrightarrow \frac{3x^2-10x-8}{(2x-1)^2}\ge 0 \Longleftrightarrow \begin{cases}
        (3x+2)(x-4)\ge 0\\
        (2x-1)^2\ne 0
    \end{cases}
\end{split} \]
$\therefore\quad $(1)的解集为$\left(-\infty,-\frac{2}{3}\right]\cup [4,+\infty)$.
\end{solution}

\begin{example}
    解不等式$|x+7|-|x-2|<3$\hfill (1)
\end{example}

\begin{analyze}
这里出现了两个绝对值号,上述三个最简绝对值不等式的结果已不能使用,只好根据“实数绝对值的定义”去脱绝对值号。为此,需要分别求出$|x+7|$与$|x-2|$的零点(使函数值为零的$x$值),再以诸零点为边界,把$(-\infty,+\infty)$分成几个相互连接的区间,然后在每个区间上去探求(1)的解。也就是说,把在实数集$\R$上解不等式(1),转化成在
$\R$的子区间上分别去解(1)。这就是此法的实质。

\end{analyze}

\begin{solution}
    由于$-7$, 2分别是$|x+7|$与$|x-2|$的零点,它们把区间$(-\infty,+\infty)$分成三个子区间(图4.21)。

\begin{figure}[htp]
    \centering
\begin{tikzpicture}[>=stealth, scale=.6]
    \draw[->](-9,0)--(4,0)node[below]{$x$};
\foreach \x in {-7,2}
{
    \node at (\x,0)[below]{$\x$};
    \draw[fill](\x,0) circle(2pt);
}    
\draw(-9,.5)--node[above]{$x<-7$}(-7.5,.5)[bend left=45] to (-7,0);
\draw(2,0)[bend left=-45] to (2-.5,.5)--node[above]{$-7\le x<2$}(-6.5,.5)[bend left=-45] to (-7,0);
\draw(4,.5)--node[above]{$x\ge 2$}(2.5,.5) [bend left=-45] to (2,0);
\end{tikzpicture}
    \caption{}
\end{figure}

由此,(1)可化成三个不等式组:
\begin{enumerate}[(1)]
    \item $\begin{cases}x\geqslant2,\\(x+7)-(x-2)<3,\end{cases}\Longleftrightarrow\begin{cases}x\geqslant2,\\9<3.\end{cases}$
    
    $\therefore\quad$ 解集$X_1=\emptyset$

    \item $\begin{cases} - 7\leq x< 2, \\ ( x+ 7) + ( x- 2) < 3, \end{cases} \Longleftrightarrow \begin{cases} - 7\leq x< 2, \\ x< - 1.  \end{cases}$
    
    $\therefore\quad$ 解集$X_2=[-7,-1)$
    
\item $\begin{cases} x< - 7, \\ - ( x+ 7) - [ - ( x- 2) ] < 3. 
\end{cases} 
\Longleftrightarrow \begin{cases} x< - 7, \\ - 9< 3. \end{cases}$

$\therefore\quad$ 解集$X_3=(-\infty,-7)$.
\end{enumerate}

$\therefore\quad$(1)的解集为$X_1\cup X_2\cup X_3=(-\infty,-1)$.
\end{solution}

\begin{thm}{思考题}
\begin{enumerate}[(1)]
    \item 根据(1)式的几何意义,你能通过画出数轴直接看出它的解集吗?
    \item 根据几何意义,你能判断不等式$|x|+|x-3|<1$无解吗?
\end{enumerate}
\end{thm}

\begin{example}
    解不等式$\frac{\log_{0.1}|x-2|}{x^2-4x}<0$\hfill (1)
\end{example}

\begin{solution}
\[\begin{split}
    (1)&\Longleftrightarrow \begin{cases}
        \log_{0.1}|x-2|>0\\
        x^2-4x<0
    \end{cases}\text{或}\quad \begin{cases}
        \log_{0.1}|x-2|<0\\
        x^2-4x>0
    \end{cases}\\
    &\Longleftrightarrow \begin{cases}
        0<|x-2|<1\\
        x(x-4)<0
    \end{cases}\text{(图4.22)\quad 或}\quad \begin{cases}
        |x-2|>1 \\
        x(x-4)>0
    \end{cases}\text{(图4.23)}\\
\end{split}\]

\begin{figure}[htp]
    \centering
\begin{minipage}{.45\textwidth}
\begin{tikzpicture}[>=stealth, scale=.7]
\draw[->](-1,0)--(5,0)node[below]{$x$};
\draw[->](0,-2)--(0,2)node[left]{$y$};
\node[below left]{$O$};
\foreach \x in {1,2,3,4}
{
    \node at (\x,0)[below]{$\x$};
}
\draw(1,0)[bend left=45] to (1.5,.5)--(3-.5,.5)[bend left=45] to (3,0);
\draw[domain=-1:5, smooth, very thick]plot(\x, {0.3*\x*(\x-4)});

\foreach \x in {0,1,3,4}
{
    \draw[fill=white](\x,0)circle (2pt);
}
\draw[fill](2,0)circle (2pt);


\end{tikzpicture}
\caption{}
\end{minipage}
\hfill
\begin{minipage}{.45\textwidth}
\begin{tikzpicture}[>=stealth, scale=.7]
    \draw[->](-1,0)--(5,0)node[below]{$x$};
    \draw[->](0,-2)--(0,2)node[left]{$y$};
    \node[below left]{$O$};
    \foreach \x in {1,2,3,4}
{
    \node at (\x,0)[below]{$\x$};
}
\draw(3,0)[bend left=45] to (3.5,.5)--(5,.5);
\draw(-1,.5)--(.5,.5)[bend left=45] to (1,0);
\draw[domain=-1:5, smooth, very thick]plot(\x, {0.3*\x*(\x-4)});

\foreach \x in {0,1,3,4}
{
    \draw[fill=white](\x,0)circle (2pt);
}
\draw[fill](2,0)circle (2pt);
\end{tikzpicture}
\caption{}
\end{minipage}
\end{figure}

$\therefore\quad $(1)的解集为$(1,2)\cup (2,3)\cup (-\infty,0)\cup (4,+\infty)$.

\end{solution}

\begin{example}
    用简捷方法解$|x^2-2|<|x|$ \hfill (1)
\end{example}

\begin{solution}
\textbf{解法1:}(1)即$\Big||x|^2-2\Big|<|x|$

令$|x|=y$,则上式$\Longleftrightarrow |y^2-2|<y\Longleftrightarrow 1<y<2$

$\therefore\quad 1<|x|<2$.

从而(1)的解集为$(-2,-1)\cup (1,2)$.

\textbf{解法2:} $\because\quad $不等号两边的函数都为偶函数,

$\therefore\quad $若$x$为(1)的解,则$-x$也为(1)的解。

当$x\ge 0$时,$(1)\Longleftrightarrow |x^2-2|<x$,
解之得$1<x<2$

$\therefore\quad $当$a\le 0$时,必有$-2<x<-1$, 
从而,(1)的解集为$(-2,-1)\cup (1,2)$.
\end{solution}

\section*{习题十三}
\begin{center}
    \bfseries A
\end{center}

\begin{enumerate}
    \item 解不等式:
\begin{multicols}{2}
\begin{enumerate}[(1)]
    \item $|x+1|<\sqrt{2}$
    \item $1<|x-4|\leqslant2$
    \item $| x^{2}-2x-3|-2>0$
    \item $3\leqslant|5-2x|<9$
    \item $|x^{2}+2x-1|<2$
    \item $|x^{2}-1|\geqslant x$
    \item $|x^{3}-1|>1-x$
\end{enumerate}
\end{multicols}

\item 解不等式:
\begin{multicols}{2}
\begin{enumerate}[(1)]
    \item $|x+6|-|3-2x|>4$
    \item $\sqrt{4-4x+x^{2}}+|x-3|-1\leq0$
    \item $\sqrt{x^{2}+2x+1}-2|2-x|>5-x$
    \item $|x|+|x-3|\leqslant1$
\end{enumerate}
\end{multicols}

\item 求$|x-a|+|x-b|+|x-c|$的最小值。(提示:从几何意义上考虑最简便)
\end{enumerate}






































































































































































 \chapter{数列、数学归纳法、数列的极限}

\section*{一、数列的一般概念}
\section{数列的定义}

\noindent
\begin{minipage}{.5\textwidth}
    \CTEXindent
我们看下面的例子:

图5.1表示堆放的钢管,
共堆放了7层,自上而下各层
的钢管数排列成一列数:
\[4,\quad 5,\quad 6,\quad 7,\quad 8,\quad 9,\quad 10\]

自然数$1,2,3,4,5,\ldots$的倒数排列成一列数:
\[1,\quad \frac{1}{2},\quad \frac{1}{3},\quad \frac{1}{4},\quad \frac{1}{5},\quad\ldots\]
\end{minipage}\hfill
\begin{minipage}{.4\textwidth}
    \centering
\begin{tikzpicture}
  \foreach \x in {1,2,3,...,10}
{
    \draw[ultra thick](\x/2,0)circle(.23);
}  
\foreach \x in {1,2,3,...,9}
{
    \draw[ultra thick](\x/2+.25,.45)circle(.23);
}  
\foreach \x in {1,2,3,...,8}
{
    \draw[ultra thick](\x/2+.5,.9)circle(.23);
}  
\foreach \x in {1,2,3,...,7}
{
    \draw[ultra thick](\x/2+.75,.45*3)circle(.23);
}  
\foreach \x in {1,2,3,...,6}
{
    \draw[ultra thick](\x/2+1,.9*2)circle(.23);
}  
\foreach \x in {1,2,3,...,5}
{
    \draw[ultra thick](\x/2+1.25,.45*5)circle(.23);
}  
\foreach \x in {1,2,3,4}
{
    \draw[ultra thick](\x/2+1.5,.9*3)circle(.23);
}  



\end{tikzpicture}
\captionof{figure}{ }

\end{minipage}

$\sqrt{2}$的精确到$1, 0.1, 0.01, 0.001,\ldots$的不足近似值
排列成一列数:
\[1,\quad 1.4,\quad 1.41,\quad 1.414,\quad \ldots\]    

$-1$的1次幂,2次幂,3次幂,4次幂,……排列成一列
数:
\[-1,\quad 1,\quad -1,\quad 1, \quad\ldots\]

再看下面的例子:

函数$y=\frac{1}{x^{3}}$, 当$x$依次取$1,2,3,\ldots,n\; (n\in\N)$时,得到一列数:
\[1, \frac 18, \frac {1}{27}, \ldots, \frac {1}{m^{3}}\]

函数$y=x^{2}-1$, 当$x$依次取$1,2,3,4,\ldots,n\; (n\in \N)$
时,得到一列数
$$0,\; 3,\; 8,\; 15,\ldots,\; n^{2}-1$$

象上面这些例子,按一定顺序排列的一列数叫做\textbf{数列}。数列中的每一个数都叫做这个数列的\textbf{项},各项依次叫做这个数列的第1项(或首项),第2项,第3项,……,第$n$项.

在一个数列中,它的任一项的数值,由它所对应的项数唯一确定,因此数列中各项的值是其项数的 函数,即$a_n=f(n)$, 其中$a_n$表示第$n$项。$n$为自变量$(n\in\mathbb{N})$, 第$n$项为$a_n$的数列记作$\{a_n\}$.

\section{数列的表示法}
数列实质上就是其定义域是自然数集$\N$(或$\N$的有限子
集$\{1,2,\ldots,n\}$)的函数,因此数列的表示方法与过去所学
的函数的表示方法完全类似。

\subsection{列表法表示数列}
将数列的各项依次列举出来:
\[a_1,a_2,a_3,a_4,\ldots, a_n, \ldots\]
其中$a_n$
表示数列第$n$项的数值,$n$是它的项数。显然$a_n$是$n$的函
数。

\subsection{图象法表示数列}

\begin{figure}[htp]
    \centering
\begin{tikzpicture}[>=stealth, scale=.5]
\begin{scope}
\draw[->](-1,0)--(7,0)node[below]{$n$};
\draw[->](0,-3)--(0,8)node[left]{$a_n$};
\foreach \x/\y in {3/1,4/3,5/5,6/7}
{
    \draw[dashed](0,\y)node[left]{$\y$}--(\x, \y)--(\x,0)node[below]{$\x$};
}
\foreach \x/\y in {1/-2,2/-1}
{
    \draw[dashed](0,\y)node[left]{$\y$}--(\x, \y)--(\x,0)node[above]{$\x$};
}
\node[below left]{$O$};
\node at (4,-2){(1)};
\end{scope}
\begin{scope}[xshift=12cm]
    \draw[->](-1,0)--(7,0)node[below]{$n$};
\draw[->](0,-1)--(0,7)node[left]{$a_n$};
\foreach \x in {1,2,3,4,5,6}
{
    \draw[dashed](\x,0)node[below]{$\x$}--(\x,{5-5/(\x+1)})--(0,{5-5/(\x+1)});
}
\node[below left]{$O$};
\node at (3,-2){(2)};
\node at (0,2.5)[left]{$\tfrac{1}{2}$};
\node at (0,4.16)[left]{$\tfrac{5}{6}$};
\draw(0,5)node[left]{1}--(.1,5);

\end{scope}
\end{tikzpicture}
    \caption{}
\end{figure}

图5.2(1)表示数列$-3,-1,1,3,5,7,\ldots$; 图5.2(2)表示数列
$\frac{1}{2},\frac{2}{3},\frac{3}{4},\frac{4}{5},\frac{5}{6},\ldots$.

用图象法表示数列时,图象是由直角坐标系中的一些孤
立点组成,其中每一个点$(n,a_n)$
的横坐标$n$表示项数,纵坐标$a_n$表示该项的值,用图象表示数列时,其两个坐标轴上的单位可以不同。

\subsection{解析法表示数列}
如果数列的第$n$项$a_n$能用项数$n$的解析式表示为:
$a_n=f(n)\; (n\in\N)$。这种表示方法称为解析法,这个解析式叫做数列的\textbf{通项公式}。

图5.2(1)所表示的数列的通项公式
$a_n=2n-5$; 图
5.2(2)所表示的数列的通项公式为$a_n=\frac{n}{n+1}$.

不是所有的数列都能用解析法来表示。例如$\sqrt{2}$
的精
确到$1,0.1,0.01,0.001,\ldots$的不足近似值组成的数列$1,
1.4,1.41,1.414,\ldots$就没有通项公式。

\subsection{用递推式表示数列}
有时数列可以用它的前几项的值(称初始条件或初始
值),和数列中相邻若干项间的关系式(称递推式)给出。例
如一个数列$\{a_n\}$
中,
$a_1=1$,
$a_{n+1}=2a_{n}+1\; (n\in\N)$
,这个数列
就是$1,3,7,15,31,63,\ldots$; 又如数列$2,5,8,11,14,
\ldots$, 可以表示为:
$a_1=2$, $a_{n+1}=a_n+3$.

\begin{ex}
    按照下列条件,写出数列的前五项。
\begin{enumerate}
    \item 数列的通项公式是:
\begin{multicols}{2}
\begin{enumerate}[(1)]
    \item $a_n=-3n+1$
    \item $a_n=\left(\frac{1}{2}\right)^n$
    \item $a_n=\frac{2n-1}{n+1}$
    \item $a_n=n^2+2$
\end{enumerate}    
\end{multicols}
\item 数列$\{a_n\}$中,已知
$a_1=2$,且$(n\in\N)$.
\item 数列$\{a_n\}$的图象如图5.3.
\end{enumerate}
\end{ex}

\begin{center}
\begin{tikzpicture}[>=stealth, scale=.6]
 \draw[->](-1,0)--(7,0)node[below]{$n$};
\draw[->](0,-3)--(0,4)node[right]{$a_n$};
\foreach \x in {1,2,3,4,5,6}
{
   \draw(\x,0)node[below]{$\x$}--(\x,.1);
}
\foreach \x in {1,2,3,-1,-2}
{
   \draw(0,\x)node[left]{$\x$}--(.1,\x);
}
\draw[dashed](0,-2)--(3,-2)--(3,0);
\draw[dashed](0,-1)--(5,-1)--(5,0);
\draw[dashed](0,.5)node[left]{$\tfrac{1}{2}$}--(6,.5)--(6,0);
\draw[dashed](0,3)--(2,3)--(2,0);
\draw[dashed](0,1)--(4,1)--(4,0);
\node[below left]{$O$};
\end{tikzpicture}
\captionof{figure}{ }
\end{center}

\section{简单数列的通项公式的求法}
用语言叙述或用列表法给出的一些数列的前几项,我们
可以利用归纳的办法,写出它们的通项公式。

\begin{example}
    写出下列各数列的通项公式。
\begin{enumerate}[(1)]
\item 自然数的倒数组成的数列;
\item 每项的值都比项数的立方少2的数列;
\item $-1$的自然数次幂组成的数列。
\end{enumerate}
\end{example}

\begin{solution}
\begin{multicols}{3}
  \begin{enumerate}[(1)]
    \item $a_n=\frac{1}{n}$
    \item $a_n=n^3-2$
    \item $a_n=(-1)^n$
\end{enumerate}  
\end{multicols}
\end{solution}

\begin{example}
    写出下列各数列的一个通项公式,使其前六项分
别是:
\begin{enumerate}[(1)]
\item  $\frac 12,\; \frac 34, \;\frac 78,\; \frac {15}{16} , \; \frac {31}{32} , \; \frac {63}{64}$
\item $2,\; -6,\; 18,\; -54,\; 162,\; -486$
\item $-1,\; \frac{3}{2},\; -\frac{1}{3},\; \frac{3}{4},\; -\frac{1}{5},\; \frac{3}{6}$
\item $9,\; 99,\; 999,\; 9999,\; 99999,\; 999999$
\end{enumerate}

\end{example}

\begin{solution}
\begin{enumerate}[(1)]
    \item 分母为$2^n$,分子均比分母少1, 所以数列的通
    项公式为$a_n=\frac{2^n-1}{2^n}$
    \item 分析数列的前6项发现,第1项是2, 第2项是
    2的$-3$倍,第3项又是第二项的$-3$倍,以此类推,可归纳
    出其通项公式为
    $a_n=2\x(-3)^{n-1}$
    \item 数列各项的分母依次为$1,2,\ldots$, 恰与项数相同;
    各项的分子为$1,3,1,3,\ldots$, 可以说是第1项比2少1,
    第2项比2多1, 以此类推;又数列相邻两项值的符号相
    反,且第1项为负值,因此可用符号$(-1)^n$来表示,由此归
    纳可得通项公式为
    $a_n=(-1)^{n}\frac{2+(-1)^n}{n}$.
    \item 数列第1项可改写为$10-1$, 第2项为
    $10^2-1$,……
    故通项公式为$a_n=10^n-1$.
\end{enumerate}
\end{solution}

\begin{rmk}
\begin{enumerate}[(1)]
    \item  分析数列中各项的值与项数间的关系,从而
    归纳出通项公式,是求数列通项公式的最基本的方法。
    \item 给出数列的前几项,求这个数列的通项公式时,
    其通项公式并不是唯一的。事实上,满足一个数列的前若干
    项的通项公式可以有无穷多个。例如,上例中的(1)的通
    项公式可以写成:
\[a_n=\frac{2^n-1}{2^n}+(n-1)(n-2)(n-3)(n-4)(n-5)(n-6)\cdot f(n)\]
(其中$f(n)$
是含$n$的任何一个函数式),其前六项均符合要
求,这是因为公式的后一部分,当$n$依次取$1,2,3,4,5,
6$时,其值都是0.
\end{enumerate}
\end{rmk}

\begin{ex}
\begin{enumerate}
    \item 写出数列的一个通项公式,使得数列的前五项分别是
\begin{enumerate}[(1)]
    \item $15,\; 25,\; 35,\; 45,\; 55$
    \item $1,\; -\frac{1}{2},\; \frac{1}{4},\; -\frac{1}{8},\; \frac{1}{16}$
    \item $1,\; 3,\; 5,\; 7,\; 9$
    \item $1-\frac{1}{2},\; \frac{1}{2}-\frac{1}{3},\; \frac{1}{3}-\frac{1}{4},\; \frac{1}{4}-\frac{1}{5},\; \frac{1}{5}-\frac{1}{6}$
    \item $0,\; -5,\; 8,\; -17,\; 24$
    \item $3,\; 33,\; 333,\; 3333,\; 33333$
\end{enumerate}
\item 观察下列数列的特点,用适当的数填空并对每一个数列各
写出一个通项公式:
\begin{enumerate}[(1)]
    \item $2,\; 4,\; (\quad ),\; 8,\; 10,\; (\quad ),\;14$
    \item $2,\;4,\;(\quad),\;16,\;32,\;(\quad),\;128,\;(\quad)$
    \item $(\quad),\;4,\;9,\;16,\;(\quad),\;(\quad),\;49,\;64,\;(\quad)$
    \item $(\quad),\;4,\;3,\;2,\;1,\;(\quad ),\; -1,\; (\quad ),\;(\quad )$
    \item $1,\;\sqrt{2},\;(\quad),\; 2,\;\sqrt{5},\; (\quad),\;(\quad),\; 2\sqrt{2}$
    \item $\frac{1}{6},\; \frac{1}{12},\; \frac{1}{20},\;(\quad),\;\frac{1}{42},\;(\quad),\;\frac{1}{72}$
\end{enumerate}
\item 已知数列的通项公式,试判断后面所给的数是否是数列
中的项,如果是,是第几项。
\begin{enumerate}[(1)]
    \item $a_n=n(n+2),\qquad$  (a) 142,\quad (b) 175
    \item $a_n=\frac{2n-1}{3n+2},\qquad $ (a) $\frac{3}{5}$,\quad (b) $\frac{17}{26}$
    \item $a_n=\frac{n^2+3n}{n+1},\qquad $ (a) $\frac{182}{13}$,\quad (b) $\frac{350}{13}$
\end{enumerate}
\end{enumerate}
\end{ex}

\section{数列的分类}

\subsection*{按照数列的项数,可将数列分成有穷数列和无穷数
列}

有穷数列:如果在某一项的后面不再有任何项,这个数
列叫做\textbf{有穷数列}。例如,在前一百个自然数中,一切质数组
成的数列$2,3,5,7,11,13,\ldots,97$是有穷数列。

无穷数列:如果在任何一项的后面都有跟随着的项,这
个数列叫做\textbf{无穷数列}。例如,自然数中所有奇数组成的数列,是无穷数列。

当用列表法表示数列时,写出末项的,表示有穷数列,
如$1,\frac{1}{2},\frac{1}{4},\frac{1}{8},\ldots,\frac{1}{2^{n-1}}$
表示有穷数列;写不出末
项的,则表示无穷数列,如$2,4,6,8,\ldots,2n,\ldots$表示
无穷数列。

\subsection*{按照项与项之间的大小关系,可将数列分成常数列,
递增数列,递减数列和摆动数列}

\begin{enumerate}[(1)]
\item 常数列:数列中各项的值都相等的数列叫\textbf{常数
列}。如$3,3,3,\ldots,3,\ldots$就是常数列;
\item 递增数列:一个数列从第2项起,每一项都大于
它的前面的一项,即
$a_{n+1}>a_n\; (n\in\N)$
,这样的数列叫做\textbf{递增
数列}。例如,自然数的平方组成的数列$1,4,9,16,\ldots,
n^2,\ldots$就是递增数列。
\item 递减数列:一个数列从第2项起,每一项都小于
它的前面的一项,即
$a_{n+1}<a_n\; (n\in\N)$
,这样的数列叫做\textbf{递减数列}。例如,自然数的倒数组成的数列$1,\frac{1}{2},\frac{1}{3},\frac{1}{4},\ldots,\frac{1}{n},\ldots$就是递减数列。

递增数列与递减数列,统称\textbf{单调数列}。

\item 摆动数列:一个数列,从第2项起,有些项大于
它的前一项,有些项又小于它的前一项,这样的数列叫做\textbf{摆
动数列}.例如数列$1,-2,4,-8,16,\ldots, (-2)^{n-1},\ldots$
是摆动数列。
\end{enumerate}

\begin{example}
    判断下列数列是递增数列、递减数列,还是摆动数
列?数列的通项公式如下:
\begin{multicols}{2}
\begin{enumerate}[(1)]
\item $a_n=2n+5$
\item $a_n=-3n+1$
\item $a_n=2\x\left(-\frac{1}{3}\right)^n$
\item $a_n=-\frac{1}{2}\x5^{n-1}$
\item $a_n=\frac{2n-3}{n+1}$
\end{enumerate}
\end{multicols}
\end{example}

\begin{solution}
判断数列的增减性要根据递增数列、递减数列的定
义,即计算
$a_{n+1}-a_n$
,并判断其值的正负。但对摆动数列,则
可写出数列的若干项予以判断。
\begin{enumerate}[(1)]
    \item $\because\quad a_{n+1}-a_n=[2(n+1)+5]-(2n+5)=2>0$
    
$\therefore\quad $数列$\{2n+5\}$
是递增数列。

\item $\because\quad a_{n+1}-a_n=[-3(n+1)+1]-(-3n+1)=-3<0$
    
$\therefore\quad $数列$\{-3n+1\}$
是递减数列。
\item 数列的前三项是:$-\frac{2}{3},\; \frac{2}{9},\; -\frac{2}{27}$,显然它是摆动数列。

\item \[\begin{split}
    \because\quad a_{n+1}-a_n&=-\frac{1}{2}\x 5^n-\left[-\frac{1}{2}\x 5^{n-1}\right]\\
    &=-\frac{1}{2}\x 5^{n-1}(5-1)=-2\x 5^{n-1}<0
\end{split}\]

$\therefore\quad $数列$\left\{-\frac{1}{2}\x 5^{n-1}\right\}$
为递减数列。

\item \[\begin{split}
    \because\quad a_{n+1}-a_n&=  \frac{2(n+1)-3}{(n+1)+1}-\frac{2n-3}{n+1}\\
    &=\frac{(2n^2+n-1)-(2n^2+n-6)}{(n+2)(n+1)}=\frac{5}{(n+2)(n+1)}>0
\end{split}\]

$\therefore\quad $数列$\left\{\frac{2n-3}{n+1}\right\}$
是递增数列。
\end{enumerate}
\end{solution}

\begin{rmk}
    这里数列的“增减性”的判断方法和函数
$f(x)$的增减性的判断方法类似。
\end{rmk}

\section{数列的前$n$项和}
数列$\{a_n\}$的前$n$项的和是指
$a_1+a_2+\cdots+a_n$,记作$S_n$,

$S_1$表示前一项之和,所以$S_1=a_1$;

$S_2$表示前两项之和,所以
$S_2=a_1+a_2$;

………………

$S_{n-1}$表示前$n-1$项之和,所以
$S_{n-1}=a_1+a_2+\cdots+a_{n-1}$

$S_n$表示前$n$项之和,所
以$S_n=a_1+a_2+\cdots+a_n$.

因此,只有当
$n\ge 1$时,$S_n$才是有意义的,例如$S_{n-1}$,只有当$n\ge 2$时,才有意义.

由$S_n$的含义可知,当$n\geqslant2$时,$a_n=S_n-S_{n-1}$, 而当$n=1$时,$a_1=S_1$。这就是$S_n$与$a_n$之间的关系:
$$a_n=\begin{cases}
    S_1& n=1\\
    S_n-S_{n-1} &n\geqslant2 
\end{cases}$$

\begin{example}
    已知数列的前$n$项和,求数列的通项公式。
\begin{multicols}{3}
\begin{enumerate}[(1)]
    \item $S_{n}= n^2- 2n+ 2$
    \item $S_{n}= 3^{n}- 2$
    \item $S_{n}=\frac{3^{n}-2^{n}}{2^{n}}$
\end{enumerate}    
\end{multicols}
\end{example}

\begin{solution}
\begin{enumerate}[(1)]
    \item $n= 1$时, $a_{1}= S_{1}= 1$
    
    $n\geqslant2$时,$a_{n}=S_{n}-S_{n-1}=(n^2-2n+2)-[(n-1)^2-2(n-1)+2]=2n-3$

又$n=1$时,$2n-3=-1\ne a_1$,

$\therefore\quad $通项公式为$a_n=\begin{cases}
    1,&n=1\\
    2n-3,&n\ge 2
\end{cases}$

\item $n= 1$时, $a_{1}= S_{1}= 1$
    
$n\geqslant2$时,$a_{n}=S_{n}-S_{n-1}=(3^n-2)-(3^{n-1}-2)=3^{n}-3^{n-1}=2\cdot 3^{n-1}$

又$n=1$时,$2\cdot 3^{n-1}=2\ne a_1$,

$\therefore\quad $通项公式为$a_n=\begin{cases}
1,&n=1\\
2\cdot 3^{n-1},&n\ge 2
\end{cases}$

\item $n= 1$时, $a_{1}= S_{1}= \frac{3-2}{2}=\frac{1}{2}$
    
$n\geqslant2$时,$a_{n}=S_{n}-S_{n-1}=\frac{1}{2}\cdot\left(\frac{3}{2}\right)^{n-1}$

又$n=1$时,$\frac{1}{2}\cdot\left(\frac{3}{2}\right)^{n-1}=\frac{1}{2}= a_1$,

$\therefore\quad $通项公式为$a_n=\frac{1}{2}\cdot\left(\frac{3}{2}\right)^{n-1}\quad (n\in\N)$
\end{enumerate}
\end{solution}

\begin{rmk}
\begin{enumerate}[(1)]
    \item $a_n=\begin{cases}
        S_1,&n=1\\
        S_n-S_{n-1},&n\ge 2
    \end{cases}$相当于分
    段函数的表达式。
    \item 若当由
    $ S_n-S_{n-1}\; (n\ge 2)$
    得到的$a_n$的解析式中,把
    $n=1$代入后的值恰好等于$a_1$时,要把通项公式写成统一的表
    达式,如本例中的(3).
\end{enumerate}
\end{rmk}


\begin{ex}
    已知数列$\{a_n\}$
的前$n$项的和$S_n$, 求它的通项公式:
\begin{enumerate}
    \item $S_n=a\cdot n^2+b\cdot n\quad \text{($a$、$b$为已知常数)}$
    \item   $S_{n}=a\cdot n^{2}+b\cdot n+c\quad  \text{($a$、$b$、$c$为已知常数)}$
    \item  $S_n=n^3+n-1$
    \item  $S_{n}=2\cdot 3^{n}-1$
    \item  $S_{n}=a\cdot b^{n-1}\quad (a,b\neq0)$
    \item  $S_{n}=a\cdot n$
\end{enumerate}
\end{ex}

\section*{习题一}
\begin{center}
    \bfseries A
\end{center}

\begin{enumerate}
    \item 写出下列各数列的一个通项公式,使它的前几项分别
    是:
\begin{multicols}{2}
\begin{enumerate}[(1)]
    \item $1,\; -2,\; 3,\; -4,\; 5$
    \item $0,\; 3,\; 8,\; 15,\; 24$
    \item $2,\; 7,\; 28,\; 63,\; 126$
    \item $1,\; -\frac{1}{3},\; \frac{1}{9},\; -\frac{1}{27},\; \frac{1}{81}$
    \item $\frac{1}{3},\; \frac{1}{8},\; \frac{1}{15},\; \frac{1}{24},\; \frac{1}{35}$
    \item $5,\; 55,\; 555,\; 5555$
    \item $1,\; 0,\; -1,\; 0,\; 1,\; 0,\; -1,\; 0$
    \item $2,\; 0,\; 2,\; 0,\; 2,\; 0$
\end{enumerate}    
\end{multicols}

\item 按照所给条件分别写出数列的前5项:
\begin{enumerate}[(1)]
    \item 通项公式为$a_n=\frac{(-1)^{n+1}}{n}$
    \item 通项公式为$a_n=-2^{n-1}+3$
    \item 若$a_1=1$, $a_{n+1}=1+\frac{1}{a_n}\; (n\in\N)$
    \item 若$a_1=5$, $a_{n+1}={a_n}+3\; (n\in\N)$
    \item 若$a_1=2$, $a_{n+1}=2{a_n}\; (n\in\N)$
    \item 若$a_1=3$, $a_2=9$, $a_{n+2}=a_{n+1}-{a_n}\; (n\in\N)$
    \item 若$a_1=1$, $a_{n+1}=a_n+\frac{1}{a_n}\; (n\in\N)$
    \item 若$a_1=1$, $a_{n+1}=\frac{2a_n}{a_n+2}\; (n\in\N)$
\end{enumerate}

\item 由$\{a_n\}$的前$n$项和$S_n$,求它的通项公式
\begin{multicols}{3}
\begin{enumerate}[(1)]
    \item $S_n=n^2+2n+1$
    \item $S_n=2\cdot 3^n-1$
    \item $S_n=(-1)^{n+1}\cdot n$
\end{enumerate}
\end{multicols}

\item 已知$\{a_n\}$的前$n$项和为$S_n=2n^3-3n$, 
则$a_5+a_6=\blank$; $a_8+a_9+a_{10}=\blank$.
\end{enumerate}

\begin{center}
    \bfseries B
\end{center}

\begin{enumerate}\setcounter{enumi}{4}
    \item 判断无穷数列是递增数列、递减数列,还是摆动数列?并
    给以证明。它们的通项公式分别是:
\begin{multicols}{2}
\begin{enumerate}[(1)]
    \item $a_n=3n-5$
    \item $a_n=-2n+5$
    \item $a_n=\frac{4}{n}$
    \item $a_n=\left(\frac{1}{3}\right)^{n+1}$
    \item $a_n=\lg n$
    \item $a_n=\sin\frac{n\pi}{2}$
    \item $a_n=n^3$
    \item $a_n=\frac{3n-2}{n+1}$
    \item $a_n=2-\frac{3}{n+1}$
    \item $a_n=n^2+2n-3$
    \item $a_n=a\cdot \left(\frac{1}{2}\right)^n$ (其中$a\ne 0$)
\end{enumerate}
\end{multicols}
\end{enumerate}

\section*{二、等差数列}

\section{等差数列的有关概念}


观察下面的数列:
\begin{align}
&1,\; 4,\; 7,\; 10,\; 13,\; \ldots \tag{1}\\
&5,\; 0,\; -5,\; -10,\; -15,\; \ldots \tag{2}\\
&2,\; 2,\; 2,\; 2,\; 2,\; \ldots \tag{3}
\end{align}

它们分别具有下述的特点:

(1)中,从第2项起,每一项与它的前一项的差都是3;

(2)中,从第2项起,每一项与它的前一项之差都是$-5$;

(3)中,从第2项起,每一项与它的前一项之差都是0.

它们的共同特点是:从第2项起,每一项与它的前一项
之差都等于同一个常数,通常把这个常数记作$d$, 即
\[a_n-a_{n-1}=d\quad \text{($n\ge 2$, \; $d$是常数)}\]
这类数列叫做\textbf{等差数列},常数$d$叫做等差数列的\textbf{公差}。上述三个等差数列的公差分别是$3,-5$和0, 数列(3)说明常数
数列一定是等差数列。

如果有三个数$x,A,y$组成等差数列,那么$A$叫做$x$和$y$
的\textbf{等差中项}。

如果$x,A,y$
成等差数列,由等差数列的定义可知$A-x=y-A$,所以
\[A=\frac{x+y}{2}\]

显然,任何两个数,都有唯一确定的等差中项。

容易看出,在一个无穷的等差数列中,从第2项起,每
一项都是它的前一项与后一项的等差中项;反之,在一个数
列中,如果从第2项起,每一项都是它的前一项与后一项的
等差中项,那么这个数列一定是等差数列。

\begin{example}
已知数列$\{a_n\}$的通项公式为
$a_n=3n-5$.
\begin{enumerate}[(1)]
\item 求证:数列$\{a_n\}$是等差数列,并求其公差;
\item 求出数列的首项及第100项;
\item 判断100和110是不是该数列中的项,如果是,是第
几项?
\end{enumerate}
\end{example}

\begin{solution}
\begin{enumerate}[(1)]
    \item 由于
    $a_n-a_{n-1}=3n-5-[3(n-1)-5]=3$(常数),所以数列
    $\{a_n\}$
    是等差数列,且公差是3.
\item $a_1=3\x1-5=-2$. $a_{100}=3\x100-5=295$.
\item 设
    $3n-5=100$,解之得
    $n=35$,所以100是数列的第
    35项。

    设
    $3n-5=110$,解之得
    $n=\frac{115}{3}$不是正整数,所以110不是数列$\{a_n\}$中的项。
\end{enumerate}
\end{solution}

\begin{example}
    求证:通项公式是
$a_n=an+b$($a$、$b$是常数)的数列$\{a_n\}$
是等差数列,且公差
$d=a$.
\end{example}

\begin{proof}
$\because\quad a_n-a_{n-1}=an+b-[a(n-1)+b]=a$,

$\therefore\quad \{a_n\}$是等差数列,且以
$a$为公差。
\end{proof}

例5.6说明当$a\ne 0$
,即通项公式是$n$的一次式时,该数列
必为等差数列,且其一次项系数恰好是此等差数列的公差。

显然,当$a=0$
时,数列是常数列,也是等差数列。

\section{等差数列的通项公式}
上面的例5.6告诉我们,一个数列的通项公式是$n$的一次
式时,该数列一定是等差数列,并且其一次项系数恰好是该
等差数列的公差。那么一个等差数列若公差$d\ne 0$时,其通项
公式是否也一定是$n$的一次式呢?

为此,我们做如下的探讨:

由等差数列的定义,有
\[\begin{split}
  a_2&=a_1+d,\\
a_3&=a_2+d=(a_1+d)+d=a_1+2d,\\
a_4&=a_3+d=(a_1+2d)+d=a_1+3d,\\
\cdots &\cdots \cdots
\end{split}\]

这样,可以归纳得出
\[a_n=a_1+(n-1)d\]
这是由对事物的部分对象的考察,观察其规律,得出的
结论。这个方法就是不完全归纳法。所得结论的正确性,今
后我们可以用数学归纳法给予证明。

这个公式,还可以用下面的方法得到

由等差数列的定义,可知
\[\begin{split}
    a_2-a_1&=d\\
    a_3-a_2&=d\\
    a_4-a_3&=d\\
    \cdots&\cdots\cdots\\
    a_{n-1}-a_{n-2}&=d\\
    a_n-a_{n-1}&=d\\
\end{split}\]

将这$n-1$个式子的等号两边分别相加,得$a_n-a_1=(n-1)d$,
即
\[a_n=a_1+(n-1)d\]
这个方法,通常称为\textbf{迭加法}。

$a_n=a_1+(n-1)d$就是用首项
$a_1$和公差$d$表示的等差数
列的\textbf{通项公式}。

$\because\quad a_n=a_1+(n-1)d=dn+(a_1-d)$

$\therefore\quad$当$d\ne 0$时,$a_n$是$n$的一次式,且一次项的系数是公差$d$.

显然,如果
$a_n=dn+(a_1-d)$
中的
$d=0$,$\{a_n\}$
也是等差数列。

结合5.6中的例5.6, 我们可以得到下述定理:

\begin{thm}
{定理1} 数列$\{a_n\}$是等差数列的充分必要条件是
$a_n=dn+b$, 其中$d$为公差,$b$为常数。
\end{thm}

用图象法表示等差数列时,点$(n,a_n)$
一定在斜率为$d$的
一条直线上。

5.6中的数列(1)可用图5.4来表示,其中点$(1,1)$,
$(2,4)$, $(3,7)$, $(4,10)$, ……都在直线
$y=3x-2$上。其中直线的斜率3, 恰好是该数列的公差。

\begin{figure}[htp]
    \centering
\begin{tikzpicture}[>=stealth, yscale=.5]
\draw[->](-1,0)--(4,0)node[right]{$n$};
\draw[->](0,-1)--(0,9)node[right]{$a_n$};
\node[below left]{$O$};
\foreach \x in {1,2,3}
{
    \draw(\x,0)node[below]{\x}--(\x,.1);
}
\foreach \x in{2,4,6,8}
{
    \draw(0,\x)node[left]{\x}--(.1,\x);
}

\draw[domain=.25:3.5, smooth, dashed]plot(\x, 3*\x-2);
\tkzDefPoints{1/1/A, 2/4/B, 3/7/C}
\tkzDrawPoints(A,B,C)
\node at (A)[right]{$(1,1)$};
\node at (B)[right]{$(2,4)$};
\node at (C)[right]{$(3,7)$};

\end{tikzpicture}
    \caption{}
\end{figure}


与确定函数式$y=kx+b$一样,要确定一个等差数列的通
项公式,需要知道两个独立条
件,例如数列中的某一项和公差,或数列中的任何两项,等等。

\begin{example}
   已知等差数列的公差为$d$, 第$m$项是
$a_m$,试求其第$n$项$a_n$. 
\end{example}

\begin{solution}
    由等差数列的通项公式可知
$a_n=a_1+(n-1)d$, 
即
\[a_m=a_1+(m-1)d\]

两式相减,得
$a_n-a_m=(n-m)d$,
即
\[a_n=a_m+(n-m)d\]
\end{solution}

此例说明:等差数列的通项公式,可以用数列中的任何
一项及公差来表示。特别当$m=1$时,就是
\[a_n=a_1+(n-1)d\]

\begin{example}
    等差数列的第5项是20,公差是3,求它的第50
项。
\end{example}

\begin{solution}
 \textbf{解法一:} $\because\quad a_5=a_1+4d$,

$\therefore\quad a_1+4\x3=20$,解得$a_1=8$

$\therefore\quad a_{50}=a_1+49d=8+49\x3=155$.

\textbf{解法二:} 因为
$d=3$,由定理1可设数列的通项公式为
\[a_n=3n+b\]
$\therefore\quad a_5=3\x5+b=20$

$\therefore\quad b=5$,于是
\[a_{50}=3\x50+5=155\]

\textbf{解法三:} 由例5.7的结论,有
\[a_{50}=a_5+(50-5)d\]
于是
\[a_{50}=20+45\x3=155\]
\end{solution}

\begin{example}
    梯子共有12级,从上面数第四级的宽是54cm, 最
低一级宽110cm, 已知各级的宽度成等差数列,试计算梯子
各级的宽。
\end{example}

\begin{solution}
    用$\{a_n\}$表示题中的等差数列,其公差为
$d$, 设$a_n=dn+b$,由已知
$a_4=54$,$a_{12}=110$,所以
\[\begin{cases}
    54=d\cdot 4+b\\
    110=d\cdot 12+b
\end{cases}\]
解关于$d$、$b$的方程组,得$d=7$,$b=26$
,于是
\[a_n=7n+26\]
\[\begin{split}
\therefore\quad a_1&=7\x 1+26=33\\
a_2&=7\x2+26=40\\
a_3&=7\x3+26=47\\
\cdots & \cdots\cdots
\end{split}\]

答:梯子自上而下各级的宽度依次为33, 40, 47, 54, 
61, 68, 75, 82, 89, 96, 103和110(cm).
\end{solution}

\begin{example}
    在$-1$和9之间插入三个数,使这五个数组成等差
数列,求插入的三个数。
\end{example}

\begin{solution}
    设数列为$\{a_n\}$,由题意知
$a_1=-1$,$a_5=9$。

由等差数列通项公式,有
\[d=\frac{a_n-a_1}{n-1}=\frac{9-(-1)}{5-1}=2.5\]

$\therefore\quad a_2=1.5,\quad a_3=4,\quad a_4=6.5$

答:插入的三个数分别为1.5, 4和6.5.

\textbf{另解:}因为等差数列$\{a_n\}$
共有5项,所以$a_3$是$a_1$和$a_5$的
等差中项,即
\[a_3=\frac{-1+9}{2}=4\]

又$a_2$是$a_1$和$a_3$的等差中项,
$a_4$是$a_3$和$a_5$的等差中项,故
\[a_3=\frac{-1+4}{2}=1.5,\quad a_4=\frac{4+9}{2}=6.5\]

答:插入的三个数分别为1.5, 4和6.5.
\end{solution}

\begin{example}
已知$a^2,\;b^2,\;c^2$成等差数列,且$b+c,\;c+a,\;a+b$均不为零,

求证:$\frac{1}{b+c},\;\frac{1}{c+a},\; \frac{1}{a+b}$也成等差数列.
\end{example}

\begin{solution}
因为$a^2,b^2,c^2$成等差数列,所以
$2b^2=a^2+c^2$,于是
\[\begin{split}
\frac{1}{b+c}+\frac{1}{a+b}&=\frac{a+b+b+c}{ab+b^2+ac+bc}=\frac{a+2b+c}{ab+\frac{a^2+c^2}{2}+ac+bc}\\
&=\frac{2(a+2b+c)}{2ab+a^2+c^2+2ac+2bc}\\
&=\frac{2(a+2b+c)}{(a+c)(a+c+2b)}=\frac{2}{a+c}
\end{split}\]

$\therefore\quad \frac{1}{b+c},\;\frac{1}{c+a},\; \frac{1}{a+b}$也成等差数列.
\end{solution}



\begin{example}
    已知一个直角形的三条边成等差数列,求证它们的
比是3:4:5.
\end{example}


\begin{solution}
    将成等差数列的三边的长从小到大排列,它们可以
表示为
$a-d$, $a$, $a+d$,
其中
$a-d>0$, $d>0$, $d$
就是它们的
公差。由于它们是直角三角形的三边的长,根据勾股定理,得
到
\[(a-d)^2+a^2=(a+d)^2\]
解之得$a=4d$, 
从而,三角形三边长依次为$3d$, $4d$和$5d$,
因此,三条边之比为$3:4:5$.
\end{solution}

\begin{rmk}
此例中设成等差数列的三数为$x-d$
,$x$和$x+d$($d$为公差)。可简化运算过程。若四数成等差数列时,可设为
$x-3d$, $x-d$, $x+d$和$x+3d$
,这样可用两个量$x,d$表示四
个数。但需注意这里$d$并非其公差,公差是$2d$.
\end{rmk}

\begin{example}
    设等差数列$\{a_n\}$
的公差为$d$, 求证:
\begin{enumerate}[(1)]
\item 当$d>0$时,数列是递增数列;
\item 当$d<0$时,数列是递减数列。
\end{enumerate}
\end{example}

\begin{solution}
因为$\{an\}$是等差数列,当
$d\ne 0$时它的通项公式是
$n$的一次式,且其一次项系数恰是数列的公差$d$, 即
$a_n=dn+c$.

由一次函数的性质可知:
\begin{itemize}
    \item 当$d>0$时,函数
$y=dx+c$
是增函数,故数列
$\{d_n+c\}$也为递增数列,
\item 当$d<0$时,函数
$y=dx+c$是减函数,故数列
$\{dn+c\}$
为递减数列。
\end{itemize}
\end{solution}


\begin{example}
    一个等差数列的首项是50, 公差是$-0.6$, 从第几
项开始是负数。
\end{example}

\begin{solution}    
该数列的通项公式是
$a_n=50-(n-1)\x0.6$,令
$50-(n-1)\x0.6=0$,
解之得:$$n=84.3$$

由于数列是首项为正的递减数列,所以从第85项开始是
负数。
\end{solution}





\begin{example}
设$\{a_n\}$是等差数列,$k,\ell,m,n\in\N$,若
$k+\ell=m+n$,则
$a_k+a_{\ell}=a_m+a_n$
\end{example}

\begin{proof}
    设数列的公差为$d$, 则有
\[\begin{split}
     a_k&=a_1+(k-1)d,\\
a_{\ell}&=a_1+(\ell-1)d.
\end{split}\]

$\therefore\quad a_k+a_{\ell}=2a_1+(k+\ell-2)d$

$\because\quad a_m=a_{\ell}+(m-\ell)d,\quad a_n=a_1+(n-1)d$

$\therefore\quad a_m+a_n=2a_1+(m+n-2)d$

$\because\quad k+\ell=m+n$

$\therefore\quad a_k+a_{\ell}=a_m+a_n$
\end{proof}

\begin{rmk}
    此例说明,在等差数列中,项数之和相等时,对
应的两项之和也相等,例如等差数列$\{a_n\}$
中,有
$a_5+a_{13}=a_2+a_{16}=a_8+a_{10}=\cdots$,这是等差数列的一个重要性质。

特别地
$a_1+a_n=a_2+a_{n-1}=a_3+a_{n-2}=\cdots$.
\end{rmk}

\begin{ex}
\begin{enumerate}
    \item 解下列各题:
\begin{enumerate}[(1)]
\item 求等差数列$3,\; 7,\; 11,\; \ldots$的第4,7,10项;
\item 求等差数列$10,\; 8,\; 6,\; \ldots$的第20项;
\item 求等差数列$2,\; 9,\; 16,\; \cdots$的第$n$项;
\item 求等差数列$0,\; -3\frac{1}{2},\; -7,\; \cdots$的第$n+1$项。
\end{enumerate}

\item 设等差数列$\{a_n\}$的公差为$d$, 求证:
\begin{enumerate}[(1)]
\item $d= \frac {a_{n}- a_{1}}{n- 1},\quad \left ( n\neq 1\right ) $     
\item $d= \frac {a_{n}- a_{m}}{n- m},\quad ( m\neq n) $
\end{enumerate}

\item 设$\{a_n\}$为等差数列,公差为$d$.
\begin{enumerate}[(1)]
\item 已知 $d=-\frac{1}{3}$, $a_{7}=8$, 求$a_1$;
\item 已知$a_{1}=12$, $a_{6}=27$, 求$d$;
\item  已知$a_1=3$, $a_n=21$, $d=2$, 求$n$; 
\item 已知$a_4=10$, $a_7=19$, 求$a_1$和$d$;
\item 已知$a_1=1.7$, $d=0.3$, 求证46.7是该数列中的项,并说明是第几项。
\end{enumerate}
\item 求证等差数列$\{a_n\}$中所有的奇数项组成的数列也是等 差数列,并求这个数列的公差。
\item 求下列各题中两个数的等差中项:
\begin{multicols}{2}
\begin{enumerate}[(1)]
    \item 647与895
    \item $-180$与360
    \item $\frac{\sqrt{3}+\sqrt{2}}{\sqrt{3}-\sqrt{2}}$与$\frac{\sqrt{3}-\sqrt{2}}{\sqrt{3}+\sqrt{2}}$
    \item $(a+b)^2$与$(a-b)^2$
\end{enumerate}    
\end{multicols}
\end{enumerate}
\end{ex}

\section{等差数列前$n$项的和}
我们观察下面的例子:

图5.5(1)表示有规则堆放着的钢管。

\begin{figure}[htp]
    \centering
\begin{tikzpicture}[scale=.7]
\begin{scope}
 \foreach \x in {1,2,3,...,13}
{
    \draw[very thick](\x/2,0)circle(.23);
}  
\foreach \x in {1,2,3,...,12}
{
    \draw[very thick](\x/2+.25,.45)circle(.23);
}  
\foreach \x in {1,2,3,...,11}
{
    \draw[very thick](\x/2+.5,.9)circle(.23);
}  
\foreach \x in {1,2,3,...,10}
{
    \draw[very thick](\x/2+.75,.45*3)circle(.23);
}  
\foreach \x in {1,2,3,...,9}
{
    \draw[very thick](\x/2+1,.9*2)circle(.23);
}  
\foreach \x in {1,2,3,...,8}
{
    \draw[very thick](\x/2+1.25,.45*5)circle(.23);
}  
\foreach \x in {1,2,3,...,7}
{
    \draw[very thick](\x/2+1.5,.9*3)circle(.23);
}  
\node at (3.5,-1){(1)};
\end{scope}

\begin{scope}[xshift=8cm]    
\node at (3.5,-1){(2)};

 \foreach \x in {1,2,3,...,13}
{
    \draw[very thick](\x/2,-0+3)circle(.23);
}  
\foreach \x in {1,2,3,...,12}
{
    \draw[very thick](\x/2+.25,-.45+3)circle(.23);
}  
\foreach \x in {1,2,3,...,11}
{
    \draw[very thick](\x/2+.5,3-.9)circle(.23);
}  
\foreach \x in {1,2,3,...,10}
{
    \draw[very thick](\x/2+.75,3-.45*3)circle(.23);
}  
\foreach \x in {1,2,3,...,9}
{
    \draw[very thick](\x/2+1,3-.9*2)circle(.23);
}  
\foreach \x in {1,2,3,...,8}
{
    \draw[very thick](\x/2+1.25,3-.45*5)circle(.23);
}  
\foreach \x in {1,2,3,...,7}
{
    \draw[very thick](\x/2+1.5,3-.9*3)circle(.23);
}  

\end{scope}
\end{tikzpicture}
    \caption{}
\end{figure}


显然,自上而下各层的钢管数排成一个等差数列:7,8、
9、10、11、12、13.

为了求出钢管的总数,我们可以设想如图5.5(2)那样,
在这堆钢管旁边倒放着同样的一堆钢管,把两堆合起来,每
层钢管数都相等,即
\[7+13=8+12=\cdots=13+7\]

由于共有七层,因此两堆钢管总数就是:$(7+13)\x7$,
那么,所求钢管总数是
$$\frac{7+13}{2}\x7=70$$
    
一般地,设有等差数列$a_1,a_2,a_3,\ldots,a_n,\ldots$,
它的前$n$项的和记作$S_n$, 即
\[S_n=a_1+a_2+\cdots+a_n\]
根据等差数列的通项公式,上式可以写成
\[S_n=a_1+(a_1+d)+(a_1+2d)+\cdots+[a_n+(n-1)d]\]

再把项的次序反过来,$S_n$又可以写成
\[S_n=a_n+(a_n-d)+(a_n-2d)+\cdots+[a_n-(n-1)d]\]

将这两个式子相加,得
\[\begin{split}
    2S_n&=\overbrace{(a_1+a_n)+(a_1+a_n)+\cdots +(a_1+a_n)}^{n\text{个}} \\
    &=n(a_1+a_n)
\end{split}\]
由此得到等差数列$\{a_n\}$的\textbf{前$n$项的和的公式}
\begin{equation}
    S_n=\frac{n(a_1+a_n)}{2}\tag{1}
\end{equation}

若将$a_n=a_1+(n-1)d$
代入(1)式,又可得出
\begin{equation}
    S_n=na_1+\frac{n(n-1)}{2}d \tag{2}
\end{equation}

从上述求$S_n$的分析过程可见,在一个等差数列的前$n$项
中,与两端(即$a_1$和$a_n$)“等距离”的两项之和都等于首末两
项之和$a_1+a_n$。这是等差数列所具有的一个重要性质。

公式(1)和公式(2)中,分别含有四个量:$a_1,\; a_n,\;  n,\; S_n$和$a_n,\; d,\; S_n,\; n$。如果已知其中三个量,那么每个公式都可看作以其第四个量为未知数的方程,从而第四个量都可
以求出。

如果再加上前一节学的等差数列的通项公 式$a_n=a_1+$ $(n-1)d$,那么在等差数列中,常涉及到以下五个量:$a_1,\; d,\; n,\; a_n$和$S_n$, 并且它们由一组公式:通项公式 和 前$n$项和公式联系着,因此若已知其中的任何三个量,即可得到以其余两个量为未知数的方程组,从而可以求出其余的两个量。


\begin{example}
    求自然数集合中,前100个自然数的和。
\end{example}

\begin{solution}
前100个自然数,组成首项是1,公差是 1 的等差数
列,共有100项,且第100项是100,所以
$$S_{100}=\frac{1+100}2\times100=5050$$

(这正是著名的德国数学家高斯(1777—1855)在他10
岁时,很快地算出$1+2+\cdots+100$的结果时所用的解法)。 
\end{solution}

\noindent
\begin{minipage}{.55\textwidth}
    \CTEXindent
    \begin{example}
图5.6表示一个堆放
铅笔的V形架,它的最下面一
层放一支铅笔,在上每一层都
比它下面一层多放一支,最上
面一层放120支,问这个V形架
上共放着多少支铅笔?

答案是7260支铅笔,请读者完成计算过程。
\end{example}
\end{minipage}\hfill
\begin{minipage}{.4\textwidth}
\centering
\begin{tikzpicture}[scale=.5]
 \foreach \x in {1,2,3,...,6}
{
    \draw[very thick](\x/2-.5-.25*5,.46+.45*5)circle(.23);
}  
\foreach \x in {1,2,3,...,5}
{
    \draw[very thick](\x/2-.5-.25*4,.46+.45*4)circle(.23);
}  
\foreach \x in {1,2,3,4}
{
    \draw[very thick](\x/2-.5-.25*3,.46+.45*3)circle(.23);
}  
\foreach \x in {1,2,3}
{
    \draw[very thick](\x/2-.5-.25*2,.46+.45*2)circle(.23);
}  
\foreach \x in {1,2}
{
    \draw[very thick](\x/2-.5-.25,.46+.45)circle(.23);
}  
\foreach \x in {1}
{
    \draw[very thick](\x/2-.5,.46)circle(.23);
}  
\draw[thick](120:8)--(0,0)--(60:8);

\foreach \x in {7,8,...,14,15}
{
    \draw[dashed](60:\x/2)--(120:\x/2);
}
\foreach \x in {1,2}
{
    \draw[very thick](\x/2-.5-.25*13,.46+.45*13)circle(.23);
    \draw[very thick](-\x/2+.5+.25*13,.46+.45*13)circle(.23);
}  



\end{tikzpicture}
\captionof{figure}{}
\end{minipage}

\begin{example}
    正偶数组成的数列$2,4,6,\ldots$的前多少项的和等于9900?
\end{example}

\begin{analyze}
    已知:$a_1=2$,公差$d=2$, $S_n=9900$,求$n$。因此,我们利用公式$S_n=n\cdot a_1+\frac{n(n-1)}{2}d$求解。
\end{analyze}

\begin{solution}
    由已知,得
\[n\cdot 2+\frac{n(n-1)}{2}=9900\]
    化简得
\[    n^2+n-9900=0\]
    解得$n=99$或$n=-100$(后者不合题意,舍去)。

    答:数列的前99项之和为9900。
\end{solution}



\begin{example}
求集合$M=\{m\mid m=7n,\; n\in\N,\;  \text{且}m<100\}$中元素的个数,并求这些元素之和。
\end{example}

\begin{solution}
$\because\quad     7n<100$
\qquad\qquad
$\therefore\quad n<\frac{100}{7}=14\frac{2}{7}$

由于满足上面不等式的自然数$n$共有14个,故集合$M$里的元素共14个,将它们从小到大列出,就是
\[7,\quad 7\x2,\quad 7\x3,\; \ldots,\; 7\x14\]
即$7,\;14,\;21,\ldots,98$

这个数列是以首项为7,第14项为$a_{14}=98$的等差数列,因此它们的和为
\[S_{14}=\frac{14(7+98)}{2}=735\]

答:集合$M$里共有14个元素,它们的和为735。
\end{solution}

这个例说明:在100以内的自然数中,共有14个数能被4
7整除,且它们的和为735。


\begin{example}
    有一个凸多边形,它的各内角的度数成等差数列,其最小角为$60^{\circ}$,公差为$20^{\circ}$,求它的边数。
\end{example}

\begin{solution}
    设此多边形的边数为$n$,因为它的内角度数成等差数列,且公差是$20^{\circ}$,第一个角是$60^{\circ}$,所以
\[S_n=n\cdot 60+\frac{n(n-1)}{2}\cdot 20\]

另一方面,凸多边形内角和为$(n-2)\cdot 180^{\circ}$,所以
\[n\cdot 60+n(x-1)\cdot 20=(m-2)\cdot 180\]
化简整理,得
\[n^2-13n+36=0\]
解之,得
$n=4$,或$n=9$.

若$n=9$,则$a_9=a_1+(9-1)\cdot d=60+8\x20>180$, 
所以$n=9$不合题意,应舍去。

答:这个多边形的边数是4。
\end{solution}

观察公式,
$S_n=na_1+\frac{n(n-1)}{2}d$是项数$n$的二次型
代数式,我们有下面的定理

\begin{thm}
    {定理2} 设数列$\{a_n\}$的前$n$项和为$S_n$,求证数列$\{a_n\}$是等差数列的充要条件是$S_n=an^2+bn$(其中$a$,$b$为已知常数)。
\end{thm}

\begin{proof}
    先证“必要性”,即证“若$\{a_n\}$是等差数列,则$S_n=a\cdot n^2+bn$.”证明如下:

$\because\quad \{a_n\}$ 是等差数列,

$\therefore\quad S_n=na_1+\frac{n(n-1)}{2}d=\frac{d}{2}n^2+\left(a_1-\frac{d}{2}\right)n$

只需令$\frac{d}{2}=a,\quad a_1-\frac{d}{2}=b$,则有$S_n=an^2+bn$.

再证“充分性”,即证“若数列的前$n$项和$S_n=an^2+bn$,则$\{a_n\}$是等差数列”,证明如下:
$\because\quad S_n=an^2+bn$

$\therefore\quad a_n=\begin{cases}
    S_n-S_{n-1}=2an-a+b, & n\ge 2\\
    S_1=a+b, & n=1
\end{cases}$

$\because\quad n=1$时,$2an-a+b=a+b=S_1$

$\therefore\quad $数列$\{a_n\}$的通项公式为$a_n=2an-a+b\; (n\ge 1)$,由定理1,$\{a_n\}$是等差数列。
\end{proof}


\begin{example}
    设$\{a_n\}$为等差数列,且$S_{10}=100$, $S_{100}=10$,试求前110项之和$S_{110}$.
\end{example}

\begin{analyze}
解法之一是,利用等差数列的求和公式$S_n=a_n+\frac{n(n-1)}{2}d$
,及条件$S_{10}=100$, $S_{100}=10$,列出关于$a_1$与$d$的二元方程组,解出$a_1$与$d$来,然后再求$S_{110}$(请读者自己来完成).

解法之二是,利用定理2,可设$S_n=an^2+bn$,依据条件用待定系数法求出$a$与$b$来,再求$S_{110}$。
\end{analyze}

\begin{solution}
因为$\{a_n\}$为等差数列,由定理2可设$S_n=an^2+bn$,由已知,有
\[\begin{cases}
    a\cdot 10^2+b\cdot 10=100\\
    a\cdot 100^2+b\cdot 100=10
\end{cases}\]
解之,得
$\begin{cases}
    a=-\frac{11}{100}\\[1.5ex]
    b=\frac{111}{10}
\end{cases}$

$\therefore\quad S_{110}=-\frac{11}{100}\x 110^2+\frac{111}{10}\x 110=-110$
\end{solution}



\begin{example}
    已知等差数列中,$a_3+a_{18}=100$,求$S_{20}$.
\end{example}

\begin{solution}
$\because\quad a_3+a_{18}=a_1+a_{20}=100$

$\therefore\quad S_{20}=\frac{a_1+a_{20}}{2}\x20=\frac{100}{2}\x 20=1000$.
\end{solution}


\begin{example}
    已知等差数列$\{a_n\}$的通项公式为$a_n=3n-26$,求其前$n$项和$S_n$的最大值或最小值。
\end{example}

\begin{solution}
 $\because\quad    a_1=3\x 1-26=-23<0,\quad d=3>0$

 $\therefore\quad S_n$有最小值,无最大值。

 设$a_n=3n-26\le 0$,得
\[n\le \frac{26}{3}=8\frac{2}{3}\]
即数列的前8项均为负数,而第9项$a_9=3\x9-26=1>0$,故前8项之和$S_8$是$S_n$的最小值。其最小值为
\[S_8=8\x(-23)+\frac{8\x7}{2}\x3=-100\]
\end{solution}

\begin{example}
    已知等差数列$\{a_n\}$中,$a_n=32$,公差为$-4$,求它的前$n$项和$S_n$的最大值或最小值。
\end{example}

\begin{solution}
    因为$a_1=32$, $d=-4<0$,所以数列的前$n$项和$S_n$有最大值,无最小值。

    设$a_n=32+(n-1)\x(-4)\ge 0$, 求得$n\le 9$, 
即数列的前9项为非负数,而第10项开始为负数,又$a_0=0$, 所以前8项的和与前9项的和相等,即$S_8=S_9$,且它们是$S_n$的最大值,最大值是:
\[S_8=32\x8+\frac{8\x7}{2}\x(-4)=144\]
或
\[S_9=32\x9+\frac{9\x8}{2}\x(-4)=144\]

\textbf{另解:}因为等差数列前$n$项和$S_n$当$d\ne 0$时是$n$的二次式,所以也可以借助于求二次函数的极值的方法,来求$S_n$的最大值或最小值。

如例5.16中,$a_n=3n-26$,所以$a_1=-23$,于是
\[\begin{split}
    S_n=\frac{n(-23+3n-26)}{2}&=\frac{3}{2}n^2-\frac{49}{2}n\\
    &=\frac{3}{2}\left(n-\frac{49}{6}\right)^2-\frac{3}{2}\x \left(\frac{49}{6}\right)^2
\end{split}\]
然而,这里$n\in\N$,所以$n-\frac{49}{6}\ne 0$,只有当$\left|n-\frac{49}{6}\right|$最小时,即$n=8$时,$S_n$取得最小值,此时,$(S_n)_{\min}=\frac{3}{2}\x 8^2-\frac{49}{2}\x 8=-100$

又如例5.17中,有
\[\begin{split}
    S_n=\frac{n(32+32-4n-4)}{2}&=-2n^2+34n\\
    &=-2\left(n-\frac{17}{2}\right)^2+2\x \left(\frac{17}{2}\right)^2
\end{split}\]
当$\left|n-\frac{49}{6}\right|$最小,即$n=8$或$n=9$时,$S_n$取得最大值144.
\end{solution}

\begin{example}
    求数列$\left\{\lg\left(1000\sin^n\frac{\pi}{4}\right)\right\}$的前$n$项之和$S_n$的最大值.
\end{example}

\begin{solution}
\[\begin{split}
\because\quad a_n-a_{n-1}=\lg1000\sin^n\frac{\pi}{4}-\lg1000\sin^{n-1}\frac{\pi}{4}   
&=\lg\frac{1000\sin^n\frac{\pi}{4}}{1000\sin^{n-1}\frac{\pi}{4}}\\
&=\lg\sin\frac{\pi}{4}=-\frac{1}{2}\lg 2
\end{split}\]

$\therefore\quad $该数列是等差数列,其公差为$-\frac{1}{2}\lg 2$,首项$a_1=\lg1000 \sin\frac{\pi}{4}=3-\frac{1}{2}\lg2$. 因为$a_i>0$, $d<0$,故$S_n$有最大值。

设$a_n=\lg 1000\sin^n\frac{\pi}{4}\ge 0$, 得
\[3-\frac{n}{2}\lg2\ge 0\quad \Rightarrow\quad n\le \frac{6}{\lg 2}\approx 19.9\]

答:$S_{19}$是$S_n$的最大值,其值$S_{19}\approx 28.40$.

(读者可利用另一种方法求其最大值)
\end{solution}

\begin{ex}
\begin{enumerate}
    \item 根据下列各题的条件,求相应的等差数列{am}的前n项和"S
\begin{enumerate}[(1)]
 \item $a_1=5,\quad a_n=65,\quad n=10$;
\item $a_1=100,\quad d=-2,\quad n=50$;
\item $a_1=\frac{2}{3},\quad a_n=-\frac{3}{2},\quad n=14$;
\item $a_1=14.5,\quad d=0.7,\quad a_n=32$.   
\end{enumerate}

    \item \begin{enumerate}[(1)]
        \item 求自然数列中前100个奇数的和;
        \item 求前100个自然数中,所有被3除余2的自然数的个数,以及它们的和;
        \item 两位自然数中被7除余1的各数之和。
    \end{enumerate} 

    \item 根据下列各题中的条件,求相应的等差数列$\{a_n\}$中的有关未知数:
\begin{enumerate}[(1)]
    \item $a_1=20,\quad a_n=54,\quad S_n=999$,求$d$和$n$;
\item $d=\frac{1}{3},\quad n=37,\quad S_n=629$,求$a_1$和$a_n$;
\item $a_1=\frac{5}{6},\quad d=-\frac{1}{6},\quad S_n=-5$,求$n$和$a_n$;
\item $d=2,\quad n=15,\quad a_n=-10$,求$a_1$和$S_n$.
\end{enumerate}
\item \begin{enumerate}[(1)]
\item 在正整数集合里,有多少个三位数,并求出它们的和;
\item 在三位正整数的集合中有多少个数是7的整数倍?并求它们的和,
\item 求等差数列$13,15,17,\ldots,81$的各项之和;
\item 求等差数列$10,7,4,\ldots,-47$的各项之和。
\end{enumerate}
\item 根据下列条件,求数列前多少项的和最大(或最小),并求出其最大值或最小值.
\begin{enumerate}[(1)]
    \item $\{a_n\}$是等差数列,且
\begin{multicols}{2}
\begin{enumerate}[(a)]
    \item $a_2=7,\quad a_6=23$;
    \item $a_{20}=-1,\quad d=-\frac{1}{2}$.
\end{enumerate}
\end{multicols}

\item 数列$\{a_n\}$的通项公式是
\begin{multicols}{2}
    \begin{enumerate}[(a)]
        \item $a_n=-5n+2$;
        \item $a_n=33-3n$.
    \end{enumerate}
    \end{multicols}

\item 数列$\{a_n\}$的前$n$项和为 $S_n=2n^2-23n+\frac{123}{2}$
\end{enumerate}
\end{enumerate}
\end{ex}

\section*{习题二}
\begin{center}
    \bfseries A
\end{center}

\begin{enumerate}
    \item 已知数列$\{a_n\}$是等差数列。
\begin{enumerate}[(1)]
\item 若$a_2+a_9+a_{12}+a_{19}=100$,求$S_{20}$;
\item 若$a_{10}=20$,求$S_{19}$;
\item 若$a_4=7$, $a_9=5$,求$a_{19}$和$a_{20}$.
\end{enumerate}

    \item 三个数成等差数列,它们的和等于18,它们的平方和等于116,求这三个数。
    \item  \begin{enumerate}[(1)]
        \item 某等差数列$\{a_n\}$的通项公式是$a_n=3n-2$,求它的前$n$项和;
    \item 某等差数列$\{a_n\}$的前$n$项和公式是$S_n=5n^2+3n$,求它的前3项及通项公式。
    \end{enumerate}   
    \item     一个等差数列的第6项是5,第3项与第8项之和也是5,求该数列的前9项之和。
    \item     一个屋顶的某一斜面成等腰梯形,最上面一层铺了瓦片21块,往下每一层多铺一块,斜面上共铺了瓦片19层,共铺了多少块瓦?
    \item     一个剧场设置了20排座位,第一排38个座位,往后每一排比前一排多2个座位。这个剧场一共设置了多少个座位?
\item \begin{enumerate}[(1)]
    \item 一个等差数列的第1项是5.6,第6项是20.6,求它的第四项。
    \item  一个等差数列第3项是9,第9项是3,求它的第2项。
    \item 一个等差数列第5项是$5a-b$,第2项是$2a+2b$,求第3项和第6项。
    \end{enumerate}
\item \begin{enumerate}[(1)]
\item 在12和60之间,插入3个数,使这5个数构成等差数列,求这三个数。
\item 在8和36之间插入6个数,使这8个数构成等差数列,求这六个数。
\item 在$a$和$b$之间插入10个数,使这12个数构成等差数列,求这个数列的第六项。
\end{enumerate}
\item 在通常情况下,从地面到1万米高空,高度每增加1千米,气温就下降某一固定数值。如果1千米高度的气温是$8.5^{\circ}{\rm C}$,5千米高度的气温是$-17.5^{\circ}{\rm C}$。求2千米、4千米及8千米高度的气温。
\item 安装在一个公共轴上的5个皮带轮的直径成等差数列,其中最大的与最小的皮带轮直径分别是216毫米与120毫米,求中间三个皮带轮的直径。

\end{enumerate}





\begin{center}
    \bfseries B
\end{center}

\begin{enumerate}\setcounter{enumi}{10}
    \item  某多边形的周长等于195cm,所有各边的长成等差数列,最大的边长等于40cm,公差是3cm,求多边形的边数。
    \item  一个梯形两条底边的长分别是12cm和22cm,将梯形的一条腰10等分,过每个分点画平行于梯形底边的直线,求这些直线夹在梯形两腰间的线段的长度之和。
    \item  长方体的三条棱的长成等差数列,它的对角线的长是$5\sqrt{2}$cm,全面积是$94{\rm cm^2}$,求它的体积。
    \item  一个凸多边形的各内角的度数成等差数列,其公差为$5^{\circ}$,又最小内角是$120^{\circ}$,求这个多边形的边数。
    \item 已知数$\lg1000,\; \lg1000\cos\frac{\pi}{3},\; \lg1000\cos^2 \frac{\pi}{3},\ldots, 
    \lg1000\cos^{n-1}\frac{\pi}{3},\ldots$的前多少项的和最大,并求出其
    最大值。
\end{enumerate}

\section*{三、等比数列}
\section{等比数列的有关概念}
观察下面的数列:
\begin{align}
 &1,\; 2,\; 4,\; 8,\; 16,\ldots  \tag{1}\\
&5,\; 5,\; 5,\; 5,\ldots\tag{2}\\
&1,\; -\frac{1}{2},\; \frac{1}{4},\; -\frac{1}{8},\; \frac{1}{16},\;\ldots \tag{3}   
\end{align}

\begin{itemize}
\item 数列(1)中,从第2项起每一项与前一项的比都等于2;
\item 数列(2)中,从第2项起每一项与前一项的比都等于1;
\item 数列(3)中,以第2项起每一项与前一项的比都等于$-\frac{1}{2}$.
\end{itemize}

它们的共同特点是:以第2项起,每一项与它的前一项之比都等于同一个非零常数,通常把这个常数记作$q$,即
\[\frac{a_n}{a_{n-1}}=q\quad (n\ge 2,\; n\in\N,\; q\text{为常数},\; q\ne 0)\]
这类数列叫做\textbf{等比数例},常数$q$叫做等比数列的\textbf{公比}。上述三个等比数列的公比分别是2, 1和$-\frac{1}{2}$.数列(2)说明一个非零常数列一定是等比数列,且它的公比为1。

由等比数列的定义可知,一个等比数列中的任何一项都不能是零。

如果有三个数$x$、$G$、$y$组成等比数列,那么$G$叫做$x$和$y$的\textbf{等比中项}。

根据等比数列的定义可知$\frac{G}{x}=\frac{y}{G}$,所以$G^2=xy$,
因此$G=\pm\sqrt{xy}$.

由此可知,在实数集合内,只有当$x$与$y$是同号的两个数时,它们才有等比中项,当两个数有等比中项时,必有两个等比中项,且它们互为相反数。

容易看出,在一个无穷的等比数列中,从第2项起,每一项都是它的前一项与后一项的等比中项。


\begin{example}
已知数列的通项公式为$a_n=\frac{1}{3}\x 2^n$,求证:
\begin{enumerate}[(1)]
\item 数列$\{a_n\}$是等比数列,并求其公比;
\item 求出数列的首项及第10项;
\item 判断256和$170\frac{2}{3}$是不是该数列中的项。如果是,是第几项?
\end{enumerate}
\end{example}

\begin{solution}
\begin{enumerate}[(1)]
    \item $\because\quad \frac{a_n}{a_{n-1}}=\frac{\frac{1}{3}\x 2^n}{\frac{1}{3}\x 2^{n-1}}=2$(常数)
    
$\therefore\quad $数列是等比数列,且公比是2。
\item 首项$a_1=\frac{1}{3}\x2=\frac{2}{3}$, $a_{10}=\frac{1}{3}\x2^{10}=\frac{1024}{3}$.
\item 设$a_n=\frac{1}{3}\x2^n=259$,则$n=8+\log_2 3$不是整数,所以256不是数列中的项。

设$a_n=\frac{1}{3}\x2^n=170\frac{2}{3}=\frac{512}{3}$则$n=9$,所以$170\frac{2}{3}$是
该数列的第9项。
\end{enumerate}
\end{solution}


\section{等比数列的通项公式}
如果等比数列$\{a_n\}$的第一项为$a_1$,公比为$q$,那么根据等比数列的定义得
\[\begin{split}
  a_2&=a_1q,\\
a_3&=a_2q=(a_1q)q=a_1q^2,\\
a_4&=a_3q=(a_1q^2)q=a_1q^3.\\  
\cdots &\cdots \cdots \cdots \cdots \cdots 
\end{split}\]

由此我们可以归纳出,数列的通项公式是
\[a_n=a_1q^{n-1}\quad (a_1\ne 0,\; q\ne 0)\]
其正确性,可用数学归纳法给予证明。

上面的公式还可用下面的方法得到:
\[\begin{split}
    a_1&=a_1\\
    \frac{a_2}{a_1}&=q\\
    \frac{a_3}{a_2}&=q\\
    \cdots&\cdots\cdots\\
    \frac{a_{n-1}}{a_{n-2}}&=q\\
    \frac{a_n}{a_{n-1}}&=q
\end{split}\]
将这$n$个等式的等号两边分别相乘,即可得出
\[a_n=a_1q^{n-1}\]
上面这个方法,称之为\textbf{迭乘法}。

等比数列通项公式还可以改写成
\[a_n=\frac{a_1}{q}\cdot q^n=cq^n\]
其中$c=\frac{a_1}{q}$,是一个不为零的常数。

要确定函数$y=c\cdot q^x$需要两个独立条件,因此要确定等比数列的通项公式,也需要两个独立条件。

\begin{example}
    求等比数列$\frac{\sqrt{2}}{2},\; -1,\; \sqrt{2},\; -2,\ldots$的第100项。
\end{example}

\begin{solution}
$\because\quad a_1=\frac{\sqrt{2}}{2},\quad q=-\sqrt{2}$

$\therefore\quad a_{100}=\frac{\sqrt{2}}{2}\cdot \left(-\sqrt{2}\right)^{100-1}=-\frac{\sqrt{2}}{2}\cdot \left(\sqrt{2}\right)^{99}=-2^{49}$
\end{solution}

\begin{example}
    一个等比数列的第3项与第4项分别是12与18,求它的第1项,第2项,和它的通项公式。
\end{example}

\begin{solution}
设这个数列为$\{a_n\}$,则它的公比
\[q=\frac{a_4}{a_3}=\frac{18}{12}=\frac{3}{2}\]

$\because\quad a_3=a_2\cdot q=a_2\cdot \frac{3}{2}=12$

$\therefore\quad a_2=8$

$\because\quad a_2=a_1q=a_1\cdot \frac{3}{2}=8$

$\therefore\quad a_1=\frac{16}{3},\quad a_n=a_1\cdot q^{n-1}=\frac{16}{3}\cdot \left(\frac{3}{2}\right)^{n-1}=2^{5-n}\cdot 3^{n-2}$
\end{solution}



\begin{example}
    培育水稻新品种,如果得到第一代120粒种子,并且从第一代起,以后各代的每一粒种子都可以得到下一代的120粒种子,到第五代大约可以得到这种新品种的种子多少粒(保留两位有效数字)?
\end{example}

\begin{solution}
由于每代的种子数是它的前一代种子数的120倍,逐代的种子数组成等比数列,记作$\{a_n\}$,其中$a_1=120$, $q=120$, 因此
\[a_5=120\x120^{5-1}\approx 2.5\x10^{10}\]
答:到第五代大约可以得到种子$2.5\x10^{10}$粒。
\end{solution}

    
\begin{example}
    求证数列$\{a_n\}$(其中$a_k\ne 0,\; k=1,2,\ldots$)是等比
数列的充要条件是其通项公式为$a_n=c\cdot q^n$(其中$c$、$q$均为非零常数).
\end{example}

\begin{proof}
先证“充分性”,即证“若$a_n=c\cdot q^n$,则$\{a_n\}$成等比数列”,证明如下:

$\because\quad a_n=c\cdot q^n\quad (cq\ne 0)$

$\therefore\quad \frac{a_n}{a_{n-1}}=\frac{c\cdot q^n}{c\cdot q^{n-1}}=q$(非零常数),

$\therefore\quad \{a_n\}$是等比数列。

再证“必要性”,即“若$\{a_n\}$成等比数列,则$a_n=cq^n$”.

$\because\quad\{a_n\}$成等比数列,

$\therefore\quad a_n=a_1\cdot q^{n-1}=\frac{a_1}{q}\cdot q^n$

令$c=\frac{a_1}{q}$,即得到$a_n=c\cdot q^n$·

综上可知,$\{a_n\}$成等比数列的充要条件$a_n=c\cdot q^n\quad (cq\ne 0)$
\end{proof}

\begin{example}
    在1和8之间插入5个数,使得这7个数组成等比数列,试求插入的5个数。
\end{example}

\begin{solution}
    设数列的公比为$q$. 由$a_1=1$得
\[a_7=8=a_1 q^6=q^5\]
$\therefore\quad q=\pm\sqrt{2}$

答:插入的5个数是$\sqrt{2},\; 2,\; 2\sqrt{2},\; 4,\; 4\sqrt{2}$或$-\sqrt{2},\; 2,\; -2\sqrt{2},\; 4,\; -4\sqrt{2}$.
\end{solution}

\begin{example}
    已知$\{a_n\}$为等比数列,公比为$q$,求证
$a_n=a_mq^{n-m}$.
\end{example}

\begin{proof}
    由等比数列通项公式,得
\[a_n=a_1q^{n-1},\qquad a_m=a_1q^{m-1}\]

$\therefore\quad a_n=a_1q^{m-1}\cdot q^{n-m}=a_m q^{n-m}$
\end{proof}


\begin{example}
    已知$\{a_n\}$为等比数列且$a_m=16$, $a_{m+4}=9$,求数列的通项公式。
\end{example}

\begin{solution}
设数列的公比为$q$,则$a_{m+4}=a_m q^4$,即
\[q^4=\frac{9}{16}\quad \Rightarrow\quad q=\pm\frac{\sqrt{3}}{2}\]

$\therefore\quad a_n=a_m\cdot q^{n-m}=16\x \left(\pm\frac{\sqrt{3}}{2}\right)^{n-m}$
\end{solution}


\begin{example}
    设$\{a_n\}$为等比数列。$k,\ell,m,n\in\N$,若$k+\ell=m+n$,则$a_k\cdot a_{\ell}=a_m\cdot a_n$
\end{example}

\begin{proof}
    设公比为$q$,则
\[\begin{split}
    a_{k}\cdot a_{\ell}&=a_1q^{k-1}\cdot a_1q^{\ell-1}=a^2_1\cdot q^{k+\ell-2}\\
    a_{m}\cdot a_{n}&=a_1q^{m-1}\cdot a_1q^{n-1}=a^2_1\cdot q^{m+n-2}\\
\end{split}\]

$\because\quad k+\ell=m+n$

$\therefore\quad a_k\cdot a_{\ell}=a_m\cdot a_n$
\end{proof}

\begin{rmk}
    此例说明,在等比数列中,当两项的项数和与另两项项数和相等时,对应的两项之积也相等。这是等比数列的重要性质之一。

特别地,$a_1\cdot a_n=a_2\cdot a_{n-1}=a_3\cdot a_{n-2}=\cdots$.
\end{rmk}

\begin{ex}
\begin{enumerate}
    \item 解下列各题:
\begin{enumerate}[(1)]
\item 求等比数列$5,\; -15,\; 45,\;\ldots$的第4项,第5项;
\item 求等比数列$\frac{2}{3},\; \frac{1}{2},\; \frac{3}{8},\;\ldots$的第6项;
\item 求等比数列$\sqrt{2},\; 1,\; \frac{\sqrt{2}}{2},\; \ldots$的第$n$项;
\item 求等比数列$1,\; -\frac{1}{2},\; \frac{1}{4},\; \ldots$的第$n+1$项。
\end{enumerate}

\item 在等比数列$\{a_n\}$中,设公比为$q$.
\begin{enumerate}[(1)]
\item 已知$a_9=\frac{4}{9},\quad q=-\frac{1}{3}$, 求$a_1$;
\item 已知$a_2=10,\quad a_3=20$,求$a_n$;
\item 已知$a_2=2,\quad a_5=54$,求$q$;
\item 已知$a_1=1,\quad a_n=256,\quad q=2$,求$n$;
\item 已知$a_4=27,\quad q=-3$,求$a_7$;
\item 已知$a_5-a_1=15,\quad a_4-a_2=6$,求$a_n$.
\end{enumerate}

\item 求下列各组数的等比中项。
\begin{enumerate}[(1)]
\item 45与80
\item $9\frac{3}{8}$与$1\frac{1}{2}$
\item $7+3\sqrt{5}$与$7-3\sqrt{5}$
\item $a^4+a^2b^2$与$b^4+a^2b^2\quad (2b\ne 0)$
\end{enumerate}

\item \begin{enumerate}[(1)]
\item 在9和243之间插入两个数,使它们同这两个数成等比数列;
\item 在160和5之间插入4个数,使它们同这两个数成等比数列。  
\end{enumerate}

\item 某林场计划第一年造林80亩,以后每一年比前一年多造林20\%,第五年造林多少亩(保留到个位)?
\end{enumerate}
\end{ex}

\section{等比数列前$n$项的和}
设等比数列$\{a_n\}$的公比为$q$,首项为$a_1$,由等比数列的定
义,有
\[\begin{split}
    a_2&=a_1q\\
    a_3&=a_2q\\
    a_4&=a_3q\\
    \cdots&\cdots\cdots\\
    a_{n-1}&=a_{n-2}q\\
    a_n &=a_{n-1}q
\end{split}\]
将上述$n-1\; (n\ge 2)$个等式的等号两边分别相加,得
\[a_2+a_3+\cdots +a_n=q(a_1+a_2+\cdots +a_{n-1})\]
即
\[\begin{split}
S_n-a_1&=q(S_n-a_n)\\
(1-q)S&=a_1-a_nq     
\end{split}\]
当$q\ne 1$时,$$S_n=\frac{a_1-a_n q}{1-q}$$
上式对$n=1$显然成立。

当$q=1$时,$S_n=a_1+a_2+\cdots +a_n=na_1$,这样就得出等比数列前$n$项和公式
\begin{equation}
    S_n=\begin{cases}
        na_1,  & q=1\\[1.5ex]
        \frac{a_1-a_nq}{1-q},& q\ne 1
    \end{cases}\tag{1}
\end{equation}

将等比数列的通项公式$a_n=a_1q^{n-1}$代入,可得
\begin{equation}
    S_n=\begin{cases}
        na_1,  & q=1\\[1.5ex]
        \frac{a_1(1-q^n)}{1-q},& q\ne 1
    \end{cases}\tag{2}
\end{equation}
公式(2)中$q\ne 1$时的结论还可以用下面的方法得到,设
\begin{equation}
    S_n=a_1+a_2q+a_3q^2+\cdots+a_1q^{n-2}+a_1q^{n-1} \tag{*}
\end{equation}
将上式中各项都乘以$q$,便得到
\begin{equation}
  qS_n=a_1q+a_1q^2+\cdots+a_1q^{n-1}+a_1q^n \tag{**}  
\end{equation}

比较上述两式的右端,可见,(*)的右式中,从第2项至第$n$项与(**)式右端的第1项至第$n-1$项完全相同,因此,以(*)式两边分别减去(**)式两边,便可得到
\[(1-q)S_n=a_1-a_1q^n\]
从而得出
\[S_n=\frac{a_1(1-q^n)}{1-q}\quad (q\ne 1)\]

等比数列的前$n$项和公式(1)和(2),及等比数列的通项公式$a_n=a_1q^{n-1}$其中涉及$a_1, q, n, a_n$和$S_n$这5个量。而它们又通过通项公式及前$n$项和公式联系着,因此只要已知其中的任何三个量,即可得到以其余两个量为未知数的方程组,从而可以求出其余两个量。

\begin{example}
    求等比数列$\frac{1}{2},\; \frac{1}{4},\; \frac{1}{8},\; \ldots$的第10项及前10项之和。
\end{example}

\begin{solution}
已知$a_1=\frac{1}{2},\quad q=\frac{1}{2},\quad n=10$
\[\therefore\quad a_{10}=a_1q^{10-1}=\frac{1}{2}\cdot \left(\frac{1}{2}\right)^9=\frac{1}{1024},\qquad 
S_{10}=\frac{\frac{1}{2}\left(1-\frac{1}{2^{10}}\right)}{1-\frac{1}{2}}=\frac{1023}{1024}\]
\end{solution}

\begin{example}
     某制糖厂今年制糖5万吨,如果平均每年的产量比上一年增加10%,那么从今年起,几年内可以使总产量达到30万吨(保留到个位)?
\end{example}

\begin{solution}
由题意可知,这个糖厂从今年起,平均每年的产量(万吨)组成一个等比数列,记为$\{a_n\}$,其中
\[a_1=5,\quad q=1+10\%=1.1,\quad S_n=30\]
于是得到
\[\frac{5(1-1.1^n)}{1-1.1}=30\]

整理后,得
\[1.1^n=1.6\]
两边取对数,得
\[n\lg 1.1=\lg 1.6\]

$\therefore\quad n=\frac{\lg 1.6}{\lg 1.1}=\frac{0.2041}{0.0414}\approx 5$

答:5年内可以使总产量达到30万吨。
\end{solution}

\begin{example}
    已知无穷数列
$10^{\tfrac{0}{5}},\; 10^{\tfrac{1}{5}},\; 10^{\tfrac{2}{5}},\; \ldots,10^{\tfrac{n-1}{5}},\ldots$
求证:
\begin{enumerate}[(1)]
\item 这个数列是等比数列;
\item 这个数列中任意一项是它后面第5项的$\frac{1}{10}$;
\item 这个数列中任意两项的积仍然在这个数列中;
\item 求该数列的前$n$项之积。
\end{enumerate}
\end{example}

\begin{solution}
\begin{enumerate}[(1)]
    \item 这个数列中的第$n$项与第$n+1$项分别是
$10^{\tfrac{n-1}{5}}$与$10^{\tfrac{n}{5}}\; (n\ge 1)$,于是$n+1$项与第$n$项的比为
\[\frac{a_{n+1}}{a_n}=\frac{10^{\tfrac{n}{5}}}{10^{\tfrac{n-1}{5}}}=10^{\tfrac{n}{5}-\tfrac{n-1}{5}}=10^{\tfrac{1}{5}}\]
即它们的比值是常数$10^{\tfrac{1}{5}}$,因此这个数是以$10^{\tfrac{1}{5}}$为公比的等比数列。

\item 这个数列的第$n$项$a_n=10^{\tfrac{n-1}{5}}$,第$n+5$项$a_{n+5}=10^{\tfrac{n+5-1}{5}}=10^{\tfrac{n+4}{5}}$,于是
\[\frac{a_n}{a_{n+5}}=\frac{10^{\tfrac{n-1}{5}}}{10^{\tfrac{n+4}{5}}}=10^{\tfrac{n-1}{5}-\tfrac{n+4}{5}}=10^{-\tfrac{5}{5}}=10^{-1}=\frac{1}{10}\]

这说明,这个数列中的任意一项,每经过5项后,就变大为原来的10倍,例如$a_8=10\cdot a_3,\; a_{19}=10\cdot a_{14}$等。

\item 从该数列中任取两项,假设它们分别是$a_{n_1}$和$a_{n_2}$,则$a_{n_1}=10^{\tfrac{n_1-1}{5}}$, $a_{n_2}=10^{\tfrac{n_2-1}{5}}$(其中$n_1,n_2\in\N$),那么
\[a_{n_1}\cdot a_{n_2}=10^{\tfrac{n_1-1}{5}}\cdot 10^{\tfrac{n_2-1}{5}}=10^{\tfrac{n_1-1}{5}+\tfrac{n_2-1}{5}}=10^{\tfrac{(n_1+n_2-1)-1}{5}}\]

因为$n_1\ge 1$, $n_2\ge 1$, $n_1\ne n_2$,所以$n_1+n_2>2$,即
\[n_1+n_2-1>1\]
又因为$n_1,n_2\in\N$,所以$n_1+n_2-1\in\N$,这就说明$10^{\tfrac{(n_1+n_2-1)-1}{5}}$是数列的第$n_1+n_2-1$项
\item 
\[\begin{split}
    a_1\cdot a_2\cdots a_n&=10^{\tfrac{0}{5}}\cdot 10^{\tfrac{1}{5}}\cdots 10^{\tfrac{n-1}{5}}\\
    &=10^{\tfrac{0}{5}+\tfrac{1}{5}+\cdots+\tfrac{n-1}{5}}\\&=10^{\tfrac{1}{5}\cdot \tfrac{0+n-1}{2}\cdot n}=10^{\tfrac{n^2-n}{10}}
\end{split}\]
\end{enumerate}
\end{solution}

\begin{ex}
\begin{enumerate}
    \item 根据下列条件,求相应的等比数列$\{a_n\}$的前$n$项和$S_n$:
\begin{enumerate}[(1)]
\item $a_1=3,\; q=2,\; n=6$;
\item $a_1=2.4,\; q=-1.5,\; n=5$;
\item $a_1=8,\; q=\frac{1}{2},\; n=5$;
\item $a_1=-27,\; q=-\frac{1}{3},\; n=6$.
\end{enumerate}

    \item \begin{enumerate}[(1)]
    \item 求等比数列$1,\;2,\;4,\ldots$从第5项到第10项的和;
    \item 求等比数列$\frac{3}{2},\; \frac{3}{4},\; \frac{3}{8},\ldots$从第3项到第7项的和.   
    \end{enumerate}

    \item    在等比数列$\{a_n\}$中
\begin{enumerate}[(1)]
\item 已知$a_1=-1.5$, $a_4=96$,求$q$及$S_4$;
\item 已知$q=\frac{1}{2}$, $S_6=3\frac{7}{8}$, 求$a_1$与$a_5$;
\item 已知$a_1=2$, $S_3=26$,求$q$与$a_3$;
\item 已知$a_3=1\frac{1}{2}$,$S_3=4\frac{1}{2}$,求$a_1$与$q$.   
\end{enumerate}

\end{enumerate}
\end{ex}

\section*{习题三}
\begin{center}
    \bfseries A
\end{center}

\begin{enumerate}
    \item 在等比数列$\{a_n\}$中:
\begin{multicols}{2}
 \begin{enumerate}[(1)]
    \item 已知$n$,$q$,$a_n$,求$a_1$与$S_n$
    \item 已知$q$,$n$,$S_n$,求$a_1$与$a_n$
    \item 已知$a_1$,$q$,$S_n$,求$a_n$
    \item 已知$q$,$a_n$,$S_n$,求$a_1$
\end{enumerate}   
\end{multicols}
\item 某工厂去年的产值是138万元,计划在今后5年内每年比上一年产值增长10\%。从今年起,到第5年这个厂的年产值是多少?这5年的总产值是多少(精确到万元)?
\item 画一个边长为2cm的正方形,再以这个正方形的对角线为边画第二个正方形,以第二个正方形的对角线为边画第3个正方形,这样一共画了10个正方形。求:
\begin{multicols}{2}
\begin{enumerate}[(1)]
\item 第10个正方形的面积;
\item 这10个正方形的面积之和。
\end{enumerate}
\end{multicols}
\item 一个球从100米高处自由落下,每次着地后又跳回到原来高度的一半再落下。当它第10次着地时,共经过了多少米(保留到个位)?
\item 从盛满20升纯酒精的容器里倒出1升,然后用水填满,再倒出1升混合液,用水填满,这样继续进行,一共倒了3次,这时容器里还有多少升纯酒精(保留到个位)?
\item 三个数成等比数列,它们的和等于14,它们的积等于64,求这三个数。
\item 抽气机的活塞每运动一次,从容器里抽出$\frac{1}{8}$的空气,因而使容器里空气的压强降低为原来的$\frac{7}{8}$,已知最初容器里空气的压强是750毫米高水银柱,求活塞运动5次后容器里空气的压强(保留到个位)。
\item 某种细菌在培养过程中,每30分钟分裂一次(一个分裂为两个),经过4小时,这种细菌由1个可繁殖成多少个?电动机轴的直径从小到大共有5种尺寸,它们的数值(单位:mm)组成一个等比数列,其中最小的数值为40,最大的数值为100,求它们的公比(保留到千分位)。
\item 一个工厂今年生产某种机器1080台,计划到后年把产量提高到每年生产机器1920台,如果每一年比上一年增长的百分率相等,求这个百分率(精确到1\%)。
\end{enumerate}

\begin{center}
    \bfseries B
\end{center}

\begin{enumerate}\setcounter{enumi}{10}
    \item 设等比数列$\{a_n\}$的公比是$q$,求证:
    $a_1a_2\cdots a_n=a^n_1q^{\tfrac{n(n-1)}{2}}$
    \item 一个等比数列的各项都是正数,求证这个数列的各项的对数组成等差数列。
    \item 已知$a_1,a_2,a_3,\ldots$是等差数列,$C$是正的常数,求证$C^{a_1},C^{a_2},C^{a_3},\ldots$是等比数列。
    \item 成等差数列的三个正数之和是15,并且将这三个数分别加上$1,4,19$后,就成等比数列,求这三个数。
    \item 某工厂的三年生产计划是:每年比上一年增产机器的台数相同。如果第三年比原计划多生产1000台,那么每年比上一年增长的百分数就相同,而且第三年生产的台数恰为原计划三年生产总台数的一半,问原计划每年各生产机器多少台?
    \item 已知无穷数列$10^{\tfrac{0}{10}},10^{\tfrac{1}{10}},10^{\tfrac{2}{10}},\ldots,10^{\tfrac{n-1}{10}},\ldots$,求证:
\begin{enumerate}[(1)]
    \item 这个数列是以$10^{\tfrac{1}{10}}$为公比的等比数列;
    \item 这个数列中的任意一项是它后面第10项的$\frac{1}{10}$;
    \item 这个数列中的任意两项的积仍然在这个数列中。
\end{enumerate}

\item 在数列$\{a_n\}$中,已知$a_1=a\ne 0$,其前$n$项为$S_n$,若$S_1,S_2,\ldots,S_n$是以$q$为公比的等比数列,求证:$a_2,a_3,\ldots,a_n$也是等比数列。
\end{enumerate}



 \section*{四、特殊数列求和问题}
本节学习等差数列与等比数列以外的一些特殊数列的求和的方法。

\section{可转化为等差(或等比)数列求和}

\subsection{分组转化法}


\begin{example}
    求$S_n=\left(x+\frac{1}{y}\right)+\left(2x+\frac{1}{y^2}\right)+\cdots+\left(nx+\frac{1}{y^n}\right)$
\end{example}


\begin{analyze}
    表面上看,数列既非等差又非等比。如果对每个括号内的前项或后项分别考察,则依次构成等差数列和等比数列。
\end{analyze}

\begin{solution}
令$S'_n=x+2x+\cdots+nx=\frac{x+nx}{2}\cdot n=\frac{n(n+1)}{2}x$

$S''_n=\frac{1}{y}+\frac{1}{y^2}+\cdots+\frac{1}{y^n}=\begin{cases}
    n,& y=1\\[1.5ex]
    \frac{y^n-1}{y^n(y-1)},& y\ne 1
\end{cases}$

$\therefore\quad S_n=S'_n+S''_n=\begin{cases}
    \frac{n(n+1)}{2}x+n,& y=1\\[1.5ex]
    \frac{n(n+1)}{2}x+\frac{y^n-1}{y^n(y-1)},& y\ne 1
\end{cases}$
\end{solution}

\begin{example}
    求下面各数列的和:
\begin{multicols}{2}
\begin{enumerate}[(1)]
    \item $9,\; 99,\; 999,\; \ldots, \overbrace{99\cdots 9}^{\text{$n$个9}}$
    \item $5,\; 55,\; 555,\; \ldots, \overbrace{55\cdots 5}^{\text{$n$个5}}$
\end{enumerate}
\end{multicols}
\end{example}

\begin{analyze}
直接求和是不可设想的。然而对于(1),把每一项分别改写为$10-1,\; 100-1,\; \ldots,10^n-1$,便很容易得到解决;对于(2)只需把数列各项改写为$9\cdot\frac{5}{9},\; 99\cdot\frac{5}{9},\ldots, \overbrace{99\cdots 9}^{\text{$n$个9}}\cdot\frac{5}{9}$,即可转化为(1).
\end{analyze}

\begin{solution}
\begin{enumerate}[(1)]
    \item $S_n=(10-1)+(100-1)+\cdots +(10^n-1)=10+100+\cdots+10^n - n=\frac{10}{9}(10^n-1)-n$
\item $S_n=\frac{5}{9}\left[\frac{10}{9}(10^n-1)-n\right]=\frac{50}{81}(10^n-1)-\frac{5}{9}n$
\end{enumerate}
\end{solution}


\begin{example}
    试求在不大于100的自然数中,能被2或3整除的各数之和。
\end{example}

\begin{analyze}
若将满足条件的数一一写出,然后相加,并不困难;但如果把题中的“100”改为“1000”或更大的数,那么将是极为麻烦的。

我们把满足条件的数中,能被2整除,被3整除,以及既被2整除又被3整除的数,分门别类地求和,再寻求最后的答案,便得到了解决问题的一般方法。
\end{analyze}

\begin{solution}
设$S'_1$,$S'_2$,$S'_3$分别表示100以内能被2整除,能被3整除以及既能被2整除又能被3整除的各数之和,则    
\[\begin{split}
    S'_1&=\frac{2+100}{2}\x 50=2550\\
    S'_2&=\frac{3+99}{2}\x 33=1683\\
    S'_3&=\frac{6+96}{2}\x 16=816\\
\end{split}\]

设满足条件的各数之和为$S$. 那么
\[S=S'_1+S'_2-S'_3=3417\]
\end{solution}

\subsection{“差比数列”的求和方法}



\begin{example}
    设$a\ne 0$, $a\ne 1$,求数列$a,\; 2a^2,\; 3a^3,\ldots,na^n,\ldots$的前$n$项和。
\end{example}

\begin{analyze}
该数列既非等差数列,又非等比数列,但系数部分构成等差数列,字母部分构成等比数列。这种数列不妨称为“差比数列”。可利用推导等比数列求和公式的第二种方法
来求和。
\end{analyze}

\begin{solution}
设
\begin{align}
 S_n&=a+2a^2+3a^3+\cdots +na^n\tag{1}\\
    a\cdot S_n&=a^2+2a^3+3a^4+\cdots +na^{n+1} \tag{2}
\end{align}
(1)与(2)的两边分别相减,得
  \[  (1-a)S_n=a+a^2+\cdots +a^n-n\cdot a^{n+1}\]    

$\because\quad a\ne 1$

$\therefore\quad S_n=\frac{1}{1-a}\left[\frac{a(1-a^n)}{1-a}-na^{n+1}\right]=\frac{a(1-a^n)}{(1-a)^2}-\frac{na^{n+1}}{1-a}$
\end{solution}

\begin{rmk}
“差比数列”的一般形式为$\{C_n\}$, $C_n=a_n\cdot b_n$,其中$\{a_n\}$为等差数列,$\{b_n\}$为等比数列。本例题给出了“差比数列”求和的一般方法。    
\end{rmk}

\begin{ex}
\begin{enumerate}
    \item 求和:
\begin{enumerate}[(1)]
\item $1\frac{1}{2}+3\frac{1}{4}+5\frac{1}{8}+\cdots+\left(2n-1+\frac{1}{2^n}\right)$
\item $\frac{1}{2}+\frac{3}{4}+\frac{7}{8}+\cdots+\frac{2^n-1}{2^n}$
\end{enumerate}

    \item 求数列$1,\; -3,\; 5,\; -7,\ldots,(-1)^{n+1}+(2n-1),\ldots$的前$2n$项的和$S_{2n}$。
    \item \begin{enumerate}[(1)]
    \item 求在小于300的自然数中,能被6整除,但不能被8整除的各数之和
    \item 求集合$\{m\mid m=7n,\; \text{且}m=5k\; (n,k\in\N)\; \text{且}m<1000\}$中元素的个数和它们的和
    \item 求1000以内的既不能被7整除,又不能被5整除的
    所有自然数的和$S$.
    \item 在前100个自然数中,所有既能被6整除,又能被4整除的数有多少个?并求其和。
    \end{enumerate} 
\item 求数列$0.9,\; 0.99,\; \ldots, 0.\overbrace{99\cdots 9}^{\text{$n$个9}}$的前$n$项的和
\item 求数列$\frac{1}{2},\; -1\frac{1}{3},\; 2\frac{1}{2},\; -3\frac{1}{3},\; 4\frac{1}{2},\; -5\frac{1}{3},\ldots,98\frac{1}{2},\; -99\frac{1}{3}$的和。
\item 求数列前$n$项的和$S_n$
\begin{enumerate}[(1)]
    \item $1,\; 3a,\; 5a^2,\ldots, (2n-1)a^{n-1},\ldots$
    \item $1,\; \frac{4}{5},\;\frac{7}{25},\ldots,\frac{3^{n-2}}{5^{n-1}},\ldots$
    \item $\frac{1}{3},\; \frac{2}{3^2},\; \frac{1}{3^3},\;\frac{2}{3^4},\; \frac{1}{3^5},\; \frac{2}{3^6},\ldots$
\end{enumerate}
\end{enumerate}   
\end{ex}

\section{裂项求和法}

\begin{example}
    求下列各式的和
\begin{enumerate}[(1)]
\item $\frac{1}{1\x 2}+\frac{1}{2\x 3}+\frac{1}{3\x 4}+\cdots +\frac{1}{n(n+1)}$
\item $\frac{1}{2\x 4}+\frac{1}{4\x 6}+\frac{1}{6\x 8}+\cdots +\frac{1}{2n(2n+2)}$
\end{enumerate}
\end{example}

\begin{analyze}
    对于(1),考虑到数列$\left\{\frac{1}{n(n+1)}\right\}$的每一项都可化为两项之差:
\[\frac{1}{1\x 2}=1-\frac{1}{2},\; \frac{1}{2\x 3}=\frac{1}{2}-\frac{1}{3},\ldots, \frac{1}{n(n+1)}=\frac{1}{n}-\frac{1}{n+1}\]
于是所求之和为
\[\left(1-\frac{1}{2}\right)+\left(\frac{1}{2}-\frac{1}{3}\right)+\cdots +\left(\frac{1}{n}-\frac{1}{n+1}\right)\]

显然,其中前一个括号的后项,与其后一个括号的前项相抵消。故称为裂项抵消法。

对于(2)可以类似地处理
\end{analyze}

\begin{solution}
\begin{enumerate}[(1)]
    \item \[\begin{split}
    \text{原式}&= \left(1-\frac{1}{2}\right)+\left(\frac{1}{2}-\frac{1}{3}\right)+\cdots +\left(\frac{1}{n}-\frac{1}{n+1}\right)\\
    &=1-\frac{1}{n+1}=\frac{n}{n+1}
    \end{split}\]
    \item \[\begin{split}
    \text{原式}&= \frac{1}{2}\left[\left(\frac{1}{2}-\frac{1}{4}\right)+\left(\frac{1}{4}-\frac{1}{6}\right)+\cdots +\left(\frac{1}{2n}-\frac{1}{2n+2}\right)\right]\\
&=\frac{1}{2}\left(\frac{1}{2}-\frac{1}{2n+2}\right)=\frac{n}{4(n+1)}        
    \end{split}\]
\end{enumerate}
\end{solution}

\begin{rmk}
    若(2)采取下述方法,可更简便:
\[\text{原式}=\frac{1}{4}\left[\frac{1}{1\x 2}+\frac{1}{2\x 3}+\frac{1}{3\x 4}+\cdots +\frac{1}{n(n+1)}\right]\]
可归结为(1).
\end{rmk}

\begin{example}
    求数列$\left\{\frac{1}{(4n-3)(4n+1)}\right\}$的前$n$项和$S_n$.
\end{example}

\begin{analyze}
以通项入手进行分析
\[\frac{1}{(4n-3)(4n+1)}=\frac{1}{4}\left(\frac{1}{4n-3}-\frac{1}{4n+1}\right)\]
于是采用例5.42的解法即可.
\end{analyze}

\begin{solution}
\[\begin{split}
S_n&=\frac{1}{1\x 5}+\frac{1}{5\x 9}+\cdots +\frac{1}{(4n-3)(4n+1)}\\
&=\frac{1}{4}\left[\left(1-\frac{1}{5}\right)+\left(\frac{1}{5}-\frac{1}{9}\right)+\cdots +\left(\frac{1}{4n-3}-\frac{1}{4n+1}\right)\right]\\
&=\frac{1}{4}\left(1-\frac{1}{4n+1}\right)=\frac{n}{4n+1}
\end{split}\]    
\end{solution}


\begin{example}
    求数列$\left\{\frac{1}{2n(2n+4)}\right\}$的前$n$项和$S_n$.
\end{example}

\begin{analyze}
    可将其通项化简:
\[\frac{1}{2n(2n+4)}=\frac{1}{4}\cdot \frac{1}{n(n+2)}=\frac{1}{4}\cdot \frac{1}{2}\left(\frac{1}{n}-\frac{1}{n+2}\right)\]
\end{analyze}

\begin{solution}
\[\begin{split}
S_n&=\frac{1}{8}\left[\left(1-\frac{1}{3}\right)+\left(\frac{1}{2}-\frac{1}{4}\right)+\left(\frac{1}{3}-\frac{1}{5}\right)+\right.\\
&\qquad \qquad \left.\cdots +\left(\frac{1}{n-1}-\frac{1}{n+1}\right)+\left(\frac{1}{n}-\frac{1}{n+2}\right)\right]\\
&=\frac{1}{8}\left[1+\frac{1}{2}-\frac{1}{n+1}-\frac{1}{n+2}\right]\\
&=\frac{3}{16}-\frac{1}{8}\left(\frac{1}{n+1}+\frac{1}{n+2}\right)
\end{split}\]
\end{solution}

\begin{rmk}
本例与例5.42、例5.43不同。裂项抵消之后,并非只剩两项,而是剩四项:$1,\; \frac{1}{2},\; -\frac{1}{n+1},\; -\frac{1}{n+2}$.
\end{rmk}
    
\begin{ex}
求和:
\begin{enumerate}
    \item $\frac{1}{1\x 4}+\frac{1}{4\x 7}+\frac{1}{7\x 10}+\cdots +\frac{1}{(3n-2)(3n+1)}$
    \item $\frac{1}{1\x 4}+\frac{1}{2\x 5}+\frac{1}{3\x 6}+\cdots +\frac{1}{n(n+3)}$
\end{enumerate}
\end{ex}

\section{前$n$个自然数的平方和的求法及应用}

\begin{example}
    求和$S_n=1^2+2^2+\cdots +n^2\quad (n\in\N)$
\end{example}

\begin{analyze}
    求和的方法很多,这里只介绍利用两数和的立方公式求此和的方法。
\end{analyze}

\begin{solution}
$\because\quad (m+1)^3=m^3+3m^2+3m+1$,分别取$m=1,2,\ldots,n-1,n$得:
\[\begin{split}
    (1+1)^3&=1^3+3\x 1^2+3\x 1+1\\
    (2+1)^3&=2^3+3\x 2^2+3\x 2+1\\
    (3+1)^3&=3^3+3\x 3^2+3\x 3+1\\
    \cdots&\cdots\cdots\cdots\cdots\\
    (n+1)^3&=n^3+3\x n^2+3\x n+1\\
\end{split}\]

将上述$n$个等式的两边分别相加,得
\[(n+1)^3=1^3+3(1^2+2^2+\cdots+n^2)+3(1+2+\cdots +n)+n\]

\[\begin{split}
\therefore\quad  1^2+2^2+\cdots +n^2 &=\frac{1}{3}\left[(n+1)^3-1-3(1+2+\cdots +n)-n\right]\\
&=\frac{1}{3}\left[n^3+3n^2+3n-3\cdot \frac{n(n+1)}{2}-n\right]\\
&=\frac{1}{6}n(n+1)(2n+1) 
\end{split}\]
此即前$n$个自然数的平方和的公式:
\[1^2+2^2+\cdots +n^2=\frac{1}{6}n(n+1)(2n+1)  \]
\end{solution}

\begin{rmk}
\begin{enumerate}
    \item 本例给出的方法可以推广。例如,利用$(m+1)^4=m^4+4m^3+6m^2+4m+1$可求出前$n$个自然数的立方和$1^3+2^3+\cdots +n^3=\frac{1}{4}n^2(n+1)^2$等等。
    \item 可根据该公式解决数列$\{an^2+bn+c\; (a\ne 0)\}$的求和问题。
\end{enumerate}
\end{rmk}

\begin{example}
求下列各数列的前$n$项的和$S_n$。
\begin{enumerate}[(1)]
    \item $1\x2,\; 2\x3,\; 3\x4,\ldots,n(n+1),\ldots$
\item $2\x5,\; 3\x6,\; 4\x7,\ldots,(n+1)(n+4),\ldots$
\end{enumerate}
\end{example}

\begin{analyze}
    欲求数列的前$n$项和,可从对通项公式变形入手。例如(1)通项为$n^2+n$. 于是$S_n=(1^2+2^2+\cdots +n^2)+(1+2+\cdots+n)$,问题可解. (2)也类似.    
\end{analyze}

\begin{solution}
\begin{enumerate}[(1)]
    \item \[\begin{split}
S_n&=(1^2+2^2+\cdots+n^2)+(1+2+\cdots +n)\\
&=\frac{1}{6}n(n+1)(2n+1)+\frac{1}{2}n(n+1)\\
&=\frac{1}{3}n(n+1)(n+2)
    \end{split}\]
    \item $\because\quad (n+1)(n+4)=n^2+5n+4$
    \[\begin{split}
\therefore\quad S_n&=(1^2+2^2+\cdots+n^2)+5(1+2+\cdots +n)+4n\\
&=\frac{1}{6}n(n+1)(2n+1)+\frac{5}{2}n(n+1)+4n\\
&=\frac{1}{6}n[2n^2+3n+1+15n+15+24]\\
&=\frac{1}{3}n(n^2+9n+20)=\frac{1}{3}n(n+4)(n+5)
    \end{split}\]
\end{enumerate}
\end{solution}

\begin{figure}[htp]
    \centering
\begin{tikzpicture}[scale=.5]
\newcommand{\cube}{
\draw(0,0)--(0,-1);
\foreach \x in {0,1,2,...,5}
{
    \tkzDefPoint(30+60*\x:1){A\x}
}
\tkzDefPoints{0/0/O}
\tkzDrawPolygon(A0,A1,A2,A3,A4,A5)
\tkzDrawPolygon[pattern=north east lines](A0,A1,A2,O)}
\begin{scope}
\cube
\end{scope}
\begin{scope}[xshift=.866cm, yshift=-1.5cm]
\cube
\end{scope}
\begin{scope}[xshift=-.866cm, yshift=-1.5cm]
    \cube
\end{scope}
\begin{scope}[xshift=2*.866cm, yshift=-3cm]
\cube
\end{scope}
\begin{scope}[xshift=-2*.866cm, yshift=-3cm]
    \cube
\end{scope}
\begin{scope}[yshift=-3cm]
    \cube
\end{scope}
\begin{scope}[xshift=3*.866cm, yshift=-4.5cm]
\cube
\end{scope}
\begin{scope}[xshift=-3*.866cm, yshift=-4.5cm]
    \cube
\end{scope}
\begin{scope}[xshift=.866cm, yshift=-4.5cm]
    \cube
\end{scope}
\begin{scope}[xshift=-.866cm, yshift=-4.5cm]
    \cube
\end{scope}
\end{tikzpicture}
    \caption{}
\end{figure}


\begin{example}
一堆零件堆积如图5.7,第1层1个,第2层$(1+2)$个,第3层$(1+2+3)$个,……,求$n$层的总个数$S_n$。
\end{example}

\begin{analyze}
    依题意,第$k$层有$1+2+3+\cdots +k=\frac{1}{2}k(k+1)=\frac{1}{2}k^2+\frac{1}{2}k$
\[\begin{split}
    \therefore\quad S_n&=\frac{1}{2}(1^2+2^2+\cdots +n^2)+\frac{1}{2}(1+2+\cdots+n)\\
    &=\frac{1}{2}\cdot \frac{1}{6}n(n+1)(2n+1)+\frac{1}{2}\cdot \frac{1}{2}n(n+1)\\
    &=\frac{1}{6}n(n+1)(n+2)
\end{split}\]


$\therefore\quad n$层的总个数为$\frac{1}{6}n(n+1)(n+2)$
\end{analyze}

\begin{rmk}
本例的实质是求数列$1,\; 1+2,\; 1+2+3,\ldots,1+2+3+\cdots +n,\ldots$的前$n$项之和。
\end{rmk}


\begin{ex}
\begin{enumerate}
    \item 求和$11^2+13^2+15^2+\cdots +(2n+9)^2$
    \item 利用$(m+1)^4=m^4+4m^3+6m^2+4m+1$,推导前$n$个自然数的立方和的公式
\[1^3+2^3+\cdots +n^3=\frac{1}{4}n^2(n+1)^2\]
\item 利用上题的公式,求和:
\begin{enumerate}[(1)]
    \item $1^3+3^3+5^3+\cdots +(2n-1)^3$
    \item $1\x 2\x 3+2\x 3\x 4+\cdots +n(n+1)(n+2)$
\end{enumerate}
\end{enumerate}
\end{ex}


\section*{习题四}
\begin{center}
    \bfseries A
\end{center}

\begin{enumerate}
    \item 求下列各数列的前$n$项的和$S_n$,已知数列的通项公式为
\begin{multicols}{2}
\begin{enumerate}[(1)]
    \item $a_n= 2n+\frac{1}{3^n}$
    \item $a_n= a^n-n$
    \item $a_n= 0.\overbrace{77\cdots 7}^{\text{$n$个7}}$
\end{enumerate}
\end{multicols}

\item 由下图所示,这$n^2$个自然数之和为14400,求$n$
\[\begin{split}
& 1,\; 2,\; 3,\ldots, n\\
&2,\; 4,\; 6,\ldots,2n\\
&3,\; 6,\; 9,\ldots,3n\\
&\cdots \cdots \cdots \cdots \cdots \\
&n,\; 2n,\; 3n,\ldots,n^2   
\end{split}\]

\item 求下列各数列的前$n$项和
\begin{enumerate}[(1)]
    \item $\frac{1}{1+\sqrt{2}},\; \frac{1}{\sqrt{2}+\sqrt{3}},\; \frac{1}{\sqrt{3}+\sqrt{4}},\ldots,\; \frac{1}{\sqrt{n}+\sqrt{n+1}}$
    \item $\frac{1}{2^2-1},\; \frac{1}{4^2-1},\; \frac{1}{6^2-1},\ldots, \frac{1}{(2n)^2-1}$
\end{enumerate}

\end{enumerate}

\begin{center}
    \bfseries B
\end{center}

\begin{enumerate}\setcounter{enumi}{3}
    \item 求下列各式的和:
\begin{enumerate}[(1)]
    \item $\frac{1}{2}+\frac{1}{6}+\frac{1}{12}+\frac{1}{20}+\frac{1}{30}+\frac{1}{42}+\frac{1}{56}+\frac{1}{72}+\frac{1}{90}$
    \item $1+\frac{1}{1+2}+\frac{1}{1+2+3}+\cdots+\frac{1}{1+2+3+\cdots +100}$
\end{enumerate}

\item 设数列$\{a_n\}$为等差数列,$a_n\ne 0\; (n\in\N)$,求证
\[\frac{1}{a_1a_2}+\frac{1}{a_2a_3}+\cdots +\frac{1}{a_{n-1}a_n}=\frac{n-1}{a_1a_n}\]
\item 设$\{a_n\}$为等比数列,且$a_n>1\; (n\in\N)$,求证
\[\frac{1}{\lg a_1\cdot \lg a_2}+\frac{1}{\lg a_2\cdot \lg a_3}+\cdots +\frac{1}{\lg a_{n-1}\cdot \lg a_n}=\frac{n-1}{\lg a_1\cdot \lg a_n}\]

\end{enumerate}


\section*{五、数学归纳法}

\section{演绎法与归纳法}
\subsection{演绎法}
演绎法是从普遍性的规律(如概念、公理、定理等)出发去认识特殊的,个别的研究对象的方法,即从一般到特殊的推理方法。

演绎法是数学推理的重要方法,其基本模式是三段论法,即:
\begin{enumerate}
\item 大前提:已知的一般原理,
\item 小前提:所研究特殊事物的特征,
\item 结论:从已知的一般原理结合特殊事物的特征做出的判断。
\end{enumerate}

例如,对$y=f(x)$定义域内任何$x$,都有$f(-x)=-f(x)$, 则$f(x)$是奇函数。今有函数$f(x)=-x^3+\sin x$. 显然$f(-x)=-(-x)^3+\sin(-x)=x^3-\sin x=-f(x)$,故$f(x)$是奇函数.

\subsection{归纳法}
归纳法是通过考察事物的部分对象而得到的有关事物的
一般性结论的方法。即从特殊到一般的推理方法。

例如,观察下列各式:
\[\begin{split}
    1&=1^2\\
1+3&=4=2^2\\
1+3+5&=9=3^2\\
1+3+5+7&=16=4^2\\
\cdots\cdots&\cdots\cdots
\end{split}\]

通过归纳得出:“自然数中,前$n$个奇数之和等于$n$的平方”,即
\[1+3+5+···+(2n-1)=n^2\]

这样得出的结论只是一种猜想(假说),其结论是否对任意自然数$n$都正确,还不能确定。

\subsection{不完全归纳法与完全归纳法}
不完全归纳法,是“在考察某类事物的部分对象后概括出某种属性的思维方法”。由于它只考察了部分对象而得出的结论,对于没考察到的对象,是否也都符合,则很难确保其结论的可靠性。例如,$f(n)=n^2+n+41$,当$n$依次取$1,2,3,\ldots,39$时,所得结果$f(n)$均为素数,如果,由此归纳出:“$n$为任意自然数时,$f(n)$必是素数”的结论,就是不正确的。因为$n=40$时,$f(n)=40^2+40+41=41^2$,不是素数。

完全归纳法,则是“在考察某类事物的全部对象之后,概括出事物的某种属性”的方法。

例如,正弦定理$\frac{a}{\sin A}=\frac{b}{\sin B}=\frac{c}{\sin C}=2R$,其中$a$、$b$、$c$依次为$\triangle ABC$中角$A$、$B$、$C$的对边,$R$为三角形外接圆的半
径。是在证明了,无论其圆心是在$\triangle ABC$的边上,在$\triangle ABC$内部,还是外部(因为只有这三种情况),都是正确的,因此断言,对任何三角形,上述结论都是正确的。

正弦定理的证明方法就是完全归纳法。即考察某类事物的所有对象的一切可能的特殊情况,将其一一列举,无一遗漏,因此也称为穷举法,所得结论必然是正确的。

完全归纳法所得结论是可靠的。

\section{数学归纳法}
对于由归纳法得到的某些与自然数有关的数学命题。如“自然数中,前$n$个奇数之和等于$n^2$”. 若对任何自然数$n$都一一考察到是不可能的,我们常采用下面的方法来证明其正确性:

第一步,证明当$n$取第一个值$n_0$(例如:$n_0=1$)时命题成立;

第二步,假设$n=k$时命题成立,证明$n=k+1$时命题也成立(由此可断定这个命题对于$n$取第一个值$n_0$后面的所有自然数也都成立)。

这种证明方法,叫做\textbf{数学归纳法}。

例如,我们用数学归纳法来证明,如果$\{a_n\}$是一个等差数列,那么
$a_n=a_1+(n-1)d$
对一切$n\in\N$都成立。

\begin{proof}
\begin{enumerate}[(1)]
    \item 当$n=1$时,左边是$a_1$,右边是$a_1+0d=a_1$, 等式是成立的。
    \item 假设当$n=k$时等式成立,就是
    \[a_k=a1+(k-1)d\]
    那么,
\[    a_{k+1}=a_k+d    =a_1+(k-1)d+d    =a_1+[(k+1)-1]d\]

这就是说,当$n=k+1$时,等式也成立。

根据(1),$n=1$时等式成立,再根据(2),$n=1+1=2$时等式也成立。由于$n=2$时等式成立,再根据(2),$n=2+1=3$时等式也成立。这样递推下去,就知道$n=4,5,6,\ldots$时等式都成立。因此根据(1)和(2)可以断定,等式对任何$n\in\N$都成立。
\end{enumerate}
\end{proof}

从上面的例子看到,用数学归纳法证明一个与自然数有关的命题的步骤是:
\begin{enumerate}[(1)]
\item 证明当$n$取第一个值$n_0$(例如$n_0=1$或2等)时结论正确;
\item 假设当$n=k\; (k\in\N,\; \text{且}k\ge n_0)$时结论正确,证明当$n=k+1$时结论也正确。
\end{enumerate}

在完成了这两个步骤以后,就可以断定命题对于从$n_0$开始的所有自然数$n$都正确。

\begin{example}
    用数学归纳法证明
$1+3+5+\cdots +(2n-1)=n^2$
\end{example}

\begin{proof}
\begin{enumerate}[(1)]
    \item 当$n=1$时,左边$=1$,右边$=1$,等式成立。
    \item 假设当$n=k$时等式成立,就是
   $ 1+3+5+\cdots +(2k-1)=k^2$,
    那么,
\[\begin{split}
    1+3+5+\cdots +(2k-1)+[2(k+1)-1]&=k^2 +[2(k+1)-1]\\
    &=k^2+2k+1=(k+1)^2
\end{split}\]
这就是说,当$n=k+1$时等式也成立。
\end{enumerate}

\noindent
\begin{minipage}{.5\textwidth}
    \CTEXindent
    根据(1)和(2),可知等式对任何$n\in\N$都成立。

本例所证明的等式($n=5$)可以用图5.8表示出来。

用数学归纳法证明命题的这两个步骤,是缺一不可的。若只完成(1),而缺(2),则是不完全归纳法,不能保证结论的正确性;若只有步骤(2)而缺少步骤(1),也可能得出不正确的结论。例如,假设$n=k$时,等式
$2+4+6+\cdots +2n=n^2+n+1$
成立,就是
\end{minipage}\hfill
\begin{minipage}{.45\textwidth}
\centering
\begin{tikzpicture}[scale=.8]
\draw(0,0) rectangle (5,5);
\foreach \x in {1,2,3,4}
{
    \draw(\x,0)--(\x,5);
    \draw(0,\x)--(5,\x);
}

\draw[pattern=north east lines](0,1) rectangle (2,2);
\draw[pattern=north east lines](0,3) rectangle (4,4);
\draw[pattern=north east lines](1,0) rectangle (2,1);
\draw[pattern=north east lines](3,0) rectangle (4,3);
\node at (.5,.5){1};
\node at (1.5,1.5)[fill=white]{3};
\node at (2.5,2.5){5};
\node at (3.5,3.5)[fill=white]{7};
\node at (2.5,0)[below]{$n$};
\node at (5,2.5)[right]{$n$};


\end{tikzpicture}
\captionof{figure}{}
\end{minipage}

\[2+4+6+\cdots +2k=k^2+k+1\]
那么,
\[\begin{split}
  2+4+6+\cdots+2k+2(k+1)&=k^2+k+1+2(k+1)\\
&=(k+1)^2+(k+1)+1
\end{split}\]
这就是说,如果$n=k$时等式成立,那么$n=k+1$时等式也成立。但如果仅根据这一步就得出等式对于任何$n\in\N$都成立的结论,那就错了。事实上,当$n=1$时,上式左边$=2$,右边$=1^2+1+1=3$,左边$\ne $右边。这也说明,如果缺少步骤(1)这个基础,步骤(2)就没有意义了。
\end{proof}

\begin{ex}
    用数学归纳法证明:
\begin{enumerate}
 \item $1+2+3+\cdots +n=\frac{1}{2}n(n+1)$.
\item 首项是$a_1$,公比是$q$的等比数列的通项公式是
$a_n=a_1q^{n-1}$.
\end{enumerate}
\end{ex}

\section{数学归纳法的应用举例}
\begin{example}
   用数学归纳法证明
\[1^2+2^2+3^2+\cdots +n^2=\frac{n(n+1)(2n+1)}{6}\quad (n\in\N)\]
\end{example}

\begin{proof}
\begin{enumerate}[(1)]
    \item 当$n=1$时,左边$=1^2=1$,
    右边$=\frac{1(1+1)(2+1)}{6}=1$,等式成立.
    \item 假设当$n=k\; (k\in\N)$时等式成立,就是
  \[  1^2+2^2+3^2+\cdots+k^2=\frac{k(k+1)(2k+1)}{6}\]
那么
\[\begin{split}
    1^2+2^2+3^2+\cdots+k^2+(k+1)^2&=\frac{1}{6}k(k+1)(2k+1)+(k+1)^2\\
    &=\frac{1}{6}(k+1)[2k^2+k+6k+6]\\
    &=\frac{1}{6}(k+1)(k+2)(2k+3)\\
&=\frac{1}{6}(k+1)[(k+1)+1][2(k+1)+1]
\end{split}\]   

这就是说,当$n=k+1$时等式成立。

根据(1)和(2),等式对任何$n\in\N$都成立。
  \end{enumerate}  
\end{proof}

\begin{example}
    用数学归纳法证明:
    \[\frac{1}{1\cdot 2}+\frac{1}{2\cdot 3}+\frac{1}{3\cdot 4}+\cdots +\frac{1}{n(n+1)}=\frac{n}{n+1}\]
\end{example}

\begin{proof}
\begin{enumerate}[(1)]
    \item 当$n=1$时,左边$=\frac{1}{1\cdot 2}=\frac{1}{2}$,右边$=\frac{1}{1+1}=\frac{1}{2}$,等式成立。
   \item 假设当$n=k\; (k\in\N)$时等式成立,就是
   \[\frac{1}{1\cdot 2}+\frac{1}{2\cdot 3}+\frac{1}{3\cdot 4}+\cdots +\frac{1}{k(k+1)}=\frac{k}{k+1}\]
   那么,
\[\begin{split}
    &\quad \frac{1}{1\cdot 2}+\frac{1}{2\cdot 3}+\frac{1}{3\cdot 4}+\cdots +\frac{1}{k(k+1)}+\frac{1}{(k+1)(k+2)}\\
    &=\frac{k}{k+1}+\frac{1}{(k+1)(k+2)}\\
    &=\frac{k(k+2)+1}{(k+1)(k+2)}=\frac{k^2+2k+1}{(k+1)(k+2)}\\
    &=\frac{(k+1)^2}{(k+1)(k+2)}=\frac{k+1}{(k+1)+1}
\end{split}\]
    这就是说,当$n=k+1$时等式成立。
\end{enumerate}

根据(1)和(2),等式对任何$n\in\N$都成立。
\end{proof}

\begin{example}
用数学归纳法证明$x^{2n}-y^{2n}\; (n\in\N)$能被$x+y$整除。
\end{example}

\begin{proof}
\begin{enumerate}[(1)]
    \item 当$n=1$时,$x^2-y^2=(x+y)(x-y)$能被$x+y$整除。
    \item 假设当$n=k\; (k\in\N)$时,$x^{2k}-y^{2k}$能被$x+y$整除,那么
\[\begin{split}
x^{2(k+1)}-y^{2(k+1)}&=x^2\cdot x^{2k}-y^2\cdot y^{2k}\\
&=x^2\cdot x^{2k}-x^2\cdot y^{2k}+x^2\cdot y^{2k}-y^2\cdot y^{2k}\\
&=x^2(x^{2k}-y^{2k})+y^{2k}(x^2-y^2)
\end{split}\]

因为$x^{2k}-y^{2k}$与$x^2-y^2$都能被$x+y$整除,所以它们的和$x^2(x^{2k}-y^{2k})+y^{2k}(x^2-y^2)$也能被$x+y$整除。这就是说,当$n=k+1$时,$x^{2(k+1)}-y^{2(k+1)}$能被$x+y$整除。
\end{enumerate}

根据(1)和(2),可知命题对任何$n\in\N$都成立.
\end{proof}

\begin{example}
    平面内有$n$条直线,其中任何两条不平行,任何三条不过同一点,证明交点的个数$f(n)$等于$\frac{1}{2}n(n-1)$.
\end{example}

\begin{proof}
\begin{enumerate}[(1)]
    \item 当$n=2$时,两条直线的交点只有1个,即$f(2)=1$. 又当$n=2$时,
\[\frac{1}{2}\x2\x(2-1)=1\]
因此命题成立。

\item 假设$n=k$时命题成立,就是说,平面内满足题设的任何$k$条直线的交点的个数$f(k)$等于$\frac{1}{2}k(k-1)$. 现在来考虑平面内有$k+1$条直线的情况。任取其中的1条直线,记为$\ell$(图5.9)由上述归纳法的假设,除$\ell$以外的其他$k$条直线
的交点的个数$f(k)$等于$\frac{1}{2}k(k-1)$. 另外,因为已知任何两
条直线不平行,所以直线$\ell$必与平面内其他$k$条直线都相交;又因为已知任何三条直线不过同一点,所以上面的$k$个交点两两不相同,且与平面内其他的$\frac{1}{2}k(k-1)$个交点也两两不相同,从而平面内交点的个数是

\noindent
\begin{minipage}{.47\textwidth}\CTEXindent
\[\begin{split}
    \frac{1}{2}k(k-1)+k&=\frac{1}{2}k[(k-1)+2]\\
    &=\frac{1}{2}(k+1)[(k+1)-1]
\end{split}\]

这就是说,当$n=k+1$时,$k+1$条直线的交点的个数$f(k+1)$等于$\frac{1}{2}(k+1)[(k+1)-1]$.    
\end{minipage}\hfill
\begin{minipage}{.45\textwidth}
\centering
\begin{tikzpicture}[scale=1]
\tkzDefPoints{-1/0/A_1, 1/0/A_k, -.5/0/A_2, .5/0/A_3, 0/1.8/B, 0/.4/C}
\tkzDrawLines[add=.6 and .3](A_1,B A_k,B A_1,A_k)
\tkzDrawLines[add=2.5 and 2.5](A_2,C A_3,C)
\node at (1.7,0)[right]{$\ell$};
\tkzDrawPoints(A_1,A_k,A_2,C,A_3)
\tkzInterLL(B,A_1)(C,A_2) \tkzGetPoint{D1}
\tkzInterLL(B,A_k)(C,A_3) \tkzGetPoint{D2}
\tkzInterLL(B,A_1)(C,A_3) \tkzGetPoint{D3}
\tkzInterLL(B,A_k)(C,A_2) \tkzGetPoint{D4}
\tkzLabelPoints[above](A_1,A_2,A_k)
\tkzDrawPoints(D1,D2,D3,D4)

\end{tikzpicture}
\captionof{figure}{}
\end{minipage}
\end{enumerate}

根据(1)和(2),命题对任何$n\ge 2\; (n\in\N)$都成立。
\end{proof}

\begin{example}
设$\sin\frac{\alpha}{2}\ne 0$,用数学归纳法证明
\[\sin\alpha+\sin2\alpha+\sin3\alpha+\cdots+\sin n\alpha=\frac{\sin\frac{n\alpha}{2}\sin\frac{(n+1)\alpha}{2}}{\sin\frac{\alpha}{2}}\]
\end{example}

\begin{proof}
\begin{enumerate}[(1)]
    \item 当$n=1$时,左边是$\sin\alpha$,右边是$\frac{\sin\frac{\alpha}{2}\sin\alpha}{\sin\frac{\alpha}{2}}=\sin\alpha$,等式成立.
    \item 假设当$n=k\; (k\in\N)$时等式成立,就是
\[\sin\alpha+\sin2\alpha+\sin3\alpha+\cdots+\sin k\alpha=\frac{\sin\frac{k\alpha}{2}\sin\frac{(k+1)}{2}\alpha}{\sin\frac{\alpha}{2}}\]
那么
\[\begin{split}
 &\qquad    \sin\alpha+\sin2\alpha+\sin3\alpha+\cdots+\sin k\alpha+\sin (k+1)\alpha\\
 &=\frac{\sin\frac{k\alpha}{2}\sin\frac{(k+1)}{2}\alpha}{\sin\frac{\alpha}{2}}+\sin (k+1)\alpha\\
 &=\frac{\sin\frac{k\alpha}{2}\sin\frac{(k+1)}{2}\alpha+\sin\frac{\alpha}{2}\sin (k+1)\alpha}{\sin\frac{\alpha}{2}}\\
 &=\frac{\frac{1}{2}\left(\cos\frac{\alpha}{2}-\cos\frac{2k+1}{2}\alpha+\cos\frac{2k+1}{2}\alpha-\cos\frac{2k+3}{2}\alpha\right)}{\sin\frac{\alpha}{2}}\\
 &=\frac{\cos\frac{\alpha}{2}-\cos\frac{2k+3}{2}\alpha}{2\sin\frac{\alpha}{2}}=\frac{\sin\frac{(k+1)\alpha}{2}\sin\frac{[(k+1)+1]\alpha}{2}}{\sin\frac{\alpha}{2}}
\end{split}\]
这就是说,当$n=k+1$时等式也成立.
\end{enumerate}

根据(1)和(2),等式对任何$n\in \N$都成立。
\end{proof}






\begin{example}
    若$n\in\N$,求证$f(n)=n^3+5n$能被6整除。
\end{example}

\begin{proof}
\begin{enumerate}[(1)]
    \item 当$n=1$时,$f(1)=1+5=6$命题成立。
\item 假设$n=k$时,命题成立,即
$f(k)=k^3+5k$能被6整除。

当$n=k+1$时,
\[\begin{split}
f(k+1)=(k+1)^3+5(k+1)&=k^3+3k^2+3k+1+5k+5\\
&=k^3+5k+3(k^2+k+2)=k^3+5k+3k(k+1)+6    
\end{split}\]
由归纳假设,$k^3+5k$能被6整除。

又$k(k+1)$能被2整除,从而$3k(k+1)$能被6整除,6能被6整除。

故$n=k+1$时命题也成立。
\end{enumerate}
(1)和(2)说明,对任何$n\in\N$, $f(n)=n^3+5n$能被6整除。
\end{proof}

\begin{example}
    若$n\in\N$,求证$f(n)=3^{3m+2}+5\cdot 2^{3m+1}$能被19整除。
\end{example}

\begin{proof}
\begin{enumerate}[(1)]
    \item 
当$n=1$时,$f(1)=3^{3+2}+5\cdot 2^{3+1}=323=19\x17$

$\therefore\quad $命题成立。
\item 假设$n=k$时,命题成立,即
$f(k)=3^{3k+2}+5\cdot 2^{3k+1}$能被19整除。

当$n=k+1$时,
\[\begin{split}
f(k+1)=3^{3(k+1)+2}+5\cdot 2^{3(k+1)+1}&=27\cdot 3^{3k+2}+5\cdot 8\cdot 2^{3k+1}\\
&=8(3^{3k+2}+5\cdot 2^{3k+1})+19\cdot 3^{3k+2}    
\end{split}\]

由归纳假设,$8(3^{3k+2}+5\cdot 2^{3k+1})$能被19整除。

又显然$19\cdot 3^{3k+2}$能被19整除。

即$n=k+1$时,命题也成立
\end{enumerate} 

由(1)、(2)对$n\in\N$, $f(n)=3^{3n+2}+5\cdot 2^{3n+1}$能被19整除。
\end{proof}



\begin{example}
    已知$x>-1$, 且$x\ne 0,\; n\in\N,\; n\ge 2$,求证
$(1+x)^n>1+nx$.
\end{example}

\begin{proof}
\begin{enumerate}[(1)]
    \item  当$n=2$时,
\[\begin{split}
\text{左边}&=(1+x)^2=1+2x+x^2\\
\text{右边}&=1+2x    
\end{split}\]
因为$x^2>0$. 所以原不等式成立。
\item 假设$n=k\; (k\ge 2)$时不等式成立,就是
$(1+x)^k>1+kx$.

当$n=k+1$时,因为$x>-1$,所以$1+x>0$,于是
\[\begin{split}
\text{左边}&=(1+x)^{k+1}=(1+x)(1+x)^k\\
&>(1+x)(1+kx)=1+(k+1)x+kx^2\\
\text{右边}&=1+(k+1)x    
\end{split}\]

因为$kx^2>0$,所以左边$>$右边,即
\[(1+x)^{k+1}>1+(k+1)x\]
这就是说,原不等式当$n=k+1$时也成立。
\end{enumerate}

根据(1)和(2),原不等式对任何不小于2的自然数$n$都成立。
\end{proof}



\begin{example}
若$n\in\N$,求证
\[\sqrt{1\cdot 2}+\sqrt{2\cdot 3}+\sqrt{3\cdot 4}+\cdots +\sqrt{n(n+1)}<\frac{(n+1)^2}{2}\]
\end{example}

\begin{proof}
\begin{enumerate}[(1)]
    \item 当$n=1$时,左边$=\sqrt{2}$,右边$=2$,左边$<$右边,所以原不等式成立。
    \item 假设$n=k\; (k\in\N)$时原不等式成立,即
\[\begin{split}
   \sqrt{1\cdot 2}&+\sqrt{2\cdot 3}+\sqrt{3\cdot 4}+\cdots +\sqrt{k(k+1)}+\sqrt{(k+1)(k+2)}\\
    &   <\frac{(k+1)^2}{2}+\sqrt{(k+1)(k+2)}
\end{split}\]
由于$k+1>0,\; k+2>0,\; k+1\ne k+2$,
所以
\[\sqrt{(k+1)(k+2)}<\frac{(k+1)+(k+2)}{2}=\frac{2k+3}{2}\]
因此 
\[\begin{split}
    \sqrt{1\cdot 2}+\sqrt{2\cdot 3}+\sqrt{3\cdot 4}+\cdots +\sqrt{(k+1)(k+2)}&<\frac{(k+1)^2}{2}+\frac{2k+3}{2}\\
    &=\frac{[(k+1)+1]^2}{2}
\end{split}\]
$\therefore\quad $原不等式当$n=k+1$时也成立。
\end{enumerate} 

由(1)与(2)可知,原不等式对于任意的$n\in\N$都成立。
\end{proof}






\begin{example}
     已知:$n\in\N$,求证
$|\sin n\theta|\le n|\sin\theta|$.
\end{example}

\begin{proof}
\begin{enumerate}[(1)]
    \item 当$n=1$时,
\[\text{左边}=|\sin\theta|=\text{右边}\]
所以原不等式成立。
\item 假设$n=k\; (k\in\N)$时原不等式成立,即
\[|\sin k\theta|\le k|\sin\theta|\]
当$n=k+1$时,
\[\begin{split}
|\sin(k+1)\theta|&=|\sin k\theta\cdot \cos\theta+\cos k\theta\cdot \sin\theta|\\
&\le |\sin k\theta\cdot \cos\theta| +|\cos k\theta\cdot \sin\theta|\\
&=|\sin k\theta|\cdot |\cos\theta| + |\cos k\theta|\cdot |\sin\theta|    
\end{split}\]

$\because\quad |\cos\theta|\le 1,\quad |\cos k\theta|\le 1$

$\therefore\quad |\sin(k+1)\theta|\le |\sin k\theta|+|\sin\theta|\le k|\sin\theta|+|\sin\theta|=(k+1)|\sin\theta|$.

这就是说,当$n=k+1$时原不等式成立。
\end{enumerate}

由(1)和(2),对任意的
$n\in\N$,原不等式都成立。
\end{proof}

\section*{习题五}

\begin{center}
    \bfseries A
\end{center}

\begin{enumerate}
    \item 用数学归纳法证明:
\begin{enumerate}[(1)]
    \item $1^2+3^2+5^2+\cdots+(2n-1)^2=\frac{1}{3}n(4n^2-1)$
    \item $1\cdot 4+2\cdot 7+3\cdot 10+\cdots+n(3n+1)=n(n+1)^2$
    \item $\frac{1}{2\cdot 4}+\frac{1}{4\cdot 6}+\frac{1}{6\cdot 8}+\cdots+\frac{1}{2n (2n+2)}=\frac{n}{4(n+1)}$
\end{enumerate}
    \item 用数学归纳法证明:
\begin{enumerate}[(1)]
    \item $n^3+5n\; (n\in\N)$能被6整除
    \item $3^{4n+2}+5^{2n+1}\; (n\in\N)$能被14整除
    \item 三个连续自然数的立方和能被9整除
\end{enumerate}
    \item 如果$\sin\alpha\ne 0$,用数学归纳法证明:
\begin{enumerate}[(1)]
    \item $\cos\alpha\cdot \cos2\alpha\cdot \cos2^2\alpha\cdots \cos2^{n-1}\alpha=\frac{\sin2^n\alpha}{2^n\sin\alpha}$
    \item $\cos\alpha+\cos3\alpha+\cos5\alpha+\cdots+\cos(2n-1)\alpha=\frac{\sin 2n\alpha}{2\sin\alpha}$
\end{enumerate}
    \item 证明凸$n$边形的对角线的条数 $f(n)=\frac{1}{2}n(n-3)\quad (n\ge 3)$
\end{enumerate}

\begin{center}
    \bfseries B
\end{center}

\begin{enumerate}\setcounter{enumi}{4}
    \item 求证:$$1-\frac{1}{2}+\frac{1}{3}-\frac{1}{4}+\frac{1}{5}-\cdots-\frac{1}{2n}=\frac{1}{n+1}+\frac{1}{n+2}+\frac{1}{n+3}+\cdots+\frac{1}{n+n}\; (n\in\N)$$
\item 用数学归纳法证明:当自然数$n\ge 2$时,
\[\left(1-\frac{1}{4}\right)\left(1-\frac{1}{9}\right)\left(1-\frac{1}{16}\right)\cdots \left(1-\frac{1}{n^2}\right)=\frac{n+1}{2n}\]
\item 证明:$f(n)=3^{2n+2}-8n-9\; (n\in\N)$被64整除。

若$n\in\N$,求证$f(n)=(3n+1)7^n-1$被9整除。
\item 求证:$1+\frac{1}{2^2}+\frac{1}{3^2}+\cdots +\frac{1}{n^2}<2-\frac{1}{n}$($n\in\N$且$n\ge 2$)
\item 若$n$是大于2的自然数,求证
$\left(1+\frac{1}{n}\right)^n<n$
\item 用数学归纳法证明:
\begin{enumerate}[(1)]
\item 平面上凸$n$边形的内角和$f(n)=(n-2)\pi\quad (n\ge 3)$;
\item 平面上有$n$条直线,其中任两条不平行,任三条不共点,求证这$n$条直线交点的个数$f(n)=\frac{1}{2}n(n-1)\quad (n\ge 2)$.
\end{enumerate}

\item 平面上有$n$个圆。其中,任两圆都相交于两个点,且任三个圆都不相交于同一个点,试求这$n$个圆把平面划分成多少个部分。
\item 空间有$n$个平面,其中任两个都相交(且交线彼此都不平行),任三个都不交于同一条直线,任四个都不交于同一个点,试求这$n$个平面把空间划分成多少个部分。
\end{enumerate}

\section*{六、数列的极限}
前面我们研究了数列的概念,等差数列,等比数列,以及一些特殊数列的求和问题,下面研究数列的极限问题。

\section{数列的极限}

我们考察下面两个无穷数列:
\begin{align}
&\frac{1}{2},\; \frac{2}{3},\; \frac{3}{4},\ldots, \frac{n}{n+1},\ldots \tag{1}\\
&1,\; -\frac{1}{2},\; \frac{1}{4},\; -\frac{1}{8},\ldots, \left(-\frac{1}{2}\right)^{n-1},\ldots \tag{2}
\end{align}

为直观起见,我们把这两个数列中的前几项分别在图象中表示出来(图5.10):
\begin{figure}[htp]
    \centering
\begin{tikzpicture}[xscale=.5, yscale=1.5, >=stealth]
\begin{scope}
\draw[->](-2,0)--(9,0)node[below]{$n$};
\draw[->](0,-.5)--(0,1.5)node[left]{$y$};
\draw[dashed](0,1.2)node[left]{$1+\varepsilon$}--(9,1.2);
\draw[dashed](0,.8)node[left]{$1-\varepsilon$}--(9,.8);
\draw[thick](0,1)node[left]{$1$}--(9,1);
\node[below left]{$O$};
\node at (0,1.5)[right]{$(a_n)$};
\foreach \x in{1,2,3,...,7}
{
    \draw(\x,0)node[below]{\x}--(\x,.06);
}
\tkzDefPoints{1/.5/A, 2/.667/B, 3/.75/C, 4/.8/D, 5/0.833/E, 6/.857/F, 7/.875/G}
\tkzDrawPoints(A,B,C,D,E,F,G)
\node at (4,-1){(1)};
\end{scope}
\begin{scope}[xshift=13cm]
    \draw[->](-1,0)--(9,0)node[below]{$n$};
\draw[->](0,-1)--(0,1.5)node[left]{$y$};
\node at (0,1.5)[right]{$(a_n)$};
\foreach \x in {1,2,3,5,7}
{
    \draw(\x,0)node[below]{\x}--(\x,.06);
}
\foreach \x in {4,6}
{
    \draw(\x,0)--(\x,.06)node[above]{\x};
}
\tkzDefPoints{1/1/A, 2/-.5/B, 3/.25/C, 4/-.125/D, 5/0.0625/E, 6/-.03125/F, 7/.015625/G}
\tkzDrawPoints(A,B,C,D,E,F,G)
\node[below left]{$O$};
\node at (4,-1){(2)};
\draw(0,1)node[left]{1}--(.1,1);
\draw(0,-.5)node[left]{$-\tfrac{1}{2}$}--(.1,-.5);

\end{scope}    
\end{tikzpicture}
    \caption{}
\end{figure}

容易看出,当项数$n$无限增大时,表示数列(1)的各点与纵坐标为1的直线的距离无限地变小,即数列各项的值,无限的趋近于1;表示数列(2)的各点与横轴的距离
无限地变小,即数列的各项的值无限地趋向于0。

我们可以用$\left|\frac{n}{n+1}-1\right|$表示数列(1)中的各项的
值与数1接近的程度,同样可以用
$\left|\left(-\frac{1}{2}\right)^{n-1}-0\right|$表示
数列(2)中各项的值与0接近的程度。

事实上,在数列(1)中,各项的值与1的差的绝对值可列表表示如下(表5.1):
\begin{table}[htp]
    \caption{}
    \centering\begin{tabular}{ccc}
\hline
        项号& 项 & 这一项与1的差的绝对值 \\
\hline
1   &$\frac{1}{2}$   &$\left|\frac{1}{2}-1\right|=\frac{1}{2}$\\ [1.5ex]
2   &$\frac{2}{3}$   &$\left|\frac{2}{3}-1\right|=\frac{1}{3}$\\[1.5ex]
3   &$\frac{3}{4}$   &$\left|\frac{3}{4}-1\right|=\frac{1}{4}$\\[1.5ex]
4   &$\frac{4}{5}$   &$\left|\frac{4}{5}-1\right|=\frac{1}{5}$\\[1.5ex]
5   &$\frac{5}{6}$   &$\left|\frac{5}{6}-1\right|=\frac{1}{6}$\\[1.5ex]
6   &$\frac{6}{7}$   &$\left|\frac{6}{7}-1\right|=\frac{1}{7}$\\[1.5ex]
7   &$\frac{7}{8}$   &$\left|\frac{7}{8}-1\right|=\frac{1}{8}$\\[1.5ex]
$\cdots $   &$\cdots $  &$\cdots\cdots $  \\
\hline
    \end{tabular}
\end{table}

我们看到,无论预先指定多么小的一个正数$\varepsilon$,总能在
数列(1)中找到这样一项,使得这一项后面的所有项与1的差的绝对值都小于$\varepsilon$。直观地看,对于任意预先指定的正数$\varepsilon$,可以在表示数列(1)的图象上作两条平行于直线$y=1$的直线$\ell_1:\; y=1+\varepsilon$和$\ell_2:\; y=1-\varepsilon$,它们与直线$y=1$的距离都是$\varepsilon$。同时,在数列(1)中总可以找到这样一项,使得这一项后面的所有项的对应点都落在直线$\ell_1$和$\ell_2$之间。显然当$\varepsilon$较小时,$\ell_1$与$\ell_2$间的距离较小,数列(1)的前若干项的对应点可能没有落在$\ell_1$与$\ell_2$之间,但由于随着$n$的增大,数列(1)的对应点越来越逼近直线$y=1$. 因此总能找到一项,使得这一项后面的所有项的对应点都落在$\ell_1$与$\ell_2$之间。例如取$\varepsilon=0.2$,则数列(1)中第4项后面的各项与1的差的绝对值都小于$\varepsilon$; 又如取$\varepsilon=0.01$,则数列(1)中第99项后面的所有项与1的差的绝对值都小于$\varepsilon$。在这种情况下我们说数列(1)的极限是1。

同样,对于数列(2),我们也可列出下表(表5.2)。
\begin{table}[htp]
    \caption{}
    \centering
\begin{tabular}{ccl}
    \hline
    项号& 项 & 这一项与0的差的绝对值 \\
    \hline
1   & $1$   & $\left|1-0\right|=1$\\
2   & $-\frac{1}{2}$   & $\left|-\frac{1}{2}-0\right|=0.5$\\ [1.5ex]
3   &$\frac{1}{4}$   & $\left|\frac{1}{4}-0\right|=0.25$\\ [1.5ex]
4   &$-\frac{1}{8}$   & $\left|-\frac{1}{8}-0\right|=0.125$\\ [1.5ex]
5   &$\frac{1}{16}$   & $\left|\frac{1}{16}-0\right|=0.0625$\\ [1.5ex]
6   &$-\frac{1}{32}$   & $\left|-\frac{1}{32}-0\right|=0.03125$\\ [1.5ex]
7   &$\frac{1}{64}$   & $\left|\frac{1}{64}-0\right|=0.015625$\\ [1.5ex]
8   &$-\frac{1}{128}$   & $\left|-\frac{1}{128}-0\right|=0.0078125$\\ [1.5ex]
$\cdots$   & $\cdots$   & $\cdots\cdots$   \\ 
\hline
\end{tabular}
\end{table}


可以看出,如果取$\varepsilon=0.1$,那么数列(2)中第4项后面的所有项与0的差的绝对值都小于$\varepsilon$;如果取$\varepsilon=0.01$,那么第7项后面的所有项与0的差的绝对值都小于$\varepsilon$。就是说,
无论预先指定多么小的一个正数$\varepsilon$,总能在数列(2)中找到这样一项,使得这一项后面的所有项与0的差的绝对值都小于$\varepsilon$。这时,我们说数列(2)的极限是0。

一般地,对于无穷数列$\{a_n\}$,如果存在一个常数$A$,无论预先指定多么小的正数$\varepsilon$,都能在数列中找到一项$a_N$,使得这一项后面的所有项与$A$的差的绝对值都小于$\varepsilon$(即当$n>N$时,$|a_n-A|<\varepsilon$恒成立),就把常数$A$叫做\textbf{数列$\{a_n\}$的极限}\footnote{lim的英文limit(极限)的头三个字母。},记作
\[\lim_{n\to\infty}a_n=A\]
这个式子读作“当$n$趋向于无穷大时,$a_n$的极限等于$A$”。“$\to $”表示“趋向于”,“$\infty$”表示“无穷大”,“$n\to\infty$”表示“$n$趋向于无穷大”,也就是$n$无限增大的意思。

$\Lim{n}{\infty}a_n=A$有时也可记作
\[\text{当$n\to\infty$时,}a_n\to A\]

从数列极限的定义可以看出,数列$\{a_n\}$以$A$为极限,是指当$n$无限增大时,数列$\{a_n\}$中的项$a_n$无限趋近于常数$A$,也
就是$a_n$与$A$的差的绝对值无限趋近于零.

\begin{example}
已知数列$1,\; -\frac{1}{2},\; \frac{1}{3},\; -\frac{1}{4},\; \ldots, (-1)^{n+1}\frac{1}{n},\ldots$
\begin{enumerate}[(1)]
\item 写出这个数列的各项与0的差的绝对值;
\item 第几项后面的所有项与0的差的绝对值都小于0.1?都小于0.001?都小于0.0003?
\item 第几项后面的所有项与0的差的绝对值都小于任何预先指定的正数$\varepsilon$?
\item 0是不是这个数列的极限?
\end{enumerate}
\end{example}

\begin{solution}
\begin{enumerate}[(1)]
    \item 这个数列的各项与0的差的绝对值依次是
\[1,\; \frac{1}{2},\; \frac{1}{3},\ldots,\frac{1}{n},\ldots\]
    \item 要使$\frac{1}{n}<0.1$,只要$n>10$就行了。这就是说,第10项后面的所有项与0的差的绝对值都小于0.1。
    
    要使$\frac{1}{n}<0.001$,只要$n>1000$就行了,这就是说,第1000项后面的所有项与0的差的绝对值都小于0.001。

    要使$\frac{1}{n}<0.0003$只要$n>3333\frac{1}{3}$就行了,这就是说,第3333项后面的所有项与0的差的绝对值都小于0.0003。
    \item 要使$\frac{1}{n}<\varepsilon$,只要$n>\frac{1}{\varepsilon}$就行了,设$\frac{1}{\varepsilon}$的整数部分是$N$,记$\left[\frac{1}{\varepsilon}\right]=N$\footnote{$[x]$表示不超过实数$x$的最大整数,称$[x]$为高斯符号.如$[5.6]=5$,$[0.21]=0$, $[\pi]=3$, $[-2.13]=-3$.},那么$N$项后面的所有项与0的
    差的绝对值都小于$\varepsilon$。
\item 从第(3)小题可以知道,0是这个数列的极限,记作:
\[\lim_{n\to\infty}(-1)^{n+1}\frac{1}{n}=0\]
\end{enumerate}
\end{solution}    







\begin{example}
    已知数列$\{a_n\}$
\[1,\; \frac{4}{3},\; \frac{6}{4},\; \frac{8}{5},\ldots,\frac{2n}{n+1},\ldots\]
求证此数列以2为极限。
\end{example}

\begin{proof}
对于任何预先指定的正数$\varepsilon$,欲使
$|a_n-2|<\varepsilon$
对于$a_N$后面的所有项都成立,只需
\[\left|\frac{2n}{n+1}-2\right|<\varepsilon\]
只需
\[\frac{2}{n+1}<\varepsilon\]
只需
\[n>\frac{2}{\varepsilon}-1\]

设$\left[\frac{2}{\varepsilon}-1\right]=N$,那么,当$n>N$时,
$|a_n-2|<\varepsilon$
恒成立,所以数列$\{a_n\}$以2为极限。
\end{proof}



\begin{example}
    求常数数列$-7,\; -7,\; -7,\;\ldots$的极限。
\end{example}

\begin{solution}
这个数列的各项与$-7$的差的绝对值都等于0,所以
从第1项起,这个绝对值就能够小于任意指定的正数$\varepsilon$。因此这个数列的极限是$-7$。

一般地,任何一个常数数列的极限都是这个常数本身,即
\[\lim_{n\to\infty}C=C\qquad \text{($C$是常数)}\]

应该指出,并不是每一个无穷数列都有极限。例如数列
$1,2,3,\ldots,n,\ldots$
与
$-1,1,-1,1,\ldots,(-1)^n,\ldots$
都没有极限。
\end{solution}


\begin{example}
    已知数列$\{a_n\}$以$A$为极限,试判断下列命题是否正确:
\begin{enumerate}[(1)]
\item 数列$\{a_n-A\}$必是递减数列:
\item 数列$\{|a_n-A|\}$必是递减数列或常数列;
\item 对于任意自然数$m$,$n$,若$m>n$,则$|a_m-A|\le |a_n-A|$;
\item 对于任何预先指定的正数$\varepsilon$,只有有限个自然数$n$,使得$|a_n-A|\ge \varepsilon$.
\end{enumerate}
\end{example}

\begin{solution}
\begin{enumerate}[(1)]
    \item 不正确。如数列$\{a_n\}:\; 1,-\frac{1}{2},\frac{1}{3},\ldots,(-1)^{n+1}\frac{1}{n},\ldots$以0为极限,但数列$\{|a_n-0|\}$是摆动数列。
    \item 不正确。如数列$\{a_n\}:\; \frac{1}{2},-\frac{1}{4},-\frac{1}{3},\ldots,\frac{1}{n}\cos\frac{n\pi}{3},\ldots$以0为极限,但数列$\{|a_n-0|\}$是摆动数列.
    \item 不正确。如数列$\left\{\frac{1}{n}\cos\frac{n\pi}{3}\right\}$不符合这一命题。
    \item 正确。因为数列$\{a_n\}$以$A$为极限,也就是说,对于任何预先指定的正数$\varepsilon$,都能找到自然数$N$,当$n>N$时,$|a_n-A|<\varepsilon$恒成立。$a_N$及$a_N$前面的项只有有限多个,在这有限多项中可能全体或部分适合$|a_n-A|\ge \varepsilon$.
\end{enumerate}    
\end{solution}

\begin{ex}
\begin{enumerate}
    \item 已知数列$\frac{2}{1},\; \frac{3}{2},\; \frac{4}{3}, \ldots,\frac{n+1}{n},\ldots$
\begin{enumerate}[(1)]
\item 把这个数列的前5项在坐标平面上表示出来;
\item 写出这个数列的各项与1的差的绝对值;
\item 第几项后面的所有项与1的差的绝对值都小于0.1?都小于0.01?都小于0.0001?都小于任何预先指定的正数$\varepsilon$?
\item 1是不是这个无穷数列的极限?
\end{enumerate}
    \item 举出两个极限是5的无穷数列的例子。
    \item 证明数列$4-\frac{1}{10},\; 4-\frac{1}{20},\ldots, 4-\frac{1}{10n},\ldots$以4为极限。
\end{enumerate}
\end{ex}


\section{数列极限的运算法则}
上面我们介绍了数列极限的定义,并根据这个定义可以确定以下几个最简单的数列的极限:
\begin{enumerate}[(1)]
    \item $\Lim{n}{\infty}C=C$($C$是常数)
    \item $\Lim{n}{\infty} \frac{1}{n}=0 $
    \item $\Lim{n}{\infty} q^n=0\quad (|q|<1) $
\end{enumerate}

现在证明(3)当$|q|<1$时,$\Lim{n}{\infty} q^n=0$.

\begin{proof}
\begin{enumerate}[(1)]
    \item 当$q=0$时,$\{q^n\}$的每一项都是0,由数列极限的定义易知$\{q^n\}$有极限0.即$\Lim{n}{\infty} q^n=0$
    \item 若$q\ne 0$,对任意给定的正数$\varepsilon<1$\footnote{取$0<\varepsilon<1$,可证$\frac{\lg \varepsilon}{\lg|q|}$是正数.},要使$|q^n-0|<\varepsilon$,即$|q|^n<\varepsilon$,通过两边取对数可知,只要$n\lg |q|<\lg \varepsilon$

$\because\quad \lg|q|<0$

$\therefore\quad $只要$n>\frac{\lg\varepsilon}{\lg|q|}$,由此可取$N=\left[\frac{\lg\varepsilon}{\lg|q|}\right]$

$\therefore\quad $对任意给定的正数$\varepsilon\; (0<\varepsilon<1)$取$N=\left[\frac{\lg\varepsilon}{\lg|q|}\right]$,则当$n>N$时,就有$|q^n-0|<\varepsilon$. 根据数列极限的定义,得到
\[\Lim{n}{\infty} q^n=0\]
\end{enumerate}

综合(1)、(2)对任意的$|q|<1$都有$\Lim{n}{\infty} q^n=0$.
\end{proof}

然而,通常求极限的问题比较复杂,这就需要分析已知数列是由哪些简单数列经过怎样的运算结合而成的,这样就能把复杂的数列极限的计算问题简化为简单的数列极限的计算问题,为此,下面引入数列极限的四则运算法则(证明从略):

\begin{thm}{}
如果$\Lim{n}{\infty} a_n=A$,$\Lim{n}{\infty} b_n=B$,那么
\[\begin{split}
    \Lim{n}{\infty} (a_n\pm b_n)&=A\pm B\\
    \Lim{n}{\infty} (a_n\cdot b_n)&=A\cdot B\\
    \Lim{n}{\infty} \frac{a_n}{b_n}&=\frac{A}{B}\quad (B\ne 0)
\end{split}\]
\end{thm}

特别地,如果$c$是常数,那么,
\[\Lim{n}{\infty}(c\cdot a_n)=\Lim{n}{\infty} c\cdot \Lim{n}{\infty}a_n=cA\]

上面的数列极限的运算法则表明,如果两个数列都有极限,那么,这两个数列的各对应项的和、差、积、商组成的数列有极限,并且分别等于这两个数列的极限的和、差、积、商(作为除数的数列的各项及其极限值不能为零)。

例如,数列$\frac{1}{2},\; \frac{2}{3},\; \frac{3}{4},\ldots, \frac{n}{n+1},\ldots$
与
$2,\;2,\;2,\ldots,2,\ldots$
的极限分别是1与2,那么根据上面的运算法则,这两个数列的各对应项的和组成的数列
$$2+\frac{1}{2},\; 2+\frac{2}{3},\; 2+\frac{3}{4},\ldots,2+\frac{n}{n+1},\ldots$$的极限是3.

上面的数列极限的运算法则,也可将两个数列推广到三个、四个、……有限个数列,得出相应的结论。特别要注意,它不能推广到无限个数列。如,以下的运算是错误的:
\[\begin{split}
   &\qquad  \Lim{n}{\infty} \left(\frac{1}{n^2}+\frac{2}{n^2}+\frac{3}{n^2}+\cdots+\frac{n}{n^2}\right)\\
&=\Lim{n}{\infty} \frac{1}{n^2}+\Lim{n}{\infty} \frac{2}{n^2}+\Lim{n}{\infty} \frac{3}{n^2}+\cdots+\Lim{n}{\infty} \frac{n}{n^2}+\cdots\\
&= 0+0+0+\cdots+0+\cdots =0
\end{split}\]
正确解法是:
\[\begin{split}
    \Lim{n}{\infty} \left(\frac{1}{n^2}+\frac{2}{n^2}+\frac{3}{n^2}+\cdots+\frac{n}{n^2}\right)&= \Lim{n}{\infty} \frac{1+2+3+\cdots+n}{n^2}\\
    &= \Lim{n}{\infty} \frac{n(n+1)}{2n^2}= \Lim{n}{\infty} \frac{1+\frac{1}{n}}{2}\\
    &=\frac{ \Lim{n}{\infty} 1+ \Lim{n}{\infty} \frac{1}{n}}{ \Lim{n}{\infty} 2}=\frac{1+0}{2}=\frac{1}{2}
\end{split}\]


\begin{example}
已知$ \Lim{n}{\infty} a_n=5$,$ \Lim{n}{\infty} b_n=3$,求$ \Lim{n}{\infty} (3a_n-4b_n)$
\end{example}

\begin{solution}
$ \Lim{n}{\infty} (3a_n-4b_n)= \Lim{n}{\infty} 3a_n- \Lim{n}{\infty} 4b_n=2 \Lim{n}{\infty} a_n-4 \Lim{n}{\infty} b_n=3$
\end{solution}

\begin{example}
求:
\begin{multicols}{2}
\begin{enumerate}[(1)]
    \item $ \Lim{n}{\infty} \left(5+\frac{1}{n}\right)  $
    \item $ \Lim{n}{\infty} \frac{2n^2+1}{3n^2+2n}  $
    \item $ \Lim{n}{\infty} \frac{\frac{1}{n}-\frac{2}{n^2}}{-\frac{3}{n}}  $
    \item $ \Lim{n}{\infty} \left(\frac{1}{2n^2}+\frac{3}{2n^2}+\cdots +\frac{2n-1}{2n^2}\right)  $
\end{enumerate}
\end{multicols}
\end{example}

\begin{solution}
\begin{enumerate}[(1)]
    \item $\Lim{n}{\infty} \left(5+\frac{1}{n}\right) =\Lim{n}{\infty} 5+\Lim{n}{\infty} \frac{1}{n}=5+0=5 $
\item 当$n$无限增大时,分式$\frac{2n^2+1}{3n^2+2n}$中的分母、分子同时无限增大,上面的极限运算法则不能直接运用,为此,我们将分式中的分子、分母同时除以$n^2$后再求它的极限,得
\[\begin{split}
    \Lim{n}{\infty} \frac{2n^2+1}{3n^2+2n}  =\Lim{n}{\infty} \frac{2+\frac{1}{n^2}}{3+\frac{2}{n}}&=\frac{\Lim{n}{\infty} \left(2+\frac{1}{n^2}\right)}{\Lim{n}{\infty} \left(3+\frac{2}{n}\right)} \\
    &=\frac{\Lim{n}{\infty} 2+\Lim{n}{\infty} \frac{1}{n^2}}{\Lim{n}{\infty} 3+\Lim{n}{\infty} \frac{2}{n}}=\frac{2+0}{3+0}=\frac{2}{3} 
\end{split}\]
\item $ \Lim{n}{\infty} \frac{\frac{1}{n}-\frac{2}{n^2}}{-\frac{3}{n}}= \Lim{n}{\infty} \frac{1+\frac{2}{n}}{-3}=\frac{ \Lim{n}{\infty} 1+ \Lim{n}{\infty} \frac{2}{n}}{ \Lim{n}{\infty} (-3)}=-\frac{1}{3}$
\item \[\begin{split}
    \Lim{n}{\infty}  \left(\frac{1}{2n^2}+\frac{3}{2n^2}+\cdots +\frac{2n-1}{2n^2}\right) &= \Lim{n}{\infty} \frac{1+3+\cdots+(2n+1)}{2n^2}\\
    &= \Lim{n}{\infty} \frac{(n+1)^2}{2n^2}= \Lim{n}{\infty} \frac{1+\frac{2}{n}+\frac{1}{n^2}}{2}\\
    &=\frac{ \Lim{n}{\infty} 1+ \Lim{n}{\infty} \frac{2}{n}+ \Lim{n}{\infty} \frac{1}{n^2}}{ \Lim{n}{\infty} 2}\\
    &=\frac{1+0+0}{2}=\frac{1}{2}
\end{split}\]
\end{enumerate}    
\end{solution}


\begin{example}
    已知等比数列$$\frac{1}{2},\;\frac{1}{4},\;\frac{1}{8},\ldots,\frac{1}{2^n},\ldots$$
    求这个数列前$n$项的和当$n\to\infty$时的极限。
\end{example}

\begin{solution}
    这个数列的公比是$\frac{\frac{1}{4}}{\frac{1}{2}}=\frac{1}{2}$. 根据等比数列前$n$项
    和的公式,得
\[S_n=\frac{\frac{1}{2}\left[1-\left(\frac{1}{2}\right)^n\right]}{1-\frac{1}{2}}=1-\frac{1}{2^n}\]
因此
\[\Lim{n}{\infty}S_n=\Lim{n}{\infty}\left(1-\frac{1}{2^n}\right)=1-\Lim{n}{\infty}\frac{1}{2^n}=1\]

上述结果可从图5.11中看出,图5.11中各小矩形与小正方形面积的和的极限等于大正方形的面积。
\end{solution}

\begin{figure}[htp]
    \centering
\begin{tikzpicture}[very thick]
\draw(0,0) rectangle (6,6);
\draw(0,3)--(6,3);
\draw(3,0)--(3,3);
\draw(3,1.5)--(6,1.5);
\node at (1.5,1.5){$\frac{1}{4}$};
\node at (3,4.5){$\frac{1}{2}$};
\node at (4.5,4.5/2){$\frac{1}{8}$};
\node at (3+.75,1.5/2){$\frac{1}{16}$};
\node at (3+1.5+.75,1.5*.75){$\tfrac{1}{32}$};
\node at (3+1.5+.375,.5*.75){$\tfrac{1}{64}$};
\draw(3+1.5,0)--(3+1.5,1.5);
\draw(3+1.5,0.75)--(6,.75);
\draw(3+1.5+.75,0)--(3+1.5+.75,.75);
\draw(3+1.5+.75,0.75/2)--(6,.75/2);
\draw(3+1.5+.75+.375,0)--(3+1.5+.75+.375,.375);
\draw(3+1.5+.75+.375,0.375/2)--(6,.375/2);
\end{tikzpicture}
    \caption{}
\end{figure}



例5.65中的无穷等比数列有这样的特点:它的公比绝对值小于1。

一般地,设无穷等比数列
$a_1,\; a_1q,\; a_1q^2,\ldots,a_1q^{n-1},\ldots$
的公比$q$的绝对值小于1,我们来求它的前$n$项的和当$n$无限增大时的极限。

无穷等比数列前$n$项的和是
\[S_n=\frac{a_1(1-q^n)}{1-q}\]
因此,
\[\begin{split}
    \Lim{n}{\infty}S_n=\Lim{n}{\infty}\frac{a_1(1-q^n)}{1-q}&=\Lim{n}{\infty}\frac{a_1}{1-q}\Lim{n}{\infty}(1-q^n)\\
    &=\frac{a_1}{1-q}\left(1-\Lim{n}{\infty}q^n\right)
\end{split}\]
因为当$|q|<1$时,$\Lim{n}{\infty} q^n=0$,所以,
\[\Lim{n}{\infty}S_n=\frac{a_1}{1-q}\cdot (1-0)=\frac{a_1}{1-q}\]

公比的绝对值小于1的无穷等比数列前$n$项的和,当$n$无限增大时的极限,叫做这个无穷等比数列\textbf{各项的和}(注意:这与有限个数的和从意义上说是不一样的),并且用符号$S$表示。从上面的推导过程可以知道,
\[S=a_1+a_1q+a_1q^2+\cdots +a_1q^{n-1}+\cdots=\frac{a_1}{1-q}\quad (|q|<1)\]
例4 
例5 



\begin{example}
    求无穷等比数列$0.3,\; 0.03,\; 0.003,\ldots$各项的和。
\end{example}

\begin{solution}
$\because\quad a_1=0.3,\quad q=0.1$

$\therefore\quad S=\frac{0.3}{1-0.1}=\frac{1}{3}$        
\end{solution}

\begin{example}
    已知无穷等比数列$\{a_n\}$各奇数项的和为2,各偶数项的和为1,求此数列。
\end{example}

\begin{solution}
设$\{a_n\}$的公比为$q$,那么$\{a_n\}$的奇数项按原次序排成首项为$a_1$,公比为$q^2$的无穷等比数列,$\{a_n\}$的偶 数项按原次序排成首项是$a_1q$,公比为$q^2$的无穷等比数列。由已知条件,
得
\[\frac{a_1}{1-q^2}=2,\qquad \frac{a_1 q}{1-q^2}=1\]
解得    
\[a_1=\frac{3}{2},\qquad q=\frac{1}{2}\]
因此,数列$\{a_n\}$是
$\frac{3}{2},\; \frac{3}{4},\; \frac{3}{8},\ldots, 3\left(\frac{1}{2}\right)^n,\ldots$
\end{solution}



\begin{example}
    将下列循环小数化成分数:
\begin{multicols}{2}
\begin{enumerate}[(1)]
    \item $0.\dot{7}$
    \item $0.2\dot{3}\dot{1}$
\end{enumerate}
\end{multicols}
\end{example}

\begin{solution}
\begin{enumerate}[(1)]
\item 纯循环小数$0.\dot{7}=0.777\cdots$可以写成
\[\frac{7}{10}+\frac{7}{100}+\frac{7}{1000}+\cdots\]
这里各项组成公比等于$\frac{1}{10}$的无穷等比数列,因此,
\[0.\dot{7}=\frac{\frac{7}{10}}{1-\frac{1}{10}}=\frac{7}{9}\]
\item 混循环小数$0.2\dot{3}\dot{1}=0.2313131\cdots$可以写成
\[\frac{2}{10}+\frac{31}{1000}+\frac{31}{100000}+\frac{31}{10000000}+\cdots\]
这里从第2项起各项组成公比等于$\frac{1}{100}$的无穷等比数列,因
此,
\[0.2\dot{3}\dot{1}=\frac{2}{10}+\frac{\frac{31}{1000}}{1-\frac{1}{100}}=\frac{2}{10}+\frac{31}{990}=\frac{229}{990}\]
\end{enumerate}
\end{solution}

\begin{ex}
\begin{enumerate}
    \item 求下列极限:
\begin{multicols}{2}
    \begin{enumerate}[(1)]
        \item $\Lim{n}{\infty} \left(3-\frac{1}{n}\right)$
        \item $\Lim{n}{\infty}\frac{2n-1}{n} $
        \item $\Lim{n}{\infty}\frac{3n+5}{n^2-10} $
        \item $\Lim{n}{\infty}\frac{1-2^n}{3^n+1} $
    \end{enumerate}
\end{multicols}
    \item 求下列无穷等比数列各项的和.
\begin{multicols}{2}
    \begin{enumerate}[(1)]
        \item $3,\; 1,\; \frac{1}{3},\; \frac{1}{9},\ldots$
        \item $1,\; -\frac{1}{2},\; \frac{1}{4},\; -\frac{1}{8},\ldots$
    \end{enumerate}
\end{multicols}
    \item 将下列循环小数化成分数.
\begin{multicols}{3}
\begin{enumerate}[(1)]
    \item $0.\dot{2}\dot{1}$
    \item $1.2\dot{3}$
    \item $0.1\dot{2}3\dot{4}$
\end{enumerate}
\end{multicols}
\end{enumerate}
\end{ex}

\section*{习题六}
\begin{center}
    \bfseries A
\end{center}

\begin{enumerate}
    \item 已知无穷数列$5+1,\;5-\frac{1}{2},\;5+\frac{1}{3},\; 5-\frac{1}{4},\ldots$
\begin{enumerate}[(1)]
\item 把这个数列的前8项在坐标平面上表示出来;
\item 计算这个数列的第$n$项与5的差的绝对值$|a_n-5|$;
\item 对于预先指定的正数$0.2,\; 0.05,\; 0.001,\;\varepsilon$,找出相应的自然数$N$,使得$n>N$时,$|a_n-5|$分别小于这些指定正数;
\item 确定这个数列的极限。
\end{enumerate}

    \item 判断下列无穷数列是否有极限,若有,求出极限值。
\begin{enumerate}[(1)]
    \item $-2,\; 0,\; 2,\; 0,\ldots,(-1)^n-1,\ldots$;
    \item $1,\; 0,\; -\frac{1}{3},\; 0,\; \frac{1}{5},\; 0,\ldots,\frac{1}{n}\sin\frac{n\pi}{2},\ldots$    
        \item $\frac{5}{3},\; \frac{9}{6},\;\frac{13}{9},\ldots, \frac{2n+3}{3n},\ldots$
        \item $\frac{1}{2},\; \frac{4}{3},\; \frac{9}{4},\ldots,\frac{n^2}{n+1},\ldots$
\end{enumerate}

\item 求下列极限:
\begin{multicols}{2}
\begin{enumerate}[(1)]
    \item $\Lim{n}{\infty} \left(\frac{1}{n^2}+\frac{2}{n}-3\right) $
    \item $\Lim{n}{\infty} \left(\frac{1}{n}-\frac{2n-3}{3n}\right) $
    \item $\Lim{n}{\infty} \frac{5-3n}{7n+4} $
    \item $\Lim{n}{\infty} \frac{2n^2+n-3}{3n^2+n-2} $
    \item $\Lim{n}{\infty} \frac{n+9}{n^2-1} $
    \item $\Lim{n}{\infty} \frac{n^2-2n+1}{n^2+2n+1} $
    \item $\Lim{n}{\infty} \frac{1+2+\cdots+n}{n^2} $
    \item $\Lim{n}{\infty} \frac{1^2+2^2+\cdots+n^2}{n^3} $
\end{enumerate}
\end{multicols}


\item 求下列极限:
\begin{enumerate}[(1)]
    \item $\Lim{n}{\infty}\left[\frac{1}{3}-\frac{1}{9}+\frac{1}{27}+\cdots+(-1)^{n-1}\frac{1}{3^n}\right]  $
    \item $\Lim{n}{\infty} \left(\frac{1}{n^2}+\frac{2}{n^2}+\frac{3}{n^2}+\cdots+\frac{2n}{n^2}\right) $
    \item $\Lim{n}{\infty} \left[\frac{1}{1\cdot 2}+\frac{1}{2\cdot 3}+\frac{1}{3\cdot 4}+\cdots+\frac{1}{n(n+1)}\right] $
    \item $\Lim{n}{\infty} \frac{1-2^n}{3^n+1} $
    \item $\Lim{n}{\infty} \frac{(n^2+1)+(n^2+2)+\cdots+(n^2+n)}{n(n-1)(n-2)} $
\end{enumerate}

\item 求下列无穷等比数列各项的和
\begin{multicols}{2}
\begin{enumerate}[(1)]
    \item $\frac{8}{9},\; -\frac{3}{2},\; \frac{1}{2},\; -\frac{3}{8},\ldots$
    \item $6\frac{2}{3},\; 1\frac{1}{3},\; \frac{4}{15},\; \frac{4}{75},\ldots$
    \item $\frac{\sqrt{3}+1}{\sqrt{3}-1},\; 1,\; \frac{\sqrt{3}-1}{\sqrt{3}+1},\ldots$
    \item $1,\; -x,\; x^2,\; -x^3,\ldots\quad (|x|<1)$
\end{enumerate}
\end{multicols}
\item 如图,三角形的一条底边是$a$,这条边上的高是$h$。
\begin{enumerate}[(1)]
\item 过高的5等分点分别作底边的平行线,并作出相应的4个矩形,求这些矩形的面积和。
\item 把高$n$等分,同样作$n-1$个矩形,求这些矩形面积的和。
\item 求证:当$n$无限增大时,这些矩形面积的和的极限等于三角形面积$\frac{ah}{2}$.
\end{enumerate}

\begin{center}
\begin{tikzpicture}
\begin{scope}
\tkzDefPoints{-2/0/A, 2/0/B, .5/3/C}
\foreach \x in {1,2,3,4}
{
    \tkzDefPointWith[linear, K=.2*\x](A,C)  \tkzGetPoint{A\x}
    \tkzDefPointWith[linear, K=.2*\x](B,C)  \tkzGetPoint{B\x}
}

\tkzDrawSegments(A1,B1 A2,B2 A3,B3 A4,B4)
\tkzDrawPolygon(A,B,C)

\tkzDefPointBy[projection = onto A--B](A1)  \tkzGetPoint{D1}
\tkzDefPointBy[projection = onto A--B](B1)  \tkzGetPoint{D2}
\tkzDrawSegments(A1,D1 B1,D2)

\tkzDefPointBy[projection = onto A1--B1](A2)  \tkzGetPoint{D11}
\tkzDefPointBy[projection = onto A1--B1](B2)  \tkzGetPoint{D21}
\tkzDrawSegments(A2,D11 B2,D21)

\tkzDefPointBy[projection = onto A2--B2](A3)  \tkzGetPoint{D12}
\tkzDefPointBy[projection = onto A2--B2](B3)  \tkzGetPoint{D22}
\tkzDrawSegments(A3,D12 B3,D22)

\tkzDefPointBy[projection = onto A3--B3](A4)  \tkzGetPoint{D13}
\tkzDefPointBy[projection = onto A3--B3](B4)  \tkzGetPoint{D23}
\tkzDrawSegments(A4,D13 B4,D23)
\draw(C)--node[left]{$h$}(.5,0)node[below left]{$a$};
\node at (0,-1){(第6题)};

\end{scope}
\begin{scope}[xshift=6cm, yshift=1.5cm]
\draw(0,0) circle (1.5);
\foreach \x in {0,1,2,...,5}
{
    \tkzDefPoint(60*\x:1.5){A\x}

}
\tkzDrawPolygon[thick](A0,A1,A2,A3,A4,A5)
\draw(0,0)--node[below right]{$R$}(60:1.5);
\draw(0,0)--node[left]{$r_n$}(0,1.3);
\node at (0,-1-1.5){(第7题)};
\end{scope}
\end{tikzpicture}
\end{center}


\item 
\begin{enumerate}[(1)]
\item 如图,在圆内接正$n$边形中,$r_n$是边心距,$p_n$是周长,$S_n$是面积,求证$S_n=\frac{1}{2}p_n\cdot r_n$.
\item 当圆内接正多形的边数无限增加时,$r_n$的极限是圆的半径$R$,$p_n$的极限是圆的周长$2\pi R$,$S_n$的极限是圆面积,求证圆面积等于$\pi R^2$.
\end{enumerate}

\item 在边长为$a$的等边三角形中,连接各边中点,作一内接正
三角形,在这个三角形中,再
用同样的方法作新的内接正三
角形;这样无限地继续下去。
\begin{enumerate}[(1)]
 \item 求所有这些三角形周长的和;
\item 求所有这些三角形面积的和。
\end{enumerate}

\item 在半径为$R$的圆里,作一内接正方形,在这个正方形里作内切圆,再在这个圆里作内接正方形,这样无限继续下去。
\begin{enumerate}[(1)]
\item 求所有这些圆的面积的和;
\item 求所有这些正方形面积的和。
\end{enumerate}

\begin{center}
    \begin{tikzpicture}
    \begin{scope}
    \tkzDefPoints{-1.5/0/A, 1.5/0/B, 0/2.6/C}
    \tkzDefMidPoint(A,B) \tkzGetPoint{D}
    \tkzDefMidPoint(A,C) \tkzGetPoint{E}
    \tkzDefMidPoint(C,B) \tkzGetPoint{F}
    \tkzDrawPolygon(A,B,C)
    \tkzDrawPolygon(D,E,F)
    \tkzDefMidPoint(D,E) \tkzGetPoint{D1}
    \tkzDefMidPoint(D,F) \tkzGetPoint{E1}
    \tkzDefMidPoint(E,F) \tkzGetPoint{F1}
    \tkzDrawPolygon(D1,E1,F1)
    \node at(0,0)[below]{$a$};
    \node at (0,-1){(第8题)};
    
    \end{scope}    
    \begin{scope}[xshift=5cm, yshift=1.5cm]
\draw(0,0)circle (1.5);
\tkzDefPoint(45:1.5){C}    
\tkzDefPoint(45+90:1.5){D}    
\tkzDefPoint(45+180:1.5){A}    
\tkzDefPoint(45+270:1.5){B}    
\tkzDrawPolygon(A,B,C,D)
\draw(0,0)circle (1.5/1.414);
\foreach \x in {0,1,2,3}
{
    \tkzDefPoint(90*\x:1.5/1.414){E\x}
}
\tkzDefPoints{0/0/O}
\tkzDrawPolygon(E0,E1,E2,E3)
\draw(0,0)node{$O$}circle (1.5/2);
\tkzDefPoint(45:1.5/2){C1}    
\tkzDefPoint(45+90:1.5/2){D1}    
\tkzDefPoint(45+180:1.5/2){A1}    
\tkzDefPoint(45+270:1.5/2){B1}  
\tkzDrawPolygon(A1,B1,C1,D1)
\tkzAutoLabelPoints[center=O](A,B,C,D)
\node at (0,-2.5){(第9题)};
    \end{scope}    
    \end{tikzpicture}
    \end{center}

\noindent
\begin{minipage}{.52\textwidth}
\item 如图:第一个半圆的直径是2.5cm,第二个半圆的直径是2cm,以后每个半圆的直径都是前一个的$\frac{4}{5}$,这样无限继续下去,求整条曲线的长。    
\end{minipage}\hfill
\begin{minipage}{.45\textwidth}
\centering
\begin{tikzpicture}
\draw[dashed](-1,0)--(3,0);
\draw[very thick](2,0) arc (0:-180:1.25);
\draw[very thick](2-2.5,0) arc (180:0:1);
\draw[very thick](-.5+2,0) arc (0:-180:.8);
\draw[very thick](1.5-1.6,0) arc (180:0:.64);
\draw[very thick](-.1+1.28,0) arc (0:-180:.64*.8);
\draw[very thick](1.18-.64*1.6,0) arc (180:0:.64*.64);
\node at (1,-1.6){(第10题)};
\end{tikzpicture}
\end{minipage}

\item 将下列循环小数化成分数:
\begin{multicols}{4}
\begin{enumerate}[(1)]
    \item $0.\dot{4}$
    \item $0.\dot{1}3\dot{5}$
    \item $0.4\dot{3}\dot{6}$
    \item $2.13\dot{8}$
\end{enumerate}
\end{multicols}



\end{enumerate}

\begin{center}
    \bfseries B
\end{center}

\begin{enumerate}\setcounter{enumi}{11}
    \item 已知$\Lim{n}{\infty}a_n=A$,$\Lim{n}{\infty}b_n=B$,求证
$\Lim{n}{\infty}(a_n+b_n)=A+B$.

\item 下列命题分别是“数列$\{a_n\}$有极限$A$”的什么条件(充分、必要、充要)?
\begin{enumerate}[(1)]
\item 对于任意预先给定的正数$\varepsilon$,总能找到自然数$n$,使得$|a_n-A|<\varepsilon$成立;
\item 对于某一个非常小的正数$\varepsilon$,能找到自然数$N$使$n>N$时,$|a_n-A|<\varepsilon$成立;
\item 对于任意预先给定的正数$\varepsilon$,$|a_n-A|<\varepsilon$对于任意自然数$n$都成立;
\item 对于任意预先给定的正数$\varepsilon$,存在无穷多个自然数$n$。使$|a_n-A|<\varepsilon$成立;
\item 对于任意预先给定的正数$\varepsilon$,总能找到自然数$N$,使$n>N$时,$A-\varepsilon<a_n<A+\varepsilon$恒成立.
\end{enumerate}

\item \begin{enumerate}[(1)]
    \item 已知$\Lim{n}{\infty}\frac{an^3+10n^2}{n^3-an^2-10}=\frac{1}{2}$,求常数$a$的值。
    \item   已知当$n\to\infty$时,$\sqrt{3n^2+1}-kn$存在极限,求常数$k$的值。
\end{enumerate}
\end{enumerate}



\section{本章小结}

\subsection{知识结构分析}
\subsubsection{数列的一般概念}
\begin{enumerate}
\item 定义:按一定顺序排列起来的一列数,叫做数列。
\item 数列表示法:列表法、图象法、解析法(即用通项公式)和递推式法。
\item 通项公式an与数列前n项和间的关系是:
\[a_n=\begin{cases}
    S_1,&n=1\\
    S_n-S_{n-1},&n\ge 2
\end{cases}\]
\item 数列分类.

按$n\; (n\in\N)$的取值范围分成有穷数列、无穷数列.

按相邻两项的大小关系来分,可分为:递增数列($a_{n+1}>a_n$),递减数列($a_{n+1}<a_n$),常数列($a_{n+1}=a_n$),摆动数列.
\end{enumerate}




\subsubsection{等差数列}
\begin{enumerate}
    \item 定义:从第2项起,每一项与它的前一项的差都等于同一个常数:$a_{n+1}-a_n=d$(常数). 也可以用递推关系式
    来定义:已知$a_1$及$a_{n+1}=a_n+d$($n\in\N$,$d$为常数)。
   \item 等差数列的通项公式:
    已知$a_1$和公差$d$时,$a_n=a_1+(n-1)d$.
    \item 等差数列前$n$项和的公式:
    $S_n=\frac{n(a_1+a_n)}{2}=na_1+\frac{n(n-1)}{2}d$
    \item 等差数列具有以下性质:
\begin{enumerate}[(1)]
\item 用图象法表示等差数列时,其各点均在以公差为斜率的一条直线。由此可得下述关系式:公差$d=\frac{a_n-a_m}{n-m}\; (n,m\in\N,\; n\ne m)$. 特别,当$m=1$时,$d=\frac{a_n-a_1}{n-1}\; (n\ne1)$.
    \item 等差数列的前$n$项中,与$a_1$和$a_n$距离相等的任何两项的和均等于$a_1+a_n$, 即
  \[  a_1+a_n=a_2+a_{n-1}=a_3+a_{n-2}=\cdots\]
    \item 等差数列中,若某两项的项数之和一定时,则相应两项的数值之和也是定值。即,若$k+\ell=m+n$, 则$a_k+a_{\ell}=a_m+a_n\; (k,\ell,m,n\in\N)$.
    \item 一个数列$\{a_n\}$是等差数列的充要条件是
\begin{enumerate}[(a)]
\item 通项公式是$a_n=dn+c$ ($d$、$c$为常数),且$n$的系数$d$就是等差数列的公差;
\item 前$n$项和公式为$S_n=an^2+bn$ ($a$、$b$是常数).
\item 从第2项起,任何一项都是它的前一项和它的后一项的等差中项。 
\end{enumerate}
\end{enumerate}
\end{enumerate}

\subsubsection{等比数列}
\begin{enumerate}
    \item 定义:从第2项起,每一项与它前一项的比都等于同一个常数,即$\frac{a_{n+1}}{a_n}=q$常数). 也可以用以下的递推关系式来表示:已知$a_1\ne 0$, $q\ne 0$,并$a_{n+1}=a_nq\; (n\in\N)$.
    \item 等比数列的通项公式:$a_n=a_1q^{n-1}$.
    \item 等比数列的前$n$项和公式:
\[S_n=\begin{cases}
    na_1,& q=1\\
    \frac{a_1(1-q^{n})}{1-q},& q\ne 1
\end{cases}\quad \text{或}\quad S_n=\begin{cases}
    na_1,& q=1\\
    \frac{a_1-a_nq}{1-q},& q\ne 1
\end{cases}\]
    \item 等比数列具有以下性质:
\begin{enumerate}[(1)]
\item 用图象法表示等比数列时,其各点均在函数$y=c\cdot q^n$的图象上,其中$q$为公比,$c=\frac{a_1}{q}$.
\item 等比数列的前$n$项中,与$a_1$和$a_n$距离相等的两项的乘积,都等于$a_1$与$a_n$的乘积,即:
    \[a_1a_n=a_2a_{n-1}=a_3a_{n-2}=\cdots\]
    \item 等比数列中,若某两项的项数之和一定时,则相应两项数值之积也是定值。即,若$k+\ell=m+n$,则
    \[a_k \cdot  a_{\ell}=a_m\cdot a_n\quad \text{(其中$k,\ell,m,n\in\N$)}\]
    \item 一个数列$\{a_n\}$是等比数列的充要条件是
\begin{enumerate}[(a)]
  \item 数列的通项公式为$a_n=c\cdot q^n\; (c\cdot q\ne 0)$.
\item 从第2项起,每一项都是它的前一项与后一项的等比中项。 
\end{enumerate}
\end{enumerate}
\end{enumerate}

\subsubsection{特殊数列求和问题}
除等差数列和等比数列以外,还应掌握求一些特殊数列前$n$项和的方法。

\begin{enumerate}
\item 可转化为等差(或等比)数列的求和问题:
\begin{enumerate}[(1)]
    \item $a_n=b_n+c_n$, 而$\{b_n\}$和$\{c_n\}$分别为等差或等比数列,于是可分别对$\{b_n\}$、$\{c_n\}$求和,再将结果进行加(或减),即可求得$\{a_n\}$的前$n$项和。
    \item $a_n=b_n\cdot c_n$,且$\{b_n\}$为等差数列,$\{c_n\}$为等比数列,可用推导等比数列前$n$项和公式的方法(简称错位相减法)求和,称为“差比数列”求和。
\end{enumerate}

\item 裂项抵消法。把数列的每一项都分裂为两项之差,再取和时,多数项互相抵消,只剩少数几项之代数和,极易求出其和。

例如$\{c_n\}$是等差数列,且$a_i\ne 0\; (i=1,2,\ldots,n)$. 设$b_n=\frac{1}{a_na_{n+1}}$,那么数列$\{b_n\}$的前$n$项和即可用裂项抵消法求得。
\item 形如通项公式为$a_n=an^2+bn+c\; (a\ne 0)$的数列的前$n$项和,可借助于前$n$个自然数的和与前$n$个自然数平方和的公式解决。
\end{enumerate}

\subsubsection{数学归纳法}
\begin{enumerate}
    \item 数学归纳法是一种证明与自然数$n$有关的数学命题的重要方法。用数学归纳法证明命题的步骤是
\begin{enumerate}[(1)]
\item 证明当$n$取第一个值$n_0$(例如$n_0=1$, $n_0=2$等)时结论正确;
\item 假设当$n=k$ ($k\in\N$,且$k\ge n_0$)时结论正确,证明当$n=k+1$时结论也正确。
\end{enumerate}

在完成了这两个步骤以后,就可以断定命题对于从$n_0$开始的所有自然数$n$都正确。

上面的第一步是递推的基础,第二步是递推的依据,两者缺一不可。
\item 用数学归纳法证明命题时,难点在第二步,即在假设$n=k$时命题成立,推出$n=k+1$时命题也成立。因此,在推导中,必须用到“归纳假设”。对于较为困难的问题,综合与分析可以同时运用。
\end{enumerate}


\subsubsection{数列的极限}
极限概念是微积分的最重要的、最基本的概念,极限概念和运算法则是研究微积分全部内容的重要工具。
\begin{enumerate}
    \item 数列极限的定义:

对于一个无穷数列$\{a_n\}$,如果存在一个常数$A$,对于无论预先指定多么小的正数$\varepsilon$,都能在数列中找到一项$a_N$,使得这一项后面所有的项与$A$的差的绝对值都小于$\varepsilon$(即当$n>N$时,$|a_n-A|<\varepsilon$恒成立),就把常数$A$叫做数列$\{a_n\}$的极限,记作$\Lim{n}{\infty}a_n=A$.

\item 掌握数列极限的定义必须深刻地理解以下几点:
\begin{enumerate}[(1)]
\item $\varepsilon$必须具有绝对的任意性;
\item $N$的存在性及对$\varepsilon$的依赖性;
\item 当$n>N$时,$|a_n-A|<\varepsilon$恒成立的意义。
\end{enumerate}

\item 数列极限的运算法则:

\begin{thm}{}
 如果$\Lim{n}{\infty}a_n=A$, $\Lim{n}{\infty}b_n=B$,那么,
 \[\begin{split}
     \Lim{n}{\infty}(a_n\pm b_n)&=A\pm B\\
     \Lim{n}{\infty}(a_n\cdot  b_n)&=A\cdot B\\
     \Lim{n}{\infty}\frac{a_n}{b_n}&=\frac{A}{B}\quad (B\ne 0,\; b_n\ne 0,\; n\in\N)\\
 \end{split}\]
\end{thm}

\begin{note}
数列的加、乘的极限运算法则能推广到(即适用于)任意有限个数列的情况。但要特别注意,不能推广到无限个数列的情况。
\end{note}

\item 无穷等比数列各项的和
\begin{enumerate}[(1)]
\item 定义:公比的绝对值小于1的无穷等比数列前$n$项的和当$n$无限增大时的极限,叫做这个无穷等比数列各项的和,并且用符号$S$表示。
\item 公式:$S=\frac{a_1}{1-q}\quad (|q|<1)$
\item 应用:可将循环小数化成分数。还可解决某些无穷变化的简单几何图形的面积或某些曲线(圆)的周长及所围成的面积问题。
\end{enumerate}
\end{enumerate}

\subsection{几点说明}

本章涉及下列重要的数学思想和方法。
\subsubsection{函数和方程的思想}
数列是一种特殊的函数,其特殊在于它的定义域是自然数集或是它的有限子集。

等差数列、等比数列的基本问题,就是已知$a_1$,$(d)q$,$n$,$a_n$,$S_n$中的任意3个量,求其它两个量。通过通项公式及前$n$项和公式组成的方程组来求解。

\subsubsection{极限的思想和方法}
由数列极限的定义可知,这里不是着眼于某一个数,而且研究一系列的数,不是静止地研究每一个数,而是研究它们的变化趋势,这是理解极限概念的关键所在。早在两千多年前,《庄子·天下篇》中就有“一尺之棰,日取其半,万世不竭”之语,这正是朴素的极限思想的体现。极限思想和方法是人们从有限中认识无限,从近似中认识精确,从量变中认识质变的一种辩证的思想。求无穷数列各项和的方法,用的就是极限的方法。

\subsubsection{归纳法与数学归纳法}
归纳法是科学的探索与发展不可少的方法,通过归纳法提出科学的猜想,猜想的结论未必可靠,然而没有猜想就没有前进。数学归纳法是一种完全归纳法,通过数学归纳法的
证明,可以保证某些猜想的合理性和正确性。

\section*{复习题五}
\begin{center}
    \bfseries A
\end{center}

\begin{enumerate}
    \item 选择题(有且只有一个正确答案)
\begin{enumerate}[(1)]
    \item 公比为$\frac{1}{2}$的等比数列一定是(\qquad )
\begin{multicols}{2}
\begin{enumerate}[(A)]
    \item 递增数列
    \item 递减数列
    \item 摆动数列
    \item 递增数列或递减数列
\end{enumerate}
\end{multicols}

\item $b^2=ac$是$a,b,c$三数成等比数列的(\qquad)
\begin{multicols}{2}
    \begin{enumerate}[(A)]
        \item 充分条件
        \item 必要条件
        \item 充要条件
        \item 不充分也不必要条件
    \end{enumerate}
    \end{multicols}

\item 两位自然数中,所有被7整除的数之和为(\qquad)
\begin{multicols}{4}
\begin{enumerate}[(A)]
    \item 726
    \item 728
    \item 730
    \item 735
\end{enumerate}
\end{multicols}

\item 某种细菌在培养过程中,每20分钟分裂一次(一个分裂成两个),经过3小时,这种细菌由1个可繁殖成(\qquad)
\begin{multicols}{2}
\begin{enumerate}[(A)]
    \item 511个
    \item 512个
    \item 1023个
    \item 1024个
\end{enumerate}
\end{multicols}

\item 等差数列$\{a_n\}$,$\{b_n\}$的前$n$项和分别为$S_n$与$T_n$,若$\frac{S_n}{T_n}=\frac{2n}{3n+1}$,则
$\Lim{n}{\infty}\frac{a_n}{b_n}$等于(\qquad )
\begin{multicols}{4}
\begin{enumerate}[(A)]
    \item 1
    \item $\frac{6}{3}$
    \item $\frac{2}{3}$
    \item $\frac{4}{9}$
\end{enumerate}
\end{multicols}

\end{enumerate}


\item     填空题
\begin{enumerate}[(1)]
    \item 数列$\{a_n\}$的前$n$项和$S_n=2n^2-3n+1$,则它的通项公式$a_n=\blank\blank$.
    \item 数列$\{a_n\}$的前$n$项和$S_n=\left(\frac{1}{3}\right)^{n-1}-1$,则它的通项公式$a_n=\blank\blank$.
    \item 等差数列中,$a_3+a_8+a_{13}+a_{18}=50$,则该数列的前20项之和为\blank\blank.
    \item 等比数列中,$a_6\cdot a_{25}+a_4\cdot a_{27}=4$,则该数列的前30项的积为\blank\blank.
    \item  已知等差数列中,$a_5=3$, $a_{15}=-7$,则数列的第20项$a_{20}=\blank\blank$.
    \item  数列$1,\; x,\; x^2,\;x^3,\;\ldots, x^{n-1},\ldots$的前$n$项之和$S_n=\blank\blank$.
    \item 已知等差数列中$a_n=A$, 则$S_{2n-1}=\blank\blank$.
    \item 已知等差数列$\{a_n\}$只有$n$项($n>10$),其前10项之和为100,后10项之和为160,则该数列各项之和$S_n=\blank\blank$.
\end{enumerate}
   
\item 解方程$\lg x+\lg x^2+\cdots+ \lg x^n=n^2+n$.

\item 有四个数,其中前三个数成等差数列,后三个数成等比数列,且第一数与第四数之和是37,第二数与第三数的和是36,求这四个数。
\item 
\begin{enumerate}[(1)]
 \item 三个数成递增的等比数列,其和为65,若小数减去1,大数减19,则所得三数成等差数列,求原来的三个数。
\item 在$a$和$b$之间插入3个数,这3个数之和为27,积为504,且这3数同原来的两数$a$、$b$构成等差数列,求这5个数。
\end{enumerate}

\item 已知$a$、$b$、$c$三数成等差数列,$x$、$y$、$z$三数成等比数列,且$x$、$y$、$z$均为正数,求证
\[(b-c)\log_m x+(c-a)\log_m y+(a-b)\log_m z=0\]
\item 已知$a$,$b$,$c$成等差数列,求证:
\begin{enumerate}[(1)]
\item $b+c$, $c+a$, $a+b$也成等差数列;
\item $a^2-bc$, $b^2-ac$, $c^2-ab$也成等差数列;
\item $a^2(b+c),\; b^2(c+a),\; c^2(a+b)$也成等差数列。
\end{enumerate}

\item 已知$a$、$b$、$c$、$d$成等比数列(公比$q\ne -1$)
求证:
\begin{enumerate}[(1)]
\item $ab$、$bc$,$cd$成等比数列;
\item $a+b,\; b+c,\; c+d$成等比数列;
\item $(a-d)^2=(b-c)^2+(c-a)^2+(b-d)^2$
\end{enumerate}
\item 求和:$S_n=\frac{1}{2}+\frac{3}{2^2}+\frac{5}{2^3}+\cdots+\frac{2n-1}{2^n}$
\item 已知数列$\{a_n\}$的前$n$项和$S_n=n(n+1)$,又知$b_n=a^2_n$,求数列$\{b_n\}$的前$n$项和$S'_n$.

\item 求数列$\frac{1}{5},\; \frac{2}{5^2},\; \frac{3}{5^3},\; \frac{1}{5^4},\; \frac{2}{5^5},\; \frac{3}{5^6},\; \frac{1}{5^7},\ldots$所有项的和.

\item 用数学归纳法证明:
\begin{enumerate}[(1)]
    \item $1\cdot 2\cdot 3+2\cdot 3\cdot 4+\cdots+n(n+1)(n+2)=\frac{1}{4}n(n+1)(n+2)(n+3)$
    \item $(a_1+a_2+\cdots+a_n)^2=a^2_1+a^2_2+\cdots +a^2_n+2(a_1a_2+a_1a_3+\cdots+a_{n-1}a_n)$
    \item $4^{2n+1}+3^{n+2}\; (n\in\N)$能被13整除;
    \item $6^{2n-1}+1\; (n\in\N)$能被7整除.
\end{enumerate}

\item 求下列极限:
\begin{multicols}{2}
\begin{enumerate}[(1)]
    \item $\Lim{n}{\infty}\frac{(n-1)(n+1)(n+2)}{2n^3}$
    \item $\Lim{n}{\infty}\frac{(2.1)^n-(1.9)^n}{3\cdot (2.1)^{n+1}}$
    \item $\Lim{n}{\infty}\frac{1+2+3+\cdots+n}{1+3+5+\cdots+(2n-1)}$
    \item $\Lim{n}{\infty}\left(n\sqrt{n^2+1}-n\sqrt{n^2-2}\right)$
\end{enumerate}
\end{multicols}

\item 已知$a>0$,求下列极限:
\begin{multicols}{2}
\begin{enumerate}[(1)]
    \item $\Lim{n}{\infty}\frac{1}{1+a^n}$
    \item $\Lim{n}{\infty}\frac{a^n}{1+a^n}$
\end{enumerate}
\end{multicols}

\item \begin{enumerate}[(1)]
\item 已知等比数列$\{a_n\}$的公比$q>1$,且$a_1=b\; (b\ne 0)$, 

求$\Lim{n}{\infty}\frac{a_1+a_2+\cdots+a_n}{a_6+a_7+\cdots+a_n}$
\item 已知等比数项$\{a_n\}$,如果$a_1+a_2+a_3=18$, $a_2+a_3+a_4=-9$,

求$\Lim{n}{\infty}(a_1+a_2+\cdots+a_n)$
\end{enumerate}

\item 已知$\sin\alpha$是$\sin\theta$和$\cos\theta$的等差中项,$\sin\theta,\; \sin\beta,\; \cos\theta$成等比数列,求证$2\cos2\alpha=\cos2\beta$.
\item 已知$a$、$b$、$c$成等比数列,$m$是$a$、$b$的等差中项,$n$是$b$、$c$的等差中项,求证$\frac{a}{m}+\frac{c}{n}=2$.
\item 已知$a$、$b$、$c$和$m$、$n$、$p$分别成等差数列,且$\frac{a}{m},\; \frac{b}{n},\; \frac{c}{p}$成等比数列,求证:$\frac{m}{p}+\frac{p}{m}=\frac{c}{a}+\frac{a}{c}$.

\item 已知数列$\{a_n\}$的通项公式为$a_n=n^2-11n+10$,从第几项起这个数列中的项都是正数?从第几项起各项都大于70?
\item 长方体的三条棱的长成等差数列,它的对角线的长是14cm,全面积是22${\rm cm^2}$,求它的体积。
\item 三角形的三个内角成等差数列,它的面积是$10\sqrt{3}{\rm cm}^2$, 周长是20cm,求三角形三边长。
\item 一个等差数列的首项是$-60$,其第十七项为$-12$,求此数列取绝对值后,所得数列的前30项的和。
\item 数列$\{a_n\}$的通项公式为$a_n=(n+1)\cdot \left(\frac{9}{10}\right)^n\; (n\in\N)$.
\begin{enumerate}[(1)]
\item 求证这个数列先增后减,
\item 当$n$为何值时的值最大,其值是多少?
\end{enumerate}

\item 已知数列$\{a_n\}$是等差数列,且各项为正,求证:
\[\frac{1}{\sqrt{a_1}+\sqrt{a_2}}+\frac{1}{\sqrt{a_2}+\sqrt{a_3}}+\cdots+\frac{1}{\sqrt{a_n}+\sqrt{a_{n+1}}}=\frac{n}{\sqrt{a_1}+\sqrt{a_{n+1}}}\]

\item 设$S_n=1^2+2^2+3^2-4^2+\cdots +(-1)^{n-1}n^2$,求证:$S_n$是连续$n$个整数的和.
\item 设数列$\{a_n\}$的前$n$项和为$\frac{n}{n+1}$,又$b_n=\frac{1}{a_n}$,试求数列$\{b_n\}$的前$n$项的和.
\item 用归纳法求数列
\[1,\; (1+2+1),\; (1+2+3+2+1),\ldots, [1+2+\cdots+n+(n-1)+\cdots+2+1],\ldots\]
的通项公式及前$n$项和的公式,并用数学归纳法予以证
明。
\item 设首项为1、公比为$q\; (q>
0)$的等比数列的前$n$项之
和为$S_n$. 求$\Lim{n}{\infty}\frac{S_n}{S_{n+1}}$.

\noindent
\begin{minipage}{.5\textwidth}
\item 如图,坐标平面上有一曲边三角形$OAB$,$A$、$B$坐标分别为$(1,0)$、$(1,1)$. 
$OA$、$AB$边是直线段,$OB$边是抛物线段:$y=x^2\; (0\le x \le 1)$
\begin{enumerate}[(1)]
    \item 过$OA$边的四等分点分别作$AB$的平行线,并作出相应的三个矩形,求这些矩形面积的和;
    \item 把$OA$边$n$等分,同样作出$n-1$个矩形,求这些矩形面积的和,并求当$n$无限增大时,这些矩形面积的极限。
\end{enumerate}    
\end{minipage}\hfill
\begin{minipage}{.45\textwidth}
     \centering
\begin{tikzpicture}[>=stealth, scale=3]
\draw[->](-.25,0)--(1.25,0)node[below]{$x$};
\draw[->](0,-.25)--(0,1.25)node[left]{$y$};
\draw[domain=0:1, smooth, very thick]plot(\x, \x*\x)node[right]{$B$};
\draw(1,0)node[above right]{$A$}--(1,1);
\node at (.5,0)[below]{$\frac{1}{2}$};
\node at (1,0)[below]{$1$};
\node [below left]{$O$};

\node at (0,.5)[left]{$\frac{1}{2}$};
\node at (0,1)[left]{$1$};

\foreach \x in {.5,1}
{
    \draw(0,\x)--(.03,\x);
}

\draw[pattern=north east lines](0.25,0) rectangle (0.5, 0.25*0.25);
\draw[pattern=north east lines](0.5,0) rectangle (0.75, 0.5*0.5);
\draw[pattern=north east lines](0.75,0) rectangle (1, 0.75*0.75);

\end{tikzpicture}
\captionof*{figure}{第29题}
\end{minipage}



\end{enumerate}
\chapter{复数}

\section{引言——数系的发展与复数的起源}
数的概念及运算是从生产和科学研究的实践中产生和发
展起来的。截至目前,同学们学过的数系是下表中实线画出
的部分:












正如表中所列出的,从正整数(自然数)集$\N$到实数集
$\R$, 数系已经经历了三次扩充(下表中实箭头表示的部分)。
本章将研究第四次扩充(虚箭头表示的部分):










“温故知新”,让我们先回顾一下前三次扩充的事实和规
律。

\subsection*{1.从数集上的元素看}
数集的每次扩充都是在原有数集的基础上引入新数,构
造出一个新数集,且使原有的数集成为新数集上的一个真子
集。也就是说,数集以\textbf{逐次包含}的方式扩充。

\subsection*{2.从数集上的运算看}
在新数集上我们这样定义新的运算法则:一方面要使新
法则运用到原有数集上时必须与原有数集上的运算结果相一
致;另一方面我们还希望新法则能使原数集上的运算规律保
持不变。例如,在有理数集上计算$(+5)+(+6)$
的结果
应与在算术(正有理数集)中计算$5+6$的结果相一致,而且
我们还希望在算术中对加法和乘法成立的交换律、结合律和
分配律能在有理数集上保持不变。也就是说,在原来数集上
成立的算律,在扩充后的数集上希望它能继续成立。这是很
重要的\textbf{数集扩充的原则}。


\subsection*{3.从数集扩充的动力看}
数集的每次扩充既是度量的需要(这一点大家比较清
楚),又是数学理论发展的需要(特别是解方程的需要)。
例如:

方程$x+4=0$
在自然数集上无解,而在扩充后的整数集
上有唯一解
$x=-4$;

方程$5x=3$,在整数集上无解,而在扩充后的有理数集
上有唯一解
$x=\frac{3}{5}$
(若不扩充数集,连这种最简单的一元一
次方程的理论也不能完整)。

方程$x^2=2$
在有理数集上无解,在扩充后的实数集上有
两个解$x=\pm\sqrt{2}$.

最早呼唤扩充实数集的问题是解二次方程。事实上,在
实数集上不能提供二次方程的完整理论。如方程
\begin{equation}
    x^2=-1 \tag{1}
\end{equation}
在实数集上就没有解,因为任意实数的平方都不可能为负
数。

摆在人们面前有两种选择:要么宣布方程(1)无解;要么
沿着数集扩充的方向朝前走——引入新数,扩充实数集,使
方程(1)有解。一些严峻的事实迫使人们选择了后者。于是在
16世纪中叶,人们开始引入一个由等式
\begin{equation}
    i^2=-1  \tag{2}
\end{equation}
所定义的新数$i$(它显然不是实数)\footnote{$i$是英文单词imaginares(虚的)的字头,瑞士大数学家欧拉首先用它
表示虚数单位。},它被称做虚数单位。
这样,方程(1)至少有了一个根$i$. 进而,根据数集扩充的原
则,我们还规定实数可以和它进行四则运算,并且进行四则
运算时,原有的加、乘算律仍然成立。于是自然造出了诸如
$2i$, $3i$, $-i$, $2+5i$
这样一些新数。

应该指出:引进的新数$i$以及上面造出的形如
$a+bi\; (a,b\in\R)$的数,起初人们对它感到迷惑不解,特别是当$b\ne 0$时,
认为这些都是“虚假的数”[这是因为当时人们习惯于把数作
为某种计数手段,然而形如$a+bi\; (a,
b\in\R, \text{ 且}b\ne 0)$的新数
却起不到计数手段的作用],“虚数”正是由此而得名。

虚数诞生后,除了用来表示方程的解以外,一时尚找不
到更多的应用,因此发展十分缓慢。到了18世纪末叶,维塞
尔(Wessel)、阿尔纲(Argand)和高斯(Gauss)几乎同
时给这些新数以几何解释(见6.2节)以后,才在数学、物
理的应用和研究中逐渐有了越来越多且越来越重要的用途,
这进一步推动了人们研究的兴趣。19世纪初,高斯把形如
$a+bi\; (a,b\in\R)$的数称为\textbf{复数}。如果说微积分的研究统治了
18世纪的话,那么,19世纪数学家们的兴奋中心便是关于复
数理论及其应用的研究。现在复数已经发展成为一门十分重
要的基础数学分支——复变函数论,它是数学、物理和技术
科学中有力的数学工具之一。

本章将学习复数的初步知识。









\subsection{利用三角形式进行复数的乘方运算}

现在我们研究$n$个复数相乘的问题。

设$z_k=r_k(\cos\theta_k+i\sin\theta_k),\quad k=1,2,3,\ldots,n$。利用数学归纳法容易证明
\[z_1\cdot z_2\cdots z_n=r_1\cdot r_2\cdots r_n\left[\pcx{\theta_1+\theta_2+\cdots+\theta_n}\right]\]

特别是当$z_1= z_2=\cdots= z_n$时,即
\[r_1=r2=\cdots=r_n=r,\qquad \theta_1=\theta_2=\cdots=\theta_n=\theta\]
代入上式就有
\[[r(\pc{\theta})]^n=r^n(\pc{n\theta})\quad (n\in\N)\]
这就是说,
\textbf{复数$z$的$n\; (n\in\N)$次幂,其模等于$z$的模的$n$次幂,其辐角等于$z$的辐角的$n$倍}。
这就是著名的\textbf{棣莫佛\footnote{棣莫佛(Abraham de Moivre)1667--1754年,法国数学家.}定理}。







\begin{example}
    
\end{example}

\begin{solution}
    
\end{solution}


\begin{example}
    
\end{example}

\begin{solution}
    
\end{solution}



\begin{example}
    
\end{example}

\begin{solution}
    
\end{solution}


\begin{example}
    
\end{example}

\begin{solution}
    
\end{solution}



\begin{example}
    
\end{example}

\begin{solution}
    
\end{solution}



\begin{example}
    
\end{example}

\begin{solution}
    
\end{solution}





\backmatter



\end{document}



\noindent
\begin{minipage}{.45\textwidth}
    \centering
\begin{tikzpicture}[>=stealth]

    
\end{tikzpicture}  
\captionof{figure}{}
\end{minipage}\hfill
\begin{minipage}{.45\textwidth}
\centering
\begin{tikzpicture}[>=stealth]


\end{tikzpicture}   
\captionof{figure}{} 
\end{minipage}











